
% Default to the notebook output style

    

% Default to the notebook output style


% Inherit from the specified cell style.




    
\documentclass[11pt]{article}

    
    
    \usepackage{graphicx} % Used to insert images
    \usepackage{adjustbox} % Used to constrain images to a maximum size 
    \usepackage{color} % Allow colors to be defined
    \usepackage{enumerate} % Needed for markdown enumerations to work
    \usepackage{geometry} % Used to adjust the document margins
    \usepackage{amsmath} % Equations
    \usepackage{amssymb} % Equations
    \usepackage{eurosym} % defines \euro
    \usepackage[mathletters]{ucs} % Extended unicode (utf-8) support
    \usepackage[utf8x]{inputenc} % Allow utf-8 characters in the tex document
    \usepackage{fancyvrb} % verbatim replacement that allows latex
    \usepackage{grffile} % extends the file name processing of package graphics 
                         % to support a larger range 
    % The hyperref package gives us a pdf with properly built
    % internal navigation ('pdf bookmarks' for the table of contents,
    % internal cross-reference links, web links for URLs, etc.)
    \usepackage{hyperref}
    \usepackage{longtable} % longtable support required by pandoc >1.10
    \usepackage{booktabs}  % table support for pandoc > 1.12.2
    \usepackage{ulem} % ulem is needed to support strikethroughs (\sout)
    

    
    
    \definecolor{orange}{cmyk}{0,0.4,0.8,0.2}
    \definecolor{darkorange}{rgb}{.71,0.21,0.01}
    \definecolor{darkgreen}{rgb}{.12,.54,.11}
    \definecolor{myteal}{rgb}{.26, .44, .56}
    \definecolor{gray}{gray}{0.45}
    \definecolor{lightgray}{gray}{.95}
    \definecolor{mediumgray}{gray}{.8}
    \definecolor{inputbackground}{rgb}{.95, .95, .85}
    \definecolor{outputbackground}{rgb}{.95, .95, .95}
    \definecolor{traceback}{rgb}{1, .95, .95}
    % ansi colors
    \definecolor{red}{rgb}{.6,0,0}
    \definecolor{green}{rgb}{0,.65,0}
    \definecolor{brown}{rgb}{0.6,0.6,0}
    \definecolor{blue}{rgb}{0,.145,.698}
    \definecolor{purple}{rgb}{.698,.145,.698}
    \definecolor{cyan}{rgb}{0,.698,.698}
    \definecolor{lightgray}{gray}{0.5}
    
    % bright ansi colors
    \definecolor{darkgray}{gray}{0.25}
    \definecolor{lightred}{rgb}{1.0,0.39,0.28}
    \definecolor{lightgreen}{rgb}{0.48,0.99,0.0}
    \definecolor{lightblue}{rgb}{0.53,0.81,0.92}
    \definecolor{lightpurple}{rgb}{0.87,0.63,0.87}
    \definecolor{lightcyan}{rgb}{0.5,1.0,0.83}
    
    % commands and environments needed by pandoc snippets
    % extracted from the output of `pandoc -s`
    \providecommand{\tightlist}{%
      \setlength{\itemsep}{0pt}\setlength{\parskip}{0pt}}
    \DefineVerbatimEnvironment{Highlighting}{Verbatim}{commandchars=\\\{\}}
    % Add ',fontsize=\small' for more characters per line
    \newenvironment{Shaded}{}{}
    \newcommand{\KeywordTok}[1]{\textcolor[rgb]{0.00,0.44,0.13}{\textbf{{#1}}}}
    \newcommand{\DataTypeTok}[1]{\textcolor[rgb]{0.56,0.13,0.00}{{#1}}}
    \newcommand{\DecValTok}[1]{\textcolor[rgb]{0.25,0.63,0.44}{{#1}}}
    \newcommand{\BaseNTok}[1]{\textcolor[rgb]{0.25,0.63,0.44}{{#1}}}
    \newcommand{\FloatTok}[1]{\textcolor[rgb]{0.25,0.63,0.44}{{#1}}}
    \newcommand{\CharTok}[1]{\textcolor[rgb]{0.25,0.44,0.63}{{#1}}}
    \newcommand{\StringTok}[1]{\textcolor[rgb]{0.25,0.44,0.63}{{#1}}}
    \newcommand{\CommentTok}[1]{\textcolor[rgb]{0.38,0.63,0.69}{\textit{{#1}}}}
    \newcommand{\OtherTok}[1]{\textcolor[rgb]{0.00,0.44,0.13}{{#1}}}
    \newcommand{\AlertTok}[1]{\textcolor[rgb]{1.00,0.00,0.00}{\textbf{{#1}}}}
    \newcommand{\FunctionTok}[1]{\textcolor[rgb]{0.02,0.16,0.49}{{#1}}}
    \newcommand{\RegionMarkerTok}[1]{{#1}}
    \newcommand{\ErrorTok}[1]{\textcolor[rgb]{1.00,0.00,0.00}{\textbf{{#1}}}}
    \newcommand{\NormalTok}[1]{{#1}}
    
    % Additional commands for more recent versions of Pandoc
    \newcommand{\ConstantTok}[1]{\textcolor[rgb]{0.53,0.00,0.00}{{#1}}}
    \newcommand{\SpecialCharTok}[1]{\textcolor[rgb]{0.25,0.44,0.63}{{#1}}}
    \newcommand{\VerbatimStringTok}[1]{\textcolor[rgb]{0.25,0.44,0.63}{{#1}}}
    \newcommand{\SpecialStringTok}[1]{\textcolor[rgb]{0.73,0.40,0.53}{{#1}}}
    \newcommand{\ImportTok}[1]{{#1}}
    \newcommand{\DocumentationTok}[1]{\textcolor[rgb]{0.73,0.13,0.13}{\textit{{#1}}}}
    \newcommand{\AnnotationTok}[1]{\textcolor[rgb]{0.38,0.63,0.69}{\textbf{\textit{{#1}}}}}
    \newcommand{\CommentVarTok}[1]{\textcolor[rgb]{0.38,0.63,0.69}{\textbf{\textit{{#1}}}}}
    \newcommand{\VariableTok}[1]{\textcolor[rgb]{0.10,0.09,0.49}{{#1}}}
    \newcommand{\ControlFlowTok}[1]{\textcolor[rgb]{0.00,0.44,0.13}{\textbf{{#1}}}}
    \newcommand{\OperatorTok}[1]{\textcolor[rgb]{0.40,0.40,0.40}{{#1}}}
    \newcommand{\BuiltInTok}[1]{{#1}}
    \newcommand{\ExtensionTok}[1]{{#1}}
    \newcommand{\PreprocessorTok}[1]{\textcolor[rgb]{0.74,0.48,0.00}{{#1}}}
    \newcommand{\AttributeTok}[1]{\textcolor[rgb]{0.49,0.56,0.16}{{#1}}}
    \newcommand{\InformationTok}[1]{\textcolor[rgb]{0.38,0.63,0.69}{\textbf{\textit{{#1}}}}}
    \newcommand{\WarningTok}[1]{\textcolor[rgb]{0.38,0.63,0.69}{\textbf{\textit{{#1}}}}}
    
    
    % Define a nice break command that doesn't care if a line doesn't already
    % exist.
    \def\br{\hspace*{\fill} \\* }
    % Math Jax compatability definitions
    \def\gt{>}
    \def\lt{<}
    % Document parameters
    \title{Teel\_Formation\_pmag}
    
    
    

    % Pygments definitions
    
\makeatletter
\def\PY@reset{\let\PY@it=\relax \let\PY@bf=\relax%
    \let\PY@ul=\relax \let\PY@tc=\relax%
    \let\PY@bc=\relax \let\PY@ff=\relax}
\def\PY@tok#1{\csname PY@tok@#1\endcsname}
\def\PY@toks#1+{\ifx\relax#1\empty\else%
    \PY@tok{#1}\expandafter\PY@toks\fi}
\def\PY@do#1{\PY@bc{\PY@tc{\PY@ul{%
    \PY@it{\PY@bf{\PY@ff{#1}}}}}}}
\def\PY#1#2{\PY@reset\PY@toks#1+\relax+\PY@do{#2}}

\expandafter\def\csname PY@tok@gd\endcsname{\def\PY@tc##1{\textcolor[rgb]{0.63,0.00,0.00}{##1}}}
\expandafter\def\csname PY@tok@gu\endcsname{\let\PY@bf=\textbf\def\PY@tc##1{\textcolor[rgb]{0.50,0.00,0.50}{##1}}}
\expandafter\def\csname PY@tok@gt\endcsname{\def\PY@tc##1{\textcolor[rgb]{0.00,0.27,0.87}{##1}}}
\expandafter\def\csname PY@tok@gs\endcsname{\let\PY@bf=\textbf}
\expandafter\def\csname PY@tok@gr\endcsname{\def\PY@tc##1{\textcolor[rgb]{1.00,0.00,0.00}{##1}}}
\expandafter\def\csname PY@tok@cm\endcsname{\let\PY@it=\textit\def\PY@tc##1{\textcolor[rgb]{0.25,0.50,0.50}{##1}}}
\expandafter\def\csname PY@tok@vg\endcsname{\def\PY@tc##1{\textcolor[rgb]{0.10,0.09,0.49}{##1}}}
\expandafter\def\csname PY@tok@vi\endcsname{\def\PY@tc##1{\textcolor[rgb]{0.10,0.09,0.49}{##1}}}
\expandafter\def\csname PY@tok@mh\endcsname{\def\PY@tc##1{\textcolor[rgb]{0.40,0.40,0.40}{##1}}}
\expandafter\def\csname PY@tok@cs\endcsname{\let\PY@it=\textit\def\PY@tc##1{\textcolor[rgb]{0.25,0.50,0.50}{##1}}}
\expandafter\def\csname PY@tok@ge\endcsname{\let\PY@it=\textit}
\expandafter\def\csname PY@tok@vc\endcsname{\def\PY@tc##1{\textcolor[rgb]{0.10,0.09,0.49}{##1}}}
\expandafter\def\csname PY@tok@il\endcsname{\def\PY@tc##1{\textcolor[rgb]{0.40,0.40,0.40}{##1}}}
\expandafter\def\csname PY@tok@go\endcsname{\def\PY@tc##1{\textcolor[rgb]{0.53,0.53,0.53}{##1}}}
\expandafter\def\csname PY@tok@cp\endcsname{\def\PY@tc##1{\textcolor[rgb]{0.74,0.48,0.00}{##1}}}
\expandafter\def\csname PY@tok@gi\endcsname{\def\PY@tc##1{\textcolor[rgb]{0.00,0.63,0.00}{##1}}}
\expandafter\def\csname PY@tok@gh\endcsname{\let\PY@bf=\textbf\def\PY@tc##1{\textcolor[rgb]{0.00,0.00,0.50}{##1}}}
\expandafter\def\csname PY@tok@ni\endcsname{\let\PY@bf=\textbf\def\PY@tc##1{\textcolor[rgb]{0.60,0.60,0.60}{##1}}}
\expandafter\def\csname PY@tok@nl\endcsname{\def\PY@tc##1{\textcolor[rgb]{0.63,0.63,0.00}{##1}}}
\expandafter\def\csname PY@tok@nn\endcsname{\let\PY@bf=\textbf\def\PY@tc##1{\textcolor[rgb]{0.00,0.00,1.00}{##1}}}
\expandafter\def\csname PY@tok@no\endcsname{\def\PY@tc##1{\textcolor[rgb]{0.53,0.00,0.00}{##1}}}
\expandafter\def\csname PY@tok@na\endcsname{\def\PY@tc##1{\textcolor[rgb]{0.49,0.56,0.16}{##1}}}
\expandafter\def\csname PY@tok@nb\endcsname{\def\PY@tc##1{\textcolor[rgb]{0.00,0.50,0.00}{##1}}}
\expandafter\def\csname PY@tok@nc\endcsname{\let\PY@bf=\textbf\def\PY@tc##1{\textcolor[rgb]{0.00,0.00,1.00}{##1}}}
\expandafter\def\csname PY@tok@nd\endcsname{\def\PY@tc##1{\textcolor[rgb]{0.67,0.13,1.00}{##1}}}
\expandafter\def\csname PY@tok@ne\endcsname{\let\PY@bf=\textbf\def\PY@tc##1{\textcolor[rgb]{0.82,0.25,0.23}{##1}}}
\expandafter\def\csname PY@tok@nf\endcsname{\def\PY@tc##1{\textcolor[rgb]{0.00,0.00,1.00}{##1}}}
\expandafter\def\csname PY@tok@si\endcsname{\let\PY@bf=\textbf\def\PY@tc##1{\textcolor[rgb]{0.73,0.40,0.53}{##1}}}
\expandafter\def\csname PY@tok@s2\endcsname{\def\PY@tc##1{\textcolor[rgb]{0.73,0.13,0.13}{##1}}}
\expandafter\def\csname PY@tok@nt\endcsname{\let\PY@bf=\textbf\def\PY@tc##1{\textcolor[rgb]{0.00,0.50,0.00}{##1}}}
\expandafter\def\csname PY@tok@nv\endcsname{\def\PY@tc##1{\textcolor[rgb]{0.10,0.09,0.49}{##1}}}
\expandafter\def\csname PY@tok@s1\endcsname{\def\PY@tc##1{\textcolor[rgb]{0.73,0.13,0.13}{##1}}}
\expandafter\def\csname PY@tok@ch\endcsname{\let\PY@it=\textit\def\PY@tc##1{\textcolor[rgb]{0.25,0.50,0.50}{##1}}}
\expandafter\def\csname PY@tok@m\endcsname{\def\PY@tc##1{\textcolor[rgb]{0.40,0.40,0.40}{##1}}}
\expandafter\def\csname PY@tok@gp\endcsname{\let\PY@bf=\textbf\def\PY@tc##1{\textcolor[rgb]{0.00,0.00,0.50}{##1}}}
\expandafter\def\csname PY@tok@sh\endcsname{\def\PY@tc##1{\textcolor[rgb]{0.73,0.13,0.13}{##1}}}
\expandafter\def\csname PY@tok@ow\endcsname{\let\PY@bf=\textbf\def\PY@tc##1{\textcolor[rgb]{0.67,0.13,1.00}{##1}}}
\expandafter\def\csname PY@tok@sx\endcsname{\def\PY@tc##1{\textcolor[rgb]{0.00,0.50,0.00}{##1}}}
\expandafter\def\csname PY@tok@bp\endcsname{\def\PY@tc##1{\textcolor[rgb]{0.00,0.50,0.00}{##1}}}
\expandafter\def\csname PY@tok@c1\endcsname{\let\PY@it=\textit\def\PY@tc##1{\textcolor[rgb]{0.25,0.50,0.50}{##1}}}
\expandafter\def\csname PY@tok@o\endcsname{\def\PY@tc##1{\textcolor[rgb]{0.40,0.40,0.40}{##1}}}
\expandafter\def\csname PY@tok@kc\endcsname{\let\PY@bf=\textbf\def\PY@tc##1{\textcolor[rgb]{0.00,0.50,0.00}{##1}}}
\expandafter\def\csname PY@tok@c\endcsname{\let\PY@it=\textit\def\PY@tc##1{\textcolor[rgb]{0.25,0.50,0.50}{##1}}}
\expandafter\def\csname PY@tok@mf\endcsname{\def\PY@tc##1{\textcolor[rgb]{0.40,0.40,0.40}{##1}}}
\expandafter\def\csname PY@tok@err\endcsname{\def\PY@bc##1{\setlength{\fboxsep}{0pt}\fcolorbox[rgb]{1.00,0.00,0.00}{1,1,1}{\strut ##1}}}
\expandafter\def\csname PY@tok@mb\endcsname{\def\PY@tc##1{\textcolor[rgb]{0.40,0.40,0.40}{##1}}}
\expandafter\def\csname PY@tok@ss\endcsname{\def\PY@tc##1{\textcolor[rgb]{0.10,0.09,0.49}{##1}}}
\expandafter\def\csname PY@tok@sr\endcsname{\def\PY@tc##1{\textcolor[rgb]{0.73,0.40,0.53}{##1}}}
\expandafter\def\csname PY@tok@mo\endcsname{\def\PY@tc##1{\textcolor[rgb]{0.40,0.40,0.40}{##1}}}
\expandafter\def\csname PY@tok@kd\endcsname{\let\PY@bf=\textbf\def\PY@tc##1{\textcolor[rgb]{0.00,0.50,0.00}{##1}}}
\expandafter\def\csname PY@tok@mi\endcsname{\def\PY@tc##1{\textcolor[rgb]{0.40,0.40,0.40}{##1}}}
\expandafter\def\csname PY@tok@kn\endcsname{\let\PY@bf=\textbf\def\PY@tc##1{\textcolor[rgb]{0.00,0.50,0.00}{##1}}}
\expandafter\def\csname PY@tok@cpf\endcsname{\let\PY@it=\textit\def\PY@tc##1{\textcolor[rgb]{0.25,0.50,0.50}{##1}}}
\expandafter\def\csname PY@tok@kr\endcsname{\let\PY@bf=\textbf\def\PY@tc##1{\textcolor[rgb]{0.00,0.50,0.00}{##1}}}
\expandafter\def\csname PY@tok@s\endcsname{\def\PY@tc##1{\textcolor[rgb]{0.73,0.13,0.13}{##1}}}
\expandafter\def\csname PY@tok@kp\endcsname{\def\PY@tc##1{\textcolor[rgb]{0.00,0.50,0.00}{##1}}}
\expandafter\def\csname PY@tok@w\endcsname{\def\PY@tc##1{\textcolor[rgb]{0.73,0.73,0.73}{##1}}}
\expandafter\def\csname PY@tok@kt\endcsname{\def\PY@tc##1{\textcolor[rgb]{0.69,0.00,0.25}{##1}}}
\expandafter\def\csname PY@tok@sc\endcsname{\def\PY@tc##1{\textcolor[rgb]{0.73,0.13,0.13}{##1}}}
\expandafter\def\csname PY@tok@sb\endcsname{\def\PY@tc##1{\textcolor[rgb]{0.73,0.13,0.13}{##1}}}
\expandafter\def\csname PY@tok@k\endcsname{\let\PY@bf=\textbf\def\PY@tc##1{\textcolor[rgb]{0.00,0.50,0.00}{##1}}}
\expandafter\def\csname PY@tok@se\endcsname{\let\PY@bf=\textbf\def\PY@tc##1{\textcolor[rgb]{0.73,0.40,0.13}{##1}}}
\expandafter\def\csname PY@tok@sd\endcsname{\let\PY@it=\textit\def\PY@tc##1{\textcolor[rgb]{0.73,0.13,0.13}{##1}}}

\def\PYZbs{\char`\\}
\def\PYZus{\char`\_}
\def\PYZob{\char`\{}
\def\PYZcb{\char`\}}
\def\PYZca{\char`\^}
\def\PYZam{\char`\&}
\def\PYZlt{\char`\<}
\def\PYZgt{\char`\>}
\def\PYZsh{\char`\#}
\def\PYZpc{\char`\%}
\def\PYZdl{\char`\$}
\def\PYZhy{\char`\-}
\def\PYZsq{\char`\'}
\def\PYZdq{\char`\"}
\def\PYZti{\char`\~}
% for compatibility with earlier versions
\def\PYZat{@}
\def\PYZlb{[}
\def\PYZrb{]}
\makeatother


    % Exact colors from NB
    \definecolor{incolor}{rgb}{0.0, 0.0, 0.5}
    \definecolor{outcolor}{rgb}{0.545, 0.0, 0.0}



    
    % Prevent overflowing lines due to hard-to-break entities
    \sloppy 
    % Setup hyperref package
    \hypersetup{
      breaklinks=true,  % so long urls are correctly broken across lines
      colorlinks=true,
      urlcolor=blue,
      linkcolor=darkorange,
      citecolor=darkgreen,
      }
    % Slightly bigger margins than the latex defaults
    
    \geometry{verbose,tmargin=1in,bmargin=1in,lmargin=1in,rmargin=1in}
    
    

    \begin{document}
    
    
    \maketitle
    \tableofcontents
    

    
    \section{Paleomagntism of the Teel
Formation}\label{paleomagntism-of-the-teel-formation}

    \subsection{Teel Volcanics Data
Analysis}\label{teel-volcanics-data-analysis}

    What follows is an analysis of the Ordovician-Silurian Teel Formation.
Most samples are basalts of Hirnantian age taken from flows to the east
of Khukh Davaa in 2014.

    \subsubsection{Import Modules}\label{import-modules}


    Write template file (no\_code.tpl) so that when the notebook is
converted to a latex (then pdf) it excludes the large code blocks. This
requires an additional arguement when using nbconveter and also requires
that adding tables of content term after the document begins; can also
add author. Can't include examples here because they affect the file
when in latex.


    \begin{Verbatim}[commandchars=\\\{\}]
Overwriting no\_code\_latex.tplx
    \end{Verbatim}

    \subsubsection{Sampling localities}\label{sampling-localities}

    Table of site locality coordeinates in WGS84. Exact stratigraphic
positions are shown in main text; order of stratigraphic position is
given in ``stat\_pos'' column of table.


            \begin{Verbatim}[commandchars=\\\{\}]
{\color{outcolor}Out[{\color{outcolor}2}]:}    er\_citation\_names er\_location\_name er\_site\_name  site\_lat  site\_lon  \textbackslash{}
        0         This study          unknown          Z30  47.10038  95.37550   
        1         This study          unknown          Z31  47.10049  95.37604   
        2         This study          unknown          Z32  47.10094  95.37684   
        3         This study          unknown          Z33  47.10107  95.37705   
        4         This study          unknown          Z34  47.10111  95.37712   
        5         This study          unknown          Z35  47.10069  95.37747   
        6         This study          unknown          Z36  47.10221  95.37959   
        7         This study          unknown          Z37  47.10211  95.37971   
        8         This study          unknown          Z38  47.09855  95.38445   
        9         This study          unknown          Z39  47.09860  95.38467   
        10        This study          unknown          Z40  47.09859  95.38474   
        11        This study          unknown          Z41  47.10109  95.37744   
        12        This study          unknown          Z42  47.09577  95.38577   
        13        This study          unknown          Z43  47.09570  95.38638   
        14        This study          unknown          Z44  47.09571  95.38651   
        15        This study          unknown          Z45  47.09562  95.38676   
        16        This study          unknown          Z46  47.09563  95.38692   
        17        This study          unknown          Z47  47.09568  95.38727   
        18        This study          unknown          Z48  47.09570  95.38744   
        19        This study          unknown          Z49  47.09581  95.38747   
        20        This study          unknown          Z50  47.09575  95.38781   
        21        This study          unknown          Z51  47.09584  95.38802   
        22        This study          unknown          Z52  47.09583  95.38815   
        23        This study          unknown          Z53  47.09442  95.37205   
        24        This study          unknown          Z54  47.09502  95.37299   
        25        This study          unknown          Z55  47.09525  95.37351   
        26        This study          unknown          Z56  47.06403  95.42075   
        27        This study          unknown          Z57  47.06277  95.42039   
        28        This study          unknown          Z58  47.06277  95.42045   
        
            strat\_pos  
        0           4  
        1           5  
        2           6  
        3           7  
        4           8  
        5           9  
        6          11  
        7          12  
        8          13  
        9          14  
        10         15  
        11         10  
        12         16  
        13         17  
        14         18  
        15         19  
        16         20  
        17         21  
        18         22  
        19         23  
        20         24  
        21         25  
        22         26  
        23          1  
        24          2  
        25          3  
        26         27  
        27         28  
        28         29  
\end{Verbatim}
        

    \begin{center}
    \adjustimage{max size={0.9\linewidth}{0.9\paperheight}}{Teel_Formation_pmag_files/Teel_Formation_pmag_10_0.png}
    \end{center}
    { \hspace*{\fill} \\}
    
    The paleomagnetic data from these sites may need to be tilt-corrected
(given the age of magnetization) according to nearby measurements of
bedding. The bedding measurements used for tilt-corrections are shown
below.


            \begin{Verbatim}[commandchars=\\\{\}]
{\color{outcolor}Out[{\color{outcolor}4}]:}               sample\_bed\_dip  sample\_bed\_dip\_direction
        er\_site\_name                                          
        Z30                       58                        88
        Z31                       58                        88
        Z32                       55                        84
        Z33                       55                        84
        Z34                       55                        84
        Z35                       55                        84
        Z36                       47                        89
        Z37                       47                        89
        Z38                       46                        87
        Z39                       46                        87
        Z40                       46                        87
        Z41                       55                        84
        Z42                       37                        87
        Z43                       37                        87
        Z44                       37                        87
        Z45                       37                        87
        Z46                       37                        87
        Z47                       37                        87
        Z48                       28                        91
        Z49                       28                        91
        Z50                       28                        91
        Z51                       28                        91
        Z52                       28                        91
        Z53                       58                        88
        Z54                       58                        88
        Z55                       58                        88
        Z56                       24                       165
        Z57                       24                       165
        Z58                       24                       165
\end{Verbatim}
        
    The bedding measurements for flows Z56, Z57, and Z58 was measured from
small lenses of sedimentary rocks between flows (\textbf{or from flow
banding????}). We average these measurements by taking a fisher mean of
the poles derived from the bedding plane measurements.



            \begin{Verbatim}[commandchars=\\\{\}]
{\color{outcolor}Out[{\color{outcolor}6}]:} \{'alpha95': 11.477330440177056,
         'csd': 7.5069268730952095,
         'dec': 345.07072586523549,
         'inc': 66.37591972421707,
         'k': 116.42484465586055,
         'n': 3,
         'r': 2.9828215360225578\}
\end{Verbatim}
        

            \begin{Verbatim}[commandchars=\\\{\}]
{\color{outcolor}Out[{\color{outcolor}7}]:} \{'alpha95': 180.0,
         'csd': 75.717272007537375,
         'dec': 44.552429367843892,
         'inc': 50.036988188443182,
         'k': 1.1444059858078446,
         'n': 8,
         'r': 1.8832895958171265\}
\end{Verbatim}
        
    For the first three measurements the bedding has as average So
(strike/dip, right-hand rule) of 75/24, or DD-D (dip direction-dip) of
165-24. For all of the measurements there is an average bedding with an
So of 104/24, or DD-D of 194-24.

We belief the first three bedding measurements are the most
representative for the outcrop panel and are applied to the table above.

    \subsubsection{Principal-component analysis of
data}\label{principal-component-analysis-of-data}

    Below we import paleomangetic results that were analyzed using
demag\_gui.py from the PmagPy python package. These are the vector
component fits to all Teel sample data, including components from all
temperature ranges.


    We will go through each site, Z30 through Z58, and calculate site mean
directions from vector fits of demagnetization data, including fits from
all temperture ranges. Components have been classified according to
their relative temperature ranges. `LOW' components are typically below
200ºC, `MAG' refers to a temperature range within the unblocking range
of magnetite (up to 580ºC), `HEM' refers to vector components fit to
data points in the unblocking range of hematite (up to 680ºC), and `MID'
refers to components with temperature ranges between `LOW' and `MAG'.

    \paragraph{Z30}\label{z30}

    Site Z30 was sampled in a rhyolite at the base of the Teel Formation
stratigraphic section.



    \begin{center}
    \adjustimage{max size={0.9\linewidth}{0.9\paperheight}}{Teel_Formation_pmag_files/Teel_Formation_pmag_25_0.png}
    \end{center}
    { \hspace*{\fill} \\}
    
    \paragraph{Z31}\label{z31}

    Only magnetite and low temperature, LOW (less than 200ºC), components
for Z31.



    \begin{center}
    \adjustimage{max size={0.9\linewidth}{0.9\paperheight}}{Teel_Formation_pmag_files/Teel_Formation_pmag_29_0.png}
    \end{center}
    { \hspace*{\fill} \\}
    
    \paragraph{Z32}\label{z32}

    Only magnetite and low temperature, LOW (less than 200ºC), components
for Z32.



    \begin{center}
    \adjustimage{max size={0.9\linewidth}{0.9\paperheight}}{Teel_Formation_pmag_files/Teel_Formation_pmag_33_0.png}
    \end{center}
    { \hspace*{\fill} \\}
    
    \paragraph{Z33}\label{z33}

    Only magnetite and low temperature, LOW (less than 200ºC), components
for Z33.



    \begin{center}
    \adjustimage{max size={0.9\linewidth}{0.9\paperheight}}{Teel_Formation_pmag_files/Teel_Formation_pmag_37_0.png}
    \end{center}
    { \hspace*{\fill} \\}
    
    \paragraph{Z34}\label{z34}

    Only magnetite and low temperature, LOW (less than 200ºC), components
for Z34.



    \begin{center}
    \adjustimage{max size={0.9\linewidth}{0.9\paperheight}}{Teel_Formation_pmag_files/Teel_Formation_pmag_41_0.png}
    \end{center}
    { \hspace*{\fill} \\}
    
    \paragraph{Z35}\label{z35}

    Hematite, magnetite, and low temperature, LOW (less than 200ºC),
components were calculated for Z35.



    \begin{center}
    \adjustimage{max size={0.9\linewidth}{0.9\paperheight}}{Teel_Formation_pmag_files/Teel_Formation_pmag_45_0.png}
    \end{center}
    { \hspace*{\fill} \\}
    
    Data points shown with triangles were vectors from sample Z35.3 that
were dropped from the mean calculation.

    \paragraph{Z36}\label{z36}

    Only magnetite and low temperature, LOW (less than 200ºC), components
were calculated for Z36.



    \begin{center}
    \adjustimage{max size={0.9\linewidth}{0.9\paperheight}}{Teel_Formation_pmag_files/Teel_Formation_pmag_50_0.png}
    \end{center}
    { \hspace*{\fill} \\}
    
    Magnetite and low temperature, LOW (less than 200ºC), components were
calculated for Z37.



    \begin{center}
    \adjustimage{max size={0.9\linewidth}{0.9\paperheight}}{Teel_Formation_pmag_files/Teel_Formation_pmag_53_0.png}
    \end{center}
    { \hspace*{\fill} \\}
    
    \paragraph{Z38}\label{z38}

    Hematite, magnetite, and low temperature, LOW (less than 200ºC),
components were calculated for Z38.



    \begin{center}
    \adjustimage{max size={0.9\linewidth}{0.9\paperheight}}{Teel_Formation_pmag_files/Teel_Formation_pmag_57_0.png}
    \end{center}
    { \hspace*{\fill} \\}
    
    Only two samples yielded hematite components, therefore no mean was
calculated for the hematite component.

    \paragraph{Z39}\label{z39}

    Hematite, magnetite, and low temperature, LOW (less than 200ºC),
components were calculated for Z39.



    \begin{center}
    \adjustimage{max size={0.9\linewidth}{0.9\paperheight}}{Teel_Formation_pmag_files/Teel_Formation_pmag_62_0.png}
    \end{center}
    { \hspace*{\fill} \\}
    
    \paragraph{Z40}\label{z40}

    Hematite, magnetite, and low temperature, LOW (less than 200ºC),
components were calculated for Z40.



    \begin{center}
    \adjustimage{max size={0.9\linewidth}{0.9\paperheight}}{Teel_Formation_pmag_files/Teel_Formation_pmag_66_0.png}
    \end{center}
    { \hspace*{\fill} \\}
    
    \paragraph{Z41}\label{z41}



    \begin{center}
    \adjustimage{max size={0.9\linewidth}{0.9\paperheight}}{Teel_Formation_pmag_files/Teel_Formation_pmag_69_0.png}
    \end{center}
    { \hspace*{\fill} \\}
    
    \paragraph{Z42}\label{z42}

    Hematite, magnetite, and low temperature, LOW (less than 200ºC),
components were calculated for Z42. A middle temperature component, MID,
is also calculated, which derives from demagnetization steps between LOW
and magnetite.



    \begin{center}
    \adjustimage{max size={0.9\linewidth}{0.9\paperheight}}{Teel_Formation_pmag_files/Teel_Formation_pmag_73_0.png}
    \end{center}
    { \hspace*{\fill} \\}
    
    \paragraph{Z43}\label{z43}

    Hematite, magnetite, and low temperature, LOW (less than 200ºC),
components were calculated for Z43. A middle temperature component, MID,
is also calculated, which derives from demagnetization steps between LOW
and magnetite.



    \begin{center}
    \adjustimage{max size={0.9\linewidth}{0.9\paperheight}}{Teel_Formation_pmag_files/Teel_Formation_pmag_77_0.png}
    \end{center}
    { \hspace*{\fill} \\}
    
    \paragraph{Z44}\label{z44}

    Hematite, magnetite, and low temperature, LOW (less than 200ºC),
components were calculated for Z44.



    \begin{center}
    \adjustimage{max size={0.9\linewidth}{0.9\paperheight}}{Teel_Formation_pmag_files/Teel_Formation_pmag_81_0.png}
    \end{center}
    { \hspace*{\fill} \\}
    
    \paragraph{Z45}\label{z45}

    Hematite, magnetite, and low temperature, LOW (less than 200ºC),
components were calculated for Z45.



    \begin{center}
    \adjustimage{max size={0.9\linewidth}{0.9\paperheight}}{Teel_Formation_pmag_files/Teel_Formation_pmag_85_0.png}
    \end{center}
    { \hspace*{\fill} \\}
    
    \paragraph{Z46}\label{z46}

    Hematite, magnetite, and low temperature, LOW (less than 200ºC),
components were calculated for Z46.



    \begin{center}
    \adjustimage{max size={0.9\linewidth}{0.9\paperheight}}{Teel_Formation_pmag_files/Teel_Formation_pmag_89_0.png}
    \end{center}
    { \hspace*{\fill} \\}
    
    \paragraph{Z47}\label{z47}

    Hematite, magnetite, and low temperature, LOW (less than 200ºC),
components were calculated for Z47.



    \begin{center}
    \adjustimage{max size={0.9\linewidth}{0.9\paperheight}}{Teel_Formation_pmag_files/Teel_Formation_pmag_93_0.png}
    \end{center}
    { \hspace*{\fill} \\}
    
    \paragraph{Z48}\label{z48}

    Hematite, magnetite, and low temperature, LOW (less than 200ºC),
components were calculated for Z48.



    \begin{center}
    \adjustimage{max size={0.9\linewidth}{0.9\paperheight}}{Teel_Formation_pmag_files/Teel_Formation_pmag_97_0.png}
    \end{center}
    { \hspace*{\fill} \\}
    
    \paragraph{Z49}\label{z49}

    Hematite, magnetite, and low temperature, LOW (less than 200ºC),
components were calculated for Z49.



    \begin{center}
    \adjustimage{max size={0.9\linewidth}{0.9\paperheight}}{Teel_Formation_pmag_files/Teel_Formation_pmag_101_0.png}
    \end{center}
    { \hspace*{\fill} \\}
    
    \paragraph{Z50}\label{z50}

    Only magnetite and hematite components could be distinguished from the
demagnetization data or flow Z50.



    \begin{center}
    \adjustimage{max size={0.9\linewidth}{0.9\paperheight}}{Teel_Formation_pmag_files/Teel_Formation_pmag_105_0.png}
    \end{center}
    { \hspace*{\fill} \\}
    
    \paragraph{Z51}\label{z51}

    Hematite, magnetite, and low temperature, LOW (less than 200ºC),
components were calculated for Z51.



    \begin{center}
    \adjustimage{max size={0.9\linewidth}{0.9\paperheight}}{Teel_Formation_pmag_files/Teel_Formation_pmag_109_0.png}
    \end{center}
    { \hspace*{\fill} \\}
    
    Magnetite components from two samples (Z51.1 and Z51.2) were excluded
because of their similarity to hematite components (the hematite
remanence mixed with that of magnetite) and different demagnetization
behavior compared to the other magnetite components.

    \paragraph{Z52}\label{z52}

    Hematite, magnetite, and low temperature, LOW (less than 200ºC),
components were calculated for Z52.



    \begin{center}
    \adjustimage{max size={0.9\linewidth}{0.9\paperheight}}{Teel_Formation_pmag_files/Teel_Formation_pmag_114_0.png}
    \end{center}
    { \hspace*{\fill} \\}
    
    \paragraph{Z53}\label{z53}

    Hematite, magnetite, and low temperature, LOW (less than 200ºC),
components were calculated for Z53.



    \begin{center}
    \adjustimage{max size={0.9\linewidth}{0.9\paperheight}}{Teel_Formation_pmag_files/Teel_Formation_pmag_118_0.png}
    \end{center}
    { \hspace*{\fill} \\}
    
    \paragraph{Z54}\label{z54}

    Hematite, magnetite, and low temperature, LOW (less than 200ºC),
components were calculated for Z54.



    \begin{center}
    \adjustimage{max size={0.9\linewidth}{0.9\paperheight}}{Teel_Formation_pmag_files/Teel_Formation_pmag_122_0.png}
    \end{center}
    { \hspace*{\fill} \\}
    
    \paragraph{Z55}\label{z55}

    Magnetite and low-temperature, LOW (less than 200ºC), components were
calculated for Z55.



    \begin{center}
    \adjustimage{max size={0.9\linewidth}{0.9\paperheight}}{Teel_Formation_pmag_files/Teel_Formation_pmag_126_0.png}
    \end{center}
    { \hspace*{\fill} \\}
    
    \paragraph{Z56}\label{z56}

    Hematite, magnetite, and low temperature, LOW (less than 200ºC),
components were calculated for Z56.



    \begin{center}
    \adjustimage{max size={0.9\linewidth}{0.9\paperheight}}{Teel_Formation_pmag_files/Teel_Formation_pmag_130_0.png}
    \end{center}
    { \hspace*{\fill} \\}
    
    The directions from flow Z56 are very different from the rest of the
sites. This may be due to the fact that these flows are from a different
outcrop panel to the southeast of the majority of sites. The tilt
correction for this panel may have led to these differences. It is also
possible that the differences in directions is due to different
(younger) age of flows Z56, Z57, and Z58. The magnetite and hematite
components are the same, perhaps \emph{the result of overprinted
magnetite that yield the same directions as hematite components}.

    \paragraph{Z57}\label{z57}

    Hematite, magnetite, and low temperature, LOW (less than 200ºC),
components were calculated for Z57.



    \begin{center}
    \adjustimage{max size={0.9\linewidth}{0.9\paperheight}}{Teel_Formation_pmag_files/Teel_Formation_pmag_135_0.png}
    \end{center}
    { \hspace*{\fill} \\}
    
    Similar story to flow Z56 - with stark similarities between the
magnetite and hematite components.

    \paragraph{Z58}\label{z58}

    Magnetite, mid-, and low- temperature, LOW (less than 200ºC), components
were calculated for Z58. The middle temperature component derives from
demagnetization steps between LOW and magnetite.



    \begin{center}
    \adjustimage{max size={0.9\linewidth}{0.9\paperheight}}{Teel_Formation_pmag_files/Teel_Formation_pmag_140_0.png}
    \end{center}
    { \hspace*{\fill} \\}
    
    Results from flow Z58 are very different from all of the other sites.

    \subsubsection{Paleomagnetic data
summary}\label{paleomagnetic-data-summary}

    Create tables, distinguished by component type, of mean directions for
all Teel flows.

    \paragraph{Magnetitie directions}\label{magnetitie-directions}

    \subparagraph{Geographic coordinates -
magnetite}\label{geographic-coordinates---magnetite}


            \begin{Verbatim}[commandchars=\\\{\}]
{\color{outcolor}Out[{\color{outcolor}67}]:}              strat\_pos  site\_lat  site\_lon     dec\_geo    inc\_geo    alpha95  \textbackslash{}
         Z30\_mag\_geo          4  47.10038  95.37550   61.199035 -73.861110  11.712582   
         Z31\_mag\_geo          5  47.10049  95.37604  168.103623 -72.234668   3.753459   
         Z32\_mag\_geo          6  47.10094  95.37684  170.358827 -65.228366   7.987595   
         Z33\_mag\_geo          7  47.10107  95.37705  184.611118 -78.580341   4.579774   
         Z34\_mag\_geo          8  47.10111  95.37712  165.710954 -76.241270   2.846954   
         Z35\_mag\_geo          9  47.10069  95.37747  180.035625 -56.117492   6.392221   
         Z36\_mag\_geo         11  47.10221  95.37959  184.602274 -73.483650   4.160116   
         Z37\_mag\_geo         12  47.10211  95.37971  172.808147 -74.694707   7.476051   
         Z38\_mag\_geo         13  47.09855  95.38445  170.143849 -68.504682   8.040139   
         Z39\_mag\_geo         14  47.09860  95.38467  196.103583 -56.022026   6.360400   
         Z40\_mag\_geo         15  47.09859  95.38474  171.882175 -71.414505   4.837145   
         Z41\_mag\_geo         10  47.10109  95.37744  175.528998 -59.942821   4.580179   
         Z42\_mag\_geo         16  47.09577  95.38577  196.391432 -44.308602  11.150650   
         Z43\_mag\_geo         17  47.09570  95.38638  188.500123 -36.936066  10.749181   
         Z44\_mag\_geo         18  47.09571  95.38651  177.320106  25.952966  12.333051   
         Z45\_mag\_geo         19  47.09562  95.38676  185.281245  22.577189   8.142853   
         Z46\_mag\_geo         20  47.09563  95.38692  209.744992 -58.036511   6.728463   
         Z47\_mag\_geo         21  47.09568  95.38727  208.133979 -59.587799   5.606451   
         Z48\_mag\_geo         22  47.09570  95.38744  207.921657 -49.345442   6.760918   
         Z49\_mag\_geo         23  47.09581  95.38747  213.177701 -57.526629   9.488165   
         Z50\_mag\_geo         24  47.09575  95.38781  183.717358  16.532802   3.910452   
         Z51\_mag\_geo         25  47.09584  95.38802  206.717692 -54.030732  10.393892   
         Z52\_mag\_geo         26  47.09583  95.38815  182.835182 -55.213178  12.719642   
         Z53\_mag\_geo          1  47.09442  95.37205  172.278467 -71.983776   3.594132   
         Z54\_mag\_geo          2  47.09502  95.37299  170.318109 -69.453189   8.193477   
         Z55\_mag\_geo          3  47.09525  95.37351  166.386553 -64.854630   5.625088   
         Z56\_mag\_geo         27  47.06403  95.42075  179.680351  -4.756236   5.057226   
         Z57\_mag\_geo         28  47.06277  95.42039  178.786826  -8.456342  13.052562   
         Z58\_mag\_geo         29  47.06277  95.42045  128.167765 -19.352349  17.533117   
         
                       n           k         r        csd  paleolatitude    vgp\_lat  \textbackslash{}
         Z30\_mag\_geo   7   27.513985  6.781929  15.442168     -59.940264 -28.018244   
         Z31\_mag\_geo   8  218.756351  7.968001   5.476520     -57.348450 -77.463763   
         Z32\_mag\_geo   9   42.504477  8.811785  12.424178     -47.295204 -83.450449   
         Z33\_mag\_geo   8  147.251468  7.952462   6.675060     -68.002034 -68.966892   
         Z34\_mag\_geo  10  288.897760  9.968847   4.765549     -63.908090 -71.445651   
         Z35\_mag\_geo   6  110.823428  5.954883   7.694302     -36.670242 -79.569519   
         Z36\_mag\_geo   8  178.256399  7.960731   6.066839     -59.330031 -77.472930   
         Z37\_mag\_geo   9   48.384817  8.834659  11.644758     -61.306496 -75.207168   
         Z38\_mag\_geo  10   37.061501  9.757160  13.305265     -51.774894 -82.077055   
         Z39\_mag\_geo   6  111.925588  5.955327   7.656324     -36.571555 -74.094885   
         Z40\_mag\_geo   9  114.250096  8.929978   7.578037     -56.078745 -79.716310   
         Z41\_mag\_geo   8  147.225628  7.952454   6.675646     -40.828095 -82.952676   
         Z42\_mag\_geo   8   25.632057  7.726904  15.999017     -26.015899 -65.259739   
         Z43\_mag\_geo   8   27.509935  7.745546  15.443305     -20.601392 -62.620370   
         Z44\_mag\_geo   5   39.443485  4.898589  12.897258      13.677537 -29.179269   
         Z45\_mag\_geo   8   47.231012  7.851792  11.786134      11.744294 -30.970825   
         Z46\_mag\_geo   8   68.732148  7.898155   9.770236     -38.705303 -66.823872   
         Z47\_mag\_geo   8   98.575940  7.928989   8.158298     -40.424813 -68.748397   
         Z48\_mag\_geo   8   68.082983  7.897184   9.816705     -30.209549 -62.672462   
         Z49\_mag\_geo   7   41.429419  6.855175  12.584345     -38.154890 -64.234334   
         Z50\_mag\_geo   8  201.619149  7.965281   5.704520       8.442107 -34.363744   
         Z51\_mag\_geo   5   55.146896  4.927466  10.907481     -34.565590 -66.394972   
         Z52\_mag\_geo   6   28.697188  5.825767  15.120472     -35.744789 -78.453731   
         Z53\_mag\_geo   8  238.494787  7.970649   5.245001     -56.957385 -79.068440   
         Z54\_mag\_geo   8   46.660914  7.849982  11.857917     -53.143586 -81.346675   
         Z55\_mag\_geo   8   97.930102  7.928520   8.185155     -46.807981 -80.714545   
         Z56\_mag\_geo   6  176.489683  5.971670   6.097129      -2.382222 -45.317329   
         Z57\_mag\_geo   6   27.299686  5.816848  15.502659      -4.251323 -47.175716   
         Z58\_mag\_geo   6   15.552391  5.678506  20.539338      -9.960183 -32.768009   
         
                         vgp\_lon  vgp\_lat\_rev  vgp\_lon\_rev  
         Z30\_mag\_geo  245.559596    28.018244    65.559596  
         Z31\_mag\_geo  244.552326    77.463763    64.552326  
         Z32\_mag\_geo  190.612182    83.450449    10.612182  
         Z33\_mag\_geo  280.189909    68.966892   100.189909  
         Z34\_mag\_geo  255.430852    71.445651    75.430852  
         Z35\_mag\_geo   95.219636    79.569519   275.219636  
         Z36\_mag\_geo  286.256471    77.472930   106.256471  
         Z37\_mag\_geo  261.763513    75.207168    81.763513  
         Z38\_mag\_geo  225.175390    82.077055    45.175390  
         Z39\_mag\_geo   41.007252    74.094885   221.007252  
         Z40\_mag\_geo  249.190583    79.716310    69.190583  
         Z41\_mag\_geo  124.113660    82.952676   304.113660  
         Z42\_mag\_geo   58.086778    65.259739   238.086778  
         Z43\_mag\_geo   77.877325    62.620370   257.877325  
         Z44\_mag\_geo   98.369133    29.179269   278.369133  
         Z45\_mag\_geo   89.353703    30.970825   269.353703  
         Z46\_mag\_geo   15.721243    66.823872   195.721243  
         Z47\_mag\_geo   13.361225    68.748397   193.361225  
         Z48\_mag\_geo   33.565359    62.672462   213.565359  
         Z49\_mag\_geo   13.523120    64.234334   193.523120  
         Z50\_mag\_geo   90.931921    34.363744   270.931921  
         Z51\_mag\_geo   27.782142    66.394972   207.782142  
         Z52\_mag\_geo   83.817965    78.453731   263.817965  
         Z53\_mag\_geo  252.646197    79.068440    72.646197  
         Z54\_mag\_geo  233.270089    81.346675    53.270089  
         Z55\_mag\_geo  188.596346    80.714545     8.596346  
         Z56\_mag\_geo   95.874936    45.317329   275.874936  
         Z57\_mag\_geo   97.200362    47.175716   277.200362  
         Z58\_mag\_geo  162.478280    32.768009   342.478280  
\end{Verbatim}
        

    \begin{center}
    \adjustimage{max size={0.9\linewidth}{0.9\paperheight}}{Teel_Formation_pmag_files/Teel_Formation_pmag_147_0.png}
    \end{center}
    { \hspace*{\fill} \\}
    

    \begin{center}
    \adjustimage{max size={0.9\linewidth}{0.9\paperheight}}{Teel_Formation_pmag_files/Teel_Formation_pmag_148_0.png}
    \end{center}
    { \hspace*{\fill} \\}
    
    \subparagraph{Tilt-corrected coordinates -
magnetite}\label{tilt-corrected-coordinates---magnetite}


            \begin{Verbatim}[commandchars=\\\{\}]
{\color{outcolor}Out[{\color{outcolor}70}]:}          strat\_pos  site\_lat  site\_lon      dec\_tc     inc\_tc    alpha95   n  \textbackslash{}
         Z30\_mag          4  47.10038  95.37550  278.379367 -46.010280  11.715664   7   
         Z31\_mag          5  47.10049  95.37604  246.928001 -33.319165   3.753646   8   
         Z32\_mag          6  47.10094  95.37684  234.132395 -32.854953   7.997970   9   
         Z33\_mag          7  47.10107  95.37705  250.685156 -32.176114   4.581718   8   
         Z34\_mag          8  47.10111  95.37712  247.131951 -35.840117   2.844796  10   
         Z35\_mag          9  47.10069  95.37747  226.181747 -25.332749   6.399180   6   
         Z36\_mag         11  47.10221  95.37959  247.531592 -39.317142   4.170654   8   
         Z37\_mag         12  47.10211  95.37971  248.052322 -42.739898   7.478423   9   
         Z38\_mag         13  47.09855  95.38445  237.351592 -42.681665   8.027422  10   
         Z39\_mag         14  47.09860  95.38467  230.880551 -26.400672   6.363220   6   
         Z40\_mag         15  47.09859  95.38474  241.380972 -42.753133   4.822532   9   
         Z41\_mag         10  47.10109  95.37744  229.065998 -29.074319   4.590751   8   
         Z42\_mag         16  47.09577  95.38577  219.102263 -24.487183  11.152830   8   
         Z43\_mag         17  47.09570  95.38638  208.946090 -22.598975  10.757156   8   
         Z44\_mag         18  47.09571  95.38651  160.933464  20.616106  12.332689   5   
         Z45\_mag         19  47.09562  95.38676  169.181253  22.723096   8.151115   8   
         Z46\_mag         20  47.09563  95.38692  235.938073 -30.336640   6.723991   8   
         Z47\_mag         21  47.09568  95.38727  236.225390 -32.077321   5.605537   8   
         Z48\_mag         22  47.09570  95.38744  227.700040 -32.087392   6.771057   8   
         Z49\_mag         23  47.09581  95.38747  235.973353 -37.612989   9.479951   7   
         Z50\_mag         24  47.09575  95.38781  175.405740  15.829522   3.922362   8   
         Z51\_mag         25  47.09584  95.38802  229.796663 -36.494284  10.382678   5   
         Z52\_mag         26  47.09583  95.38815  216.149983 -45.789543  12.698064   6   
         Z53\_mag          1  47.09442  95.37205  246.701511 -31.999187   3.599051   8   
         Z54\_mag          2  47.09502  95.37299  243.678159 -32.414475   8.199873   8   
         Z55\_mag          3  47.09525  95.37351  238.040663 -33.518307   5.611932   8   
         Z56\_mag         27  47.06403  95.42075  181.601802 -27.881082   5.069905   6   
         Z57\_mag         28  47.06277  95.42039  181.062329 -31.634060  13.082813   6   
         Z58\_mag         29  47.06277  95.42045  119.444412 -37.602921  17.552705   6   
         
                           k         r        csd  paleolatitude    vgp\_lat  \textbackslash{}
         Z30\_mag   27.500012  6.781818  15.446091     -27.381947 -14.408220   
         Z31\_mag  218.734693  7.967998   5.476791     -18.194560 -28.826591   
         Z32\_mag   42.396758  8.811306  12.439952     -17.895640 -37.203135   
         Z33\_mag  147.127336  7.952422   6.677876     -17.462312 -25.759713   
         Z34\_mag  289.334789  9.968894   4.761948     -19.856873 -29.843343   
         Z35\_mag  110.584559  5.954786   7.702608     -13.316749 -38.856026   
         Z36\_mag  177.361534  7.960533   6.082124     -22.268860 -31.221631   
         Z37\_mag   48.354733  8.834556  11.648380     -24.798438 -32.562210   
         Z38\_mag   37.175977  9.757908  13.284763     -24.753981 -39.808772   
         Z39\_mag  111.827254  5.955288   7.659690     -13.939647 -36.393048   
         Z40\_mag  114.937717  8.930397   7.555335     -24.808549 -37.108819   
         Z41\_mag  146.552710  7.952236   6.690955     -15.536089 -38.748850   
         Z42\_mag   25.622407  7.726802  16.002029     -12.829071 -42.668310   
         Z43\_mag   27.470571  7.745182  15.454365     -11.756542 -47.094786   
         Z44\_mag   39.445745  4.898595  12.896888      10.652755 -29.797321   
         Z45\_mag   47.137245  7.851497  11.797851      11.826377 -30.288956   
         Z46\_mag   68.822329  7.898289   9.763833     -16.309766 -34.865831   
         Z47\_mag   98.607774  7.929012   8.156981     -17.399618 -35.463534   
         Z48\_mag   67.882078  7.896880   9.831221     -17.406005 -41.018935   
         Z49\_mag   41.499592  6.855420  12.573700     -21.068304 -38.228444   
         Z50\_mag  200.402350  7.965070   5.721812       8.068715 -34.684493   
         Z51\_mag   55.264038  4.927620  10.895915     -20.299603 -41.779935   
         Z52\_mag   28.791577  5.826338  15.095667     -27.201966 -55.462438   
         Z53\_mag  237.845848  7.970569   5.252151     -17.350100 -28.388510   
         Z54\_mag   46.589636  7.849752  11.866984     -17.613960 -30.621692   
         Z55\_mag   98.385329  7.928851   8.166197     -18.323455 -34.914855   
         Z56\_mag  175.612751  5.971528   6.112333     -14.816796 -57.725144   
         Z57\_mag   27.177964  5.816027  15.537336     -17.119456 -60.043844   
         Z58\_mag   15.519815  5.677831  20.560882     -21.061319 -35.140582   
         
                     vgp\_lon  vgp\_lat\_rev  vgp\_lon\_rev  
         Z30\_mag  340.470733    14.408220   160.470733  
         Z31\_mag    9.314505    28.826591   189.314505  
         Z32\_mag   19.865666    37.203135   199.865666  
         Z33\_mag    7.086542    25.759713   187.086542  
         Z34\_mag    7.791244    29.843343   187.791244  
         Z35\_mag   31.001433    38.856026   211.001433  
         Z36\_mag    5.455763    31.221631   185.455763  
         Z37\_mag    2.867308    32.562210   182.867308  
         Z38\_mag   10.899760    39.808772   190.899760  
         Z39\_mag   26.087975    36.393048   206.087975  
         Z40\_mag    7.746124    37.108819   187.746124  
         Z41\_mag   26.425574    38.748850   206.425574  
         Z42\_mag   38.628730    42.668310   218.628730  
         Z43\_mag   51.278792    47.094786   231.278792  
         Z44\_mag  117.098874    29.797321   297.098874  
         Z45\_mag  107.671016    30.288956   287.671016  
         Z46\_mag   19.680166    34.865831   199.680166  
         Z47\_mag   18.516120    35.463534   198.516120  
         Z48\_mag   26.092301    41.018935   206.092301  
         Z49\_mag   15.487683    38.228444   195.487683  
         Z50\_mag  100.922274    34.684493   280.922274  
         Z51\_mag   21.526652    41.779935   201.526652  
         Z52\_mag   27.657758    55.462438   207.657758  
         Z53\_mag   10.165158    28.388510   190.165158  
         Z54\_mag   12.284724    30.621692   192.284724  
         Z55\_mag   16.200615    34.914855   196.200615  
         Z56\_mag   92.519901    57.725144   272.519901  
         Z57\_mag   93.386864    60.043844   273.386864  
         Z58\_mag  179.028128    35.140582   359.028128  
\end{Verbatim}
        

    \begin{center}
    \adjustimage{max size={0.9\linewidth}{0.9\paperheight}}{Teel_Formation_pmag_files/Teel_Formation_pmag_151_0.png}
    \end{center}
    { \hspace*{\fill} \\}
    

    \begin{center}
    \adjustimage{max size={0.9\linewidth}{0.9\paperheight}}{Teel_Formation_pmag_files/Teel_Formation_pmag_152_0.png}
    \end{center}
    { \hspace*{\fill} \\}
    
    \paragraph{Hematite directions}\label{hematite-directions}

    \subparagraph{Geographic coordinates -
hematite}\label{geographic-coordinates---hematite}


            \begin{Verbatim}[commandchars=\\\{\}]
{\color{outcolor}Out[{\color{outcolor}73}]:}              strat\_pos  site\_lat  site\_lon     dec\_geo    inc\_geo    alpha95  \textbackslash{}
         Z35\_hem\_geo          4  47.10038  95.37550  185.352162  -5.560896   5.747798   
         Z30\_hem\_geo          9  47.10069  95.37747  148.799913 -68.209834  49.957167   
         Z38\_hem\_geo         13  47.09855  95.38445  195.565071 -26.364592   8.833229   
         Z39\_hem\_geo         14  47.09860  95.38467  196.313095 -15.623428   8.663295   
         Z40\_hem\_geo         15  47.09859  95.38474  189.173033 -23.507115   7.794855   
         Z41\_hem\_geo         10  47.10109  95.37744  178.834997 -26.225350  16.086290   
         Z42\_hem\_geo         16  47.09577  95.38577  165.448451  27.278167  16.388007   
         Z43\_hem\_geo         17  47.09570  95.38638  182.753379  12.734651   6.220142   
         Z44\_hem\_geo         18  47.09571  95.38651  182.138846  26.690156   2.023836   
         Z45\_hem\_geo         19  47.09562  95.38676  179.431606  23.111384   4.657126   
         Z46\_hem\_geo         20  47.09563  95.38692  199.630710  23.039855  43.076043   
         Z47\_hem\_geo         21  47.09568  95.38727  186.530284  -3.818259  14.407211   
         Z48\_hem\_geo         22  47.09570  95.38744  186.680865  20.745840   5.248580   
         Z49\_hem\_geo         23  47.09581  95.38747  201.506939   6.066986  11.535574   
         Z50\_hem\_geo         24  47.09575  95.38781  185.576900  22.848691   3.695319   
         Z51\_hem\_geo         25  47.09584  95.38802  184.630448  22.563173  10.926624   
         Z52\_hem\_geo         26  47.09583  95.38815  182.777548  21.100104  13.120981   
         Z53\_hem\_geo          1  47.09442  95.37205  176.883446 -21.454440   5.750351   
         Z54\_hem\_geo          2  47.09502  95.37299  166.032855 -40.835976  52.801938   
         Z56\_hem\_geo         27  47.06403  95.42075  179.194265   1.511589   8.593573   
         Z57\_hem\_geo         28  47.06277  95.42039  181.947932   4.332196   7.473594   
         
                      n           k         r        csd  paleolatitude    vgp\_lat  \textbackslash{}
         Z35\_hem\_geo  6  136.842402  5.963462   6.924281      -2.787011 -45.444034   
         Z30\_hem\_geo  7    2.413452  4.513935  52.139357     -51.356099 -69.351223   
         Z38\_hem\_geo  2  801.477440  1.998752   2.861142     -13.918471 -54.362780   
         Z39\_hem\_geo  5   78.962990  4.949343   9.115347      -7.959659 -48.458523   
         Z40\_hem\_geo  7   60.926778  6.901521  10.377217     -12.269627 -54.326542   
         Z41\_hem\_geo  5   23.576274  4.830338  16.681974     -13.836835 -56.721476   
         Z42\_hem\_geo  6   17.664613  5.716948  19.272274      14.457588 -27.077402   
         Z43\_hem\_geo  8   80.262873  7.912787   9.041233       6.446942 -36.401745   
         Z44\_hem\_geo  7  890.663468  6.993263   2.714115      14.109891 -28.764330   
         Z45\_hem\_geo  8  142.431974  7.950854   6.787053      12.045434 -30.856760   
         Z46\_hem\_geo  2   35.743651  1.972023  13.548324      12.005012 -28.348190   
         Z47\_hem\_geo  7   18.506577  6.675791  18.828771      -1.911251 -44.460110   
         Z48\_hem\_geo  8  112.342494  7.937691   7.642105      10.724316 -31.873027   
         Z49\_hem\_geo  6   34.686217  5.855851  13.753288       3.042020 -36.413576   
         Z50\_hem\_geo  8  225.663876  7.968980   5.392051      11.897134 -30.796567   
         Z51\_hem\_geo  5   49.990867  4.919985  11.456176      11.736415 -31.022182   
         Z52\_hem\_geo  6   27.025581  5.814990  15.581078      10.920183 -31.930956   
         Z53\_hem\_geo  7  111.157505  6.946023   7.682731     -11.116766 -53.926099   
         Z54\_hem\_geo  8    2.056913  4.596843  56.477701     -23.370939 -63.768703   
         Z56\_hem\_geo  6   61.741195  5.919017  10.308548       0.755926 -42.174837   
         Z57\_hem\_geo  4  152.109621  3.980277   6.567600       2.169198 -40.738280   
         
                         vgp\_lon  vgp\_lat\_rev  vgp\_lon\_rev  
         Z35\_hem\_geo   87.744611    45.444034   267.744611  
         Z30\_hem\_geo  208.833918    69.351223    28.833918  
         Z38\_hem\_geo   68.832083    54.362780   248.832083  
         Z39\_hem\_geo   70.583219    48.458523   250.583219  
         Z40\_hem\_geo   79.891757    54.326542   259.891757  
         Z41\_hem\_geo   97.439300    56.721476   277.439300  
         Z42\_hem\_geo  111.243174    27.077402   291.243174  
         Z43\_hem\_geo   91.986451    36.401745   271.986451  
         Z44\_hem\_geo   93.020083    28.764330   273.020083  
         Z45\_hem\_geo   96.034300    30.856760   276.034300  
         Z46\_hem\_geo   73.462329    28.348190   253.462329  
         Z47\_hem\_geo   86.223719    44.460110   266.223719  
         Z48\_hem\_geo   87.651814    31.873027   267.651814  
         Z49\_hem\_geo   68.327763    36.413576   248.327763  
         Z50\_hem\_geo   89.031882    30.796567   269.031882  
         Z51\_hem\_geo   90.095926    31.022182   270.095926  
         Z52\_hem\_geo   92.174224    31.930956   272.174224  
         Z53\_hem\_geo  100.570110    53.926099   280.570110  
         Z54\_hem\_geo  125.457583    63.768703   305.457583  
         Z56\_hem\_geo   96.507901    42.174837   276.507901  
         Z57\_hem\_geo   92.851012    40.738280   272.851012  
\end{Verbatim}
        

    \begin{center}
    \adjustimage{max size={0.9\linewidth}{0.9\paperheight}}{Teel_Formation_pmag_files/Teel_Formation_pmag_156_0.png}
    \end{center}
    { \hspace*{\fill} \\}
    

    \begin{center}
    \adjustimage{max size={0.9\linewidth}{0.9\paperheight}}{Teel_Formation_pmag_files/Teel_Formation_pmag_157_0.png}
    \end{center}
    { \hspace*{\fill} \\}
    
    \subparagraph{Tilt-corrected coordinates -
hematite}\label{tilt-corrected-coordinates---hematite}


            \begin{Verbatim}[commandchars=\\\{\}]
{\color{outcolor}Out[{\color{outcolor}76}]:}          strat\_pos  site\_lat  site\_lon      dec\_tc     inc\_tc    alpha95  n  \textbackslash{}
         Z35\_hem          4  47.10038  95.37550  185.114311   6.025809   5.727598  6   
         Z30\_hem          9  47.10069  95.37747  242.928054 -40.237690  49.931876  7   
         Z38\_hem         13  47.09855  95.38445  208.352571  -5.900502   8.908547  2   
         Z39\_hem         14  47.09860  95.38467  201.539571   2.413034   8.658960  5   
         Z40\_hem         15  47.09859  95.38474  202.157748  -7.928196   7.788149  7   
         Z41\_hem         10  47.10109  95.37744  198.411151 -11.058439  16.113432  5   
         Z42\_hem         16  47.09577  95.38577  151.386122  14.995751  16.374487  6   
         Z43\_hem         17  47.09570  95.38638  173.787706  13.587096   6.227989  8   
         Z44\_hem         18  47.09571  95.38651  163.946455  23.981346   2.033549  7   
         Z45\_hem         19  47.09562  95.38676  164.416633  19.689336   4.665334  8   
         Z46\_hem         20  47.09563  95.38692  180.134056  31.708243  43.051197  2   
         Z47\_hem         21  47.09568  95.38727  186.913270   2.679432  14.418197  7   
         Z48\_hem         22  47.09570  95.38744  175.819816  20.880528   5.256939  8   
         Z49\_hem         23  47.09581  95.38747  196.482372  14.863765  11.529913  6   
         Z50\_hem         24  47.09575  95.38781  173.713227  22.176830   3.699165  8   
         Z51\_hem         25  47.09584  95.38802  173.080330  21.478208  10.923883  5   
         Z52\_hem         26  47.09583  95.38815  172.236460  19.385124  13.100041  6   
         Z53\_hem          1  47.09442  95.37205  195.874742 -12.092261   5.754882  7   
         Z54\_hem          2  47.09502  95.37299  210.517882 -28.621488  52.793211  8   
         Z56\_hem         27  47.06403  95.42075  180.301491 -21.711722   8.592210  6   
         Z57\_hem         28  47.06277  95.42039  182.841002 -18.592084   7.462069  4   
         
                           k         r        csd  paleolatitude    vgp\_lat  \textbackslash{}
         Z35\_hem  137.802632  5.963716   6.900114       3.021259 -39.676603   
         Z30\_hem    2.414928  4.515454  52.123423     -22.933069 -34.802858   
         Z38\_hem  788.017587  1.998731   2.885473      -2.958094 -39.500385   
         Z39\_hem   79.041121  4.949393   9.110841       1.207052 -38.143076   
         Z40\_hem   61.030111  6.901688  10.368429      -3.983165 -42.830083   
         Z41\_hem   23.500114  4.829788  16.708984      -5.581195 -45.565273   
         Z42\_hem   17.692229  5.717390  19.257227       7.628503 -29.676516   
         Z43\_hem   80.063147  7.912569   9.052503       6.890414 -35.733259   
         Z44\_hem  882.184819  6.993199   2.727126      12.539568 -28.658505   
         Z45\_hem  141.934618  7.950682   6.798934      10.144077 -31.097230   
         Z46\_hem   35.782409  1.972053  13.540984      17.166216 -25.738041   
         Z47\_hem   18.479833  6.675322  18.842390       1.340449 -41.186055   
         Z48\_hem  111.988554  7.937494   7.654172      10.798704 -31.985346   
         Z49\_hem   34.719353  5.855988  13.746724       7.559053 -33.419750   
         Z50\_hem  225.196841  7.968916   5.397639      11.519700 -31.115698   
         Z51\_hem   50.015473  4.920025  11.453358      11.129977 -31.446845   
         Z52\_hem   27.109018  5.815560  15.557081       9.978044 -32.507612   
         Z53\_hem  110.984038  6.945938   7.688733      -6.114213 -46.812655   
         Z54\_hem    2.057276  4.597443  56.472722     -15.261779 -49.341300   
         Z56\_hem   61.760484  5.919042  10.306938     -11.259937 -54.195001   
         Z57\_hem  152.576865  3.980338   6.557536      -9.547317 -52.406934   
         
                     vgp\_lon  vgp\_lat\_rev  vgp\_lon\_rev  
         Z35\_hem   88.733786    39.676603   268.733786  
         Z30\_hem    8.295178    34.802858   188.295178  
         Z38\_hem   57.459211    39.500385   237.459211  
         Z39\_hem   67.563030    38.143076   247.563030  
         Z40\_hem   64.518459    42.830083   244.518459  
         Z41\_hem   68.698517    45.565273   248.698517  
         Z42\_hem  128.501026    29.676516   308.501026  
         Z43\_hem  102.991555    35.733259   282.991555  
         Z44\_hem  113.302682    28.658505   293.302682  
         Z45\_hem  113.374942    31.097230   293.374942  
         Z46\_hem   95.244729    25.738041   275.244729  
         Z47\_hem   86.186406    41.186055   266.186406  
         Z48\_hem  100.230031    31.985346   280.230031  
         Z49\_hem   75.695091    33.419750   255.695091  
         Z50\_hem  102.587676    31.115698   282.587676  
         Z51\_hem  103.352735    31.446845   283.352735  
         Z52\_hem  104.465028    32.507612   284.465028  
         Z53\_hem   71.955927    46.812655   251.955927  
         Z54\_hem   46.618190    49.341300   226.618190  
         Z56\_hem   94.915322    54.195001   274.915322  
         Z57\_hem   90.824846    52.406934   270.824846  
\end{Verbatim}
        

    \begin{center}
    \adjustimage{max size={0.9\linewidth}{0.9\paperheight}}{Teel_Formation_pmag_files/Teel_Formation_pmag_160_0.png}
    \end{center}
    { \hspace*{\fill} \\}
    

    \begin{center}
    \adjustimage{max size={0.9\linewidth}{0.9\paperheight}}{Teel_Formation_pmag_files/Teel_Formation_pmag_161_0.png}
    \end{center}
    { \hspace*{\fill} \\}
    
    \paragraph{Mid-temperature directions}\label{mid-temperature-directions}


            \begin{Verbatim}[commandchars=\\\{\}]
{\color{outcolor}Out[{\color{outcolor}79}]:}          strat\_pos  site\_lat  site\_lon     dec\_geo    inc\_geo    alpha95  n  \textbackslash{}
         Z42\_mid         16  47.09577  95.38577  332.936127 -68.973663  22.704625  4   
         Z43\_mid         17  47.09570  95.38638  214.749151 -62.316153  27.256978  4   
         Z58\_mid         29  47.06277  95.42045  169.686782 -61.093602   4.004771  6   
         
                           k         r        csd  paleolatitude    vgp\_lat  \textbackslash{}
         Z42\_mid   17.342624  3.827016  19.450359     -52.447527 -12.194273   
         Z43\_mid   12.329873  3.756688  23.067776     -43.621783 -65.542782   
         Z58\_mid  280.878503  5.982199   4.833100     -42.161138 -81.183719   
         
                     vgp\_lon  vgp\_lat\_rev  vgp\_lon\_rev  
         Z42\_mid  291.867415    12.194273   111.867415  
         Z43\_mid    0.679322    65.542782   180.679322  
         Z58\_mid  155.401297    81.183719   335.401297  
\end{Verbatim}
        

    \begin{center}
    \adjustimage{max size={0.9\linewidth}{0.9\paperheight}}{Teel_Formation_pmag_files/Teel_Formation_pmag_164_0.png}
    \end{center}
    { \hspace*{\fill} \\}
    
    Flow Z58 yielded a completely different mid-temperature result compared
to all other sites. The magnetite direction is completely different than
all other results. The mean direction is very imprecise (SE and
moderately-shallow down) but is closest in orientation to the Middle to
Late Carboniferous `A' component of Edel et al. (2014).

    \subsubsection{Paleomagnetic Poles for the Teel
Formation}\label{paleomagnetic-poles-for-the-teel-formation}

    We beleive that the primary paleomagnetic pole for the Teel basalts is
held by magnetite in some form (i.e., magnetite with slightly different
amounts of titanium). However, demagnetization data from some sites show
similarities to the remanence directions of magnetite and hematite. We
suspect that the hematite directions were acquired later in the
Paleozoic Era, therefore when the magnetite remanence is similar to that
of hematite we suspect that the magnetite has been chemically
overprinted by dominant amounts of hematite.

    \paragraph{Primary magnetite pole - including fold
test}\label{primary-magnetite-pole---including-fold-test}

    All of these mean directions are derived from unblocking temperatures in
the magnetite range. However, we beleive that a number of sites yield
samples where the hematite remanence demagnetizes at the same time as
magnetite, therefore preventing a measurement of the pure magnetite
magnetization. When this behavior is suspected or evident in samples or
sites, those results are documented and excluded.




            \begin{Verbatim}[commandchars=\\\{\}]
{\color{outcolor}Out[{\color{outcolor}83}]:} \{'alpha95': 4.9431252055096131,
          'csd': 13.058928417331206,
          'dec': 236.6128693429381,
          'inc': -34.995940978983391,
          'k': 38.472902790170508,
          'n': 23,
          'r': 22.428168960372265\}
\end{Verbatim}
        

            \begin{Verbatim}[commandchars=\\\{\}]
{\color{outcolor}Out[{\color{outcolor}84}]:} \{'alpha95': 5.7116821039230583,
          'csd': 15.025346173655382,
          'dec': 186.62753154703333,
          'inc': -64.852074989296611,
          'k': 29.061703291478157,
          'n': 23,
          'r': 22.24299000029874\}
\end{Verbatim}
        

    \begin{center}
    \adjustimage{max size={0.9\linewidth}{0.9\paperheight}}{Teel_Formation_pmag_files/Teel_Formation_pmag_174_0.png}
    \end{center}
    { \hspace*{\fill} \\}
    
    Bootstrap fold test (Tauxe and Watson, 1994)


    \begin{Verbatim}[commandchars=\\\{\}]
doing  1000  iterations{\ldots}please be patient{\ldots}
    \end{Verbatim}

    \begin{center}
    \adjustimage{max size={0.9\linewidth}{0.9\paperheight}}{Teel_Formation_pmag_files/Teel_Formation_pmag_176_1.png}
    \end{center}
    { \hspace*{\fill} \\}
    
    \begin{center}
    \adjustimage{max size={0.9\linewidth}{0.9\paperheight}}{Teel_Formation_pmag_files/Teel_Formation_pmag_176_2.png}
    \end{center}
    { \hspace*{\fill} \\}
    
    \begin{Verbatim}[commandchars=\\\{\}]
tightest grouping of vectors obtained at (95\% confidence bounds):
52 - 97 percent unfolding
range of all bootstrap samples: 
37  -  109 percent unfolding
    \end{Verbatim}

    \begin{center}
    \adjustimage{max size={0.9\linewidth}{0.9\paperheight}}{Teel_Formation_pmag_files/Teel_Formation_pmag_176_4.png}
    \end{center}
    { \hspace*{\fill} \\}
    
    Below the tilt-corrected magnetite VGPs are plotted on the globe and
shaded according to their relative stratigraphic positions.


    \begin{center}
    \adjustimage{max size={0.9\linewidth}{0.9\paperheight}}{Teel_Formation_pmag_files/Teel_Formation_pmag_178_0.png}
    \end{center}
    { \hspace*{\fill} \\}
    
    \paragraph{Secondary hematite pole - including fold
test}\label{secondary-hematite-pole---including-fold-test}




            \begin{Verbatim}[commandchars=\\\{\}]
{\color{outcolor}Out[{\color{outcolor}90}]:} \{'alpha95': 9.2608689622177227,
          'csd': 20.97901071644646,
          'dec': 184.58560616364625,
          'inc': 3.8459585385145614,
          'k': 14.907335584818336,
          'n': 18,
          'r': 16.85962183494998\}
\end{Verbatim}
        

            \begin{Verbatim}[commandchars=\\\{\}]
{\color{outcolor}Out[{\color{outcolor}91}]:} \{'alpha95': 9.4955663885263011,
          'csd': 21.474669531178204,
          'dec': 182.89089648457883,
          'inc': 6.1094491428847837,
          'k': 14.227122114351561,
          'n': 18,
          'r': 16.805099171613119\}
\end{Verbatim}
        

    \begin{center}
    \adjustimage{max size={0.9\linewidth}{0.9\paperheight}}{Teel_Formation_pmag_files/Teel_Formation_pmag_184_0.png}
    \end{center}
    { \hspace*{\fill} \\}
    
    Bootstrap fold test (Tauxe and Watson, 1994)


    \begin{Verbatim}[commandchars=\\\{\}]
doing  1000  iterations{\ldots}please be patient{\ldots}
    \end{Verbatim}

    \begin{center}
    \adjustimage{max size={0.9\linewidth}{0.9\paperheight}}{Teel_Formation_pmag_files/Teel_Formation_pmag_186_1.png}
    \end{center}
    { \hspace*{\fill} \\}
    
    \begin{center}
    \adjustimage{max size={0.9\linewidth}{0.9\paperheight}}{Teel_Formation_pmag_files/Teel_Formation_pmag_186_2.png}
    \end{center}
    { \hspace*{\fill} \\}
    
    \begin{Verbatim}[commandchars=\\\{\}]
tightest grouping of vectors obtained at (95\% confidence bounds):
-20 - 119 percent unfolding
range of all bootstrap samples: 
-20  -  119 percent unfolding
    \end{Verbatim}

    \begin{center}
    \adjustimage{max size={0.9\linewidth}{0.9\paperheight}}{Teel_Formation_pmag_files/Teel_Formation_pmag_186_4.png}
    \end{center}
    { \hspace*{\fill} \\}
    
    Below the geographic and tilt-corrected hematite VGPs are plotted on the
globe and shaded according to their relative stratigraphic positions.
Note the similar positions between the two coordinate system means.


    \begin{center}
    \adjustimage{max size={0.9\linewidth}{0.9\paperheight}}{Teel_Formation_pmag_files/Teel_Formation_pmag_188_0.png}
    \end{center}
    { \hspace*{\fill} \\}
    

    \begin{center}
    \adjustimage{max size={0.9\linewidth}{0.9\paperheight}}{Teel_Formation_pmag_files/Teel_Formation_pmag_189_0.png}
    \end{center}
    { \hspace*{\fill} \\}
    
    \paragraph{Present local field overprint - negative fold
test}\label{present-local-field-overprint---negative-fold-test}


            \begin{Verbatim}[commandchars=\\\{\}]
{\color{outcolor}Out[{\color{outcolor}96}]:}              strat\_pos  site\_lat  site\_lon     dec\_geo    inc\_geo    alpha95  \textbackslash{}
         Z30\_low\_geo          4  47.10038  95.37550  346.473053  68.261419  11.458361   
         Z31\_low\_geo          5  47.10049  95.37604    0.054372  64.923839   3.618073   
         Z32\_low\_geo          6  47.10094  95.37684   25.452043  61.325382   8.812910   
         Z33\_low\_geo          7  47.10107  95.37705    0.930272  64.380739   6.003504   
         Z34\_low\_geo          8  47.10111  95.37712    0.648074  62.405014   3.865736   
         
                       n           k         r        csd  paleolatitude    vgp\_lat  \textbackslash{}
         Z30\_low\_geo   7   28.705783  6.790983  15.118208      51.429142  80.188638   
         Z31\_low\_geo   8  235.361535  7.970259   5.279798      46.897847  89.793993   
         Z32\_low\_geo   8   40.461392  7.826996  12.733993      42.434489  71.428192   
         Z33\_low\_geo   8   86.089978  7.918690   8.729889      46.196993  88.893147   
         Z34\_low\_geo  10  157.128808  9.942722   6.461854      43.729780  86.598151   
         
                         vgp\_lon  vgp\_lat\_rev  vgp\_lon\_rev  
         Z30\_low\_geo   36.526238   -80.188638   216.526238  
         Z31\_low\_geo  264.986185   -89.793993    84.986185  
         Z32\_low\_geo  190.579622   -71.428192    10.579622  
         Z33\_low\_geo  239.802233   -88.893147    59.802233  
         Z34\_low\_geo  267.460060   -86.598151    87.460060  
\end{Verbatim}
        

            \begin{Verbatim}[commandchars=\\\{\}]
{\color{outcolor}Out[{\color{outcolor}97}]:}          strat\_pos  site\_lat  site\_lon     dec\_tc     inc\_tc    alpha95   n  \textbackslash{}
         Z30\_low          4  47.10038  95.37550  62.140855  33.713657  11.463681   7   
         Z31\_low          5  47.10049  95.37604  59.403835  27.876060   3.615270   8   
         Z32\_low          6  47.10094  95.37684  25.452043  61.325382   8.812910   8   
         Z33\_low          7  47.10107  95.37705  54.800714  28.344304   6.009620   8   
         Z34\_low          8  47.10111  95.37712  52.688703  27.653908   3.866823  10   
         
                           k         r        csd  paleolatitude    vgp\_lat  \textbackslash{}
         Z30\_low   28.680029  6.790795  15.124995      18.450285  32.248029   
         Z31\_low  235.725142  7.970304   5.275724      14.813793  31.483537   
         Z32\_low   40.461392  7.826996  12.733993      42.434489  71.428192   
         Z33\_low   85.916769  7.918526   8.738684      15.094685  34.722681   
         Z34\_low  157.041051  9.942690   6.463659      14.681124  35.788473   
         
                     vgp\_lon  vgp\_lat\_rev  vgp\_lon\_rev  
         Z30\_low  192.800152   -32.248029    12.800152  
         Z31\_low  198.002746   -31.483537    18.002746  
         Z32\_low  190.579622   -71.428192    10.579622  
         Z33\_low  201.658195   -34.722681    21.658195  
         Z34\_low  203.849471   -35.788473    23.849471  
\end{Verbatim}
        
    A number of poles are excluded because of inconsistencies between
samples within site which resulted in large a95 values for these sites:
Z45, Z51, and Z52.



    \begin{center}
    \adjustimage{max size={0.9\linewidth}{0.9\paperheight}}{Teel_Formation_pmag_files/Teel_Formation_pmag_195_0.png}
    \end{center}
    { \hspace*{\fill} \\}
    
    \subparagraph{Bootstrap fold test (Tauxe and Watson,
1994)}\label{bootstrap-fold-test-tauxe-and-watson-1994}


    \begin{Verbatim}[commandchars=\\\{\}]
doing  1000  iterations{\ldots}please be patient{\ldots}
    \end{Verbatim}

    \begin{center}
    \adjustimage{max size={0.9\linewidth}{0.9\paperheight}}{Teel_Formation_pmag_files/Teel_Formation_pmag_197_1.png}
    \end{center}
    { \hspace*{\fill} \\}
    
    \begin{center}
    \adjustimage{max size={0.9\linewidth}{0.9\paperheight}}{Teel_Formation_pmag_files/Teel_Formation_pmag_197_2.png}
    \end{center}
    { \hspace*{\fill} \\}
    
    \begin{Verbatim}[commandchars=\\\{\}]
tightest grouping of vectors obtained at (95\% confidence bounds):
-18 - 10 percent unfolding
range of all bootstrap samples: 
-20  -  28 percent unfolding
    \end{Verbatim}

    \begin{center}
    \adjustimage{max size={0.9\linewidth}{0.9\paperheight}}{Teel_Formation_pmag_files/Teel_Formation_pmag_197_4.png}
    \end{center}
    { \hspace*{\fill} \\}
    
    \paragraph{Teel poles summary}\label{teel-poles-summary}


            \begin{Verbatim}[commandchars=\\\{\}]
{\color{outcolor}Out[{\color{outcolor}126}]:}                     Pole\_Lat  Pole\_Long      A\_95          K        CSD   N  \textbackslash{}
          Teel\_magnetite\_tc -36.495314  16.038788  5.236274  34.392364  13.811918  23   
          Teel\_hematite\_tc  -39.717588  91.918678  7.536314  22.013320  17.264033  18   
          Teel\_hematite\_geo -40.795648  89.426839  5.608638  38.960325  12.976983  18   
          
                                     r   Paleolat  
          Teel\_magnetite\_tc  22.360323 -19.292649  
          Teel\_hematite\_tc   17.227740   3.063432  
          Teel\_hematite\_geo  17.563659   1.925148  
\end{Verbatim}
        
    \subsection{Pole compilation for Siberia, North China, and Mongolian
terranes}\label{pole-compilation-for-siberia-north-china-and-mongolian-terranes}

    \subsubsection{Import existing paleomagnetic
data}\label{import-existing-paleomagnetic-data}

    \paragraph{Siberia}\label{siberia}

    Note that a rotation needs to be applied for relative rotation between
Aldan and Anabar blocks before the Devonian Period due to Devonian
rifting in the Viljuy Basin near the centre of the craton. Here are two
rotations used in the literature: Euler Pole (Lat, Long, rotation)

\begin{itemize}
\tightlist
\item
  (60ºN, 120ºE, 13º) from Smethurst et al. (1998)
\item
  (60ºN, 115ºE, 25º) everything pre-Devonian (Evans, 2009)
\item
  (60ºN, 120ºE, 16º) Cambrian to Early Silurian correction (Cocks and
  Torsvik, 2007)
\item
  (62ºN, 117ºE, 20º) pre-Devonian (Pavlov et al. (2008) also used in
  Powerman et al. (2013))
\end{itemize}

Most Siberia poles are imported from Cocks and Torsvik (2007) which
rotates data from the ``southern'' Siberia (Aldan) into the northern
Siberia (Anabar) reference frame according to Smethurst et al. (1998),
which the authors claim brings N and S pre-Devonian poles into the best
agreement.

    Torsvik et al. (2012) updated their Siberia apparent polar wander path
by adding data from Shatsillo et al. (2007) that superceded results from
the coeval Lena River sediments (Rodianov et al., 1982; Torsvik et al.,
1995). However, there are more results from Siberia that must have been
discarded by Cocks and Torsvik (2007) and subsequently by other authors.
We discuss these poles later on.



    Below we plot the paleomagnetic data compiled by Cocks and Torsvik
(2007) for Siberia, shaded according to age.


    \begin{center}
    \adjustimage{max size={0.9\linewidth}{0.9\paperheight}}{Teel_Formation_pmag_files/Teel_Formation_pmag_208_0.png}
    \end{center}
    { \hspace*{\fill} \\}
    
    The data from Cocks and Torsvik (2007) are slightly updated by Torsvik
et al. (2012) with appropriate euler corrections and evaluation of
Ordovician-Silurian poles.


    \begin{center}
    \adjustimage{max size={0.9\linewidth}{0.9\paperheight}}{Teel_Formation_pmag_files/Teel_Formation_pmag_210_0.png}
    \end{center}
    { \hspace*{\fill} \\}
    
    In the following analyses, we update this pole list to include
additional poles from the area in order to construct a paleolatitdue
plot of Siberia through the Phanerozoic Era.

    We use the Haversine formula to calculate the distance between the VGPs
and a reference point on a given plate. This is then used to calculate
the paleolatite of the reference point.


    We first load poles for stable Europe from 250 Ma to the present day.


            \begin{Verbatim}[commandchars=\\\{\}]
{\color{outcolor}Out[{\color{outcolor}107}]:}    high\_age  low\_age  median\_age   A95  PLat   PLon  Paleolat  PLat\_N  PLon\_N
          0         1        0         0.5   3.6 -80.6  267.5    60.638    80.6    87.5
          1         1        0         0.5   4.4 -86.4  296.1    55.206    86.4   116.1
          2        10        6         8.0  12.9 -84.3  357.7    52.907    84.3   177.7
          3        11        8         9.5   1.8 -78.9  328.3    58.733    78.9   148.3
          4        11        9        10.0   3.5 -77.4  314.2    61.900    77.4   134.2
\end{Verbatim}
        
    Poles from Siberia are then loaded (545 to 250 Ma). Most of the poles
were taken from Cocks and Torsvik (2007) and Torsvik et al. (2012), but
we also added additional data gathered from the Global Paleomagnetic
Database.


            \begin{Verbatim}[commandchars=\\\{\}]
{\color{outcolor}Out[{\color{outcolor}146}]:}     plate\_ID  high\_age  low\_age  median\_age   A95  PLat   PLon  \textbackslash{}
          0        401       245      243       244.0  10.0 -59.0  330.0   
          1        401       258      238       248.0   7.8 -59.3  325.8   
          2        401       253      248       251.0   3.3 -56.2  326.0   
          3        401       253      248       251.0   9.7 -52.8  334.4   
          4        401       253      248       251.0   2.2 -56.6  307.9   
          5        401       285      265       275.0   8.6 -50.5  301.4   
          6        401       363      290       326.5   1.3 -21.0  350.0   
          7        401       352      332       342.0  17.0 -16.0  295.0   
          8        401       348      340       344.0   5.8 -25.2  320.0   
          9        401       377      350       360.0   8.9 -11.1  329.7   
          10       401       377      350       363.5  10.1 -27.8  339.9   
          11       401       377      350       363.5  11.9 -22.8  339.4   
          12       401       391      363       377.0   5.0 -13.0  302.0   
          13       410       430      397       413.5   3.2   8.2  292.0   
          14       401       443      423       433.0   4.6  19.0  308.0   
          15       410       444      423       433.5   4.4  18.4  302.7   
          16       401       454      424       439.0   8.0  14.0  304.0   
          17       401       460      440       450.0  17.3  19.4  315.3   
          18       410       461      443       452.0   5.1  27.5  332.0   
          19       401       464      458       461.0   2.5  22.8  334.2   
          20       401       464      458       461.0   5.1  22.1  324.9   
          21       401       473      453       463.0   4.0  23.0  338.0   
          22       401       470      464       467.0   3.2  30.9  332.7   
          23       401       478      458       468.0   3.1  24.4  346.0   
          24       401       479      459       469.0   4.0  30.0  337.0   
          25       401       480      460       470.0   9.0  17.9  342.8   
          26       401       488      468       478.0   2.2  33.9  331.7   
          27       401       495      470       482.5   5.8  36.2  338.8   
          28       401       493      473       483.0   9.0  40.0  318.0   
          29       401       495      485       490.0   4.9  35.2  307.2   
          30       401       495      485       490.0   2.3  41.9  315.8   
          31       401       510      490       500.0   6.0  37.0  318.0   
          32       401       505      495       500.0   3.0  36.1  310.7   
          33       401       518      495       506.5   4.5  32.6  333.8   
          34       401       514      500       507.0   2.6  43.7  320.5   
          35       401       518      505       511.5   4.6  36.4  319.6   
          36       401       520      510       515.0   5.1  53.3  315.0   
          37       401       535      518       526.5   6.8  44.8  338.7   
          38       401       538      518       528.0   7.0  32.0  317.0   
          39       401       545      525       535.0   6.2  16.6  244.5   
          40       401       545      535       540.0  12.8  37.6  345.0   
          
                                                      Reference  Paleolat  PLat\_N  \textbackslash{}
          0   GPDB2832, Gurevitch et al. (1995) from Cocks a{\ldots}    63.178    59.0   
          1   Walderhaug et al. (2005) from Cocks and Torsvi{\ldots}    65.344    59.3   
          2   Gurevitch et al. (2004) from Cocks and Torsvik{\ldots}    65.004    56.2   
          3   GPDB3486, Kravchinsky et al. (2002) from Cocks{\ldots}    59.477    52.8   
          4   Pavlov and Gallet (1996) from Cocks and Torsvi{\ldots}    74.988    56.6   
          5   Pisarevsky et al. (2006) from Cocks and Torsvi{\ldots}    78.724    50.5   
          6            GPDB1991, Davydov and Kravchinsky (1973)    30.794    21.0   
          7                          GPDB1986, Kamysheva (1971)    53.142    16.0   
          8                     GPDB3041, Zhitkov et al. (1994)    51.715    25.2   
          9   GPDB3486, Kravchinsky et al. (2002) from Cocks{\ldots}    34.892    11.1   
          10                GPDB3486, Kravchinsky et al. (2002)    42.021    27.8   
          11                GPDB3486, Kravchinsky et al. (2002)    38.641    22.8   
          12                         GPDB1997, Kamysheva (1975)    48.523    13.0   
          13                             Powerman et al. (2013)    29.655    -8.2   
          14  Shatsillo et al. (2007) from Cocks and Torsvik{\ldots}    16.126   -19.0   
          15                             Powerman et al. (2013)    17.919   -18.4   
          16  Smethurst et al. (1998) from Cocks and Torsvik{\ldots}    21.927   -14.0   
          17  Smethurst et al. (1998) from Cocks and Torsvik{\ldots}    13.660   -19.4   
          18                             Powerman et al. (2013)     0.109   -27.5   
          19  GPDB3473, Iosifidi et al. (1999) from Cocks an{\ldots}     3.312   -22.8   
          20  GPDB3473, Iosifidi et al. (1999) from Cocks an{\ldots}     7.787   -22.1   
          21  Smethurst et al. (1998) from Cocks and Torsvik{\ldots}     1.413   -23.0   
          22  GPDB3448 Gallet and Pavlov (1998) from Cocks a{\ldots}    -3.183   -30.9   
          23  Smethurst et al. (1998) from Cocks and Torsvik{\ldots}    -3.645   -24.4   
          24  Smethurst et al. (1998) from Cocks and Torsvik{\ldots}    -4.193   -30.0   
          25  Smethurst et al. (1998) from Cocks and Torsvik{\ldots}     3.434   -17.9   
          26  Smethurst et al. (1998) from Cocks and Torsvik{\ldots}    -5.441   -33.9   
          27  GPDB3474, Surkis et al. (1999) from Cocks and {\ldots}   -10.298   -36.2   
          28  Smethurst et al. (1998) from Cocks and Torsvik{\ldots}    -6.498   -40.0   
          29  GPDB3448, Gallet and Pavlov (1998) from Cocks {\ldots}     0.651   -35.2   
          30  GPDB3192, Pavlov and Gallet (1998) from Cocks {\ldots}    -7.711   -41.9   
          31  Smethurst et al. (1998) from Cocks and Torsvik{\ldots}    -3.690   -37.0   
          32  GPDB3192, Pavlov and Gallet (1998) from Cocks {\ldots}    -0.974   -36.1   
          33  GPDB3472, Rodionov et al. (1998) from Cocks an{\ldots}    -5.123   -32.6   
          34  GPDB3537, Gallet et al. (2003) from Cocks and {\ldots}   -10.622   -43.7   
          35  GPDB3164, Pisarevsky et al. (1997) from Cocks {\ldots}    -3.591   -36.4   
          36  GPDB3537, Gallet et al. (2003) from Cocks and {\ldots}   -18.264   -53.3   
          37  GPDB3164, Pisarevsky et al. (1997) from Cocks {\ldots}   -17.577   -44.8   
          38  Smethurst et al. (1998) from Cocks and Torsvik{\ldots}     1.285   -32.0   
          39  GPDB1627, Kirschvink and Rozanov (1984) from C{\ldots}    13.732   -16.6   
          40  GPDB3164, Pisarevsky et al. (1997) from Cocks {\ldots}   -14.154   -37.6   
          
              PLon\_N  
          0    150.0  
          1    145.8  
          2    146.0  
          3    154.4  
          4    127.9  
          5    121.4  
          6    170.0  
          7    115.0  
          8    140.0  
          9    149.7  
          10   159.9  
          11   159.4  
          12   122.0  
          13   112.0  
          14   128.0  
          15   122.7  
          16   124.0  
          17   135.3  
          18   152.0  
          19   154.2  
          20   144.9  
          21   158.0  
          22   152.7  
          23   166.0  
          24   157.0  
          25   162.8  
          26   151.7  
          27   158.8  
          28   138.0  
          29   127.2  
          30   135.8  
          31   138.0  
          32   130.7  
          33   153.8  
          34   140.5  
          35   139.6  
          36   135.0  
          37   158.7  
          38   137.0  
          39    64.5  
          40   165.0  
\end{Verbatim}
        

    \begin{center}
    \adjustimage{max size={0.9\linewidth}{0.9\paperheight}}{Teel_Formation_pmag_files/Teel_Formation_pmag_218_0.png}
    \end{center}
    { \hspace*{\fill} \\}
    
    \paragraph{North China}\label{north-china}

    We first load the paleomagnetic data for North China compile in Cocks
and Torsvik (2013).



    \begin{center}
    \adjustimage{max size={0.9\linewidth}{0.9\paperheight}}{Teel_Formation_pmag_files/Teel_Formation_pmag_222_0.png}
    \end{center}
    { \hspace*{\fill} \\}
    
    We also add some additional poles from North China that were not
included in Cocks and Torsvik (2013) including a compilation of data
from Huang et al. (1999) and additional poles from Embleton et al.
(1996), Huang et al. (2001), and Doh and Piper (1994).


            \begin{Verbatim}[commandchars=\\\{\}]
{\color{outcolor}Out[{\color{outcolor}112}]:}    high\_age  low\_age  median\_age  A95  PLat   PLon  \textbackslash{}
          0        88       68          78  5.8  79.7  170.8   
          1        98       78          88  5.3  81.1  294.5   
          2       110       90         100  4.7  70.6  156.7   
          3       123      103         113  5.2  76.8  192.1   
          4       126      106         116  4.6  83.6  172.3   
          
                                References  Paleolat  
          0  see Van der Voo et al. (2015)    46.149  
          1  see Van der Voo et al. (2015)    33.136  
          2  see Van der Voo et al. (2015)    52.870  
          3  see Van der Voo et al. (2015)    42.209  
          4  see Van der Voo et al. (2015)    44.602  
\end{Verbatim}
        

    \begin{Verbatim}[commandchars=\\\{\}]
/Users/taylorkilian/Library/Enthought/Canopy\_64bit/User/lib/python2.7/site-packages/matplotlib/axes/\_axes.py:519: UserWarning: No labelled objects found. Use label='{\ldots}' kwarg on individual plots.
  warnings.warn("No labelled objects found. "
    \end{Verbatim}

    \begin{center}
    \adjustimage{max size={0.9\linewidth}{0.9\paperheight}}{Teel_Formation_pmag_files/Teel_Formation_pmag_225_1.png}
    \end{center}
    { \hspace*{\fill} \\}
    
    \paragraph{Mongolia pole compilation}\label{mongolia-pole-compilation}

    Edel et al. (2014) published paleomagnetic data from 12 sites in the
Trans-Altai and South Gobi zones. This work identified magnetic
overprint directions for which a variety of arguments are made as to
their temporal relationship. The progression of directions as
interpretted by the authors leads to an appreciable change in magnetic
declination from overprints intrepretted to be Middle--Late
Carboniferous in age to magnetizations that are interpreted to be
Permian in age. The authors propose that this declination change is the
result of vertical axis rotation associated with oroclinal bending of a
Mongolian ribbon continent (cartoon model shown below).


    \begin{center}
    \adjustimage{max size={0.9\linewidth}{0.9\paperheight}}{Teel_Formation_pmag_files/Teel_Formation_pmag_228_0.png}
    \end{center}
    { \hspace*{\fill} \\}
    
    We import paleomagnetic data that have been compiled for Mongolia and
some of the surrounding terranes between Siberia and North China and
assigned them to established terranes pertinent to this study.


            \begin{Verbatim}[commandchars=\\\{\}]
{\color{outcolor}Out[{\color{outcolor}127}]:}               terrane  high\_age  low\_age  median\_age   A95  PLat   PLon  \textbackslash{}
          0     Zavkhan\_Baidrag       105       92        98.5   3.9  81.1  165.7   
          1     Zavkhan\_Baidrag       146       65       105.5  21.4  86.9  252.8   
          2     Zavkhan\_Baidrag       124       92       108.0   2.5  80.8  158.4   
          3     Zavkhan\_Baidrag       119      115       117.0   4.9  75.6  132.3   
          4      greater\_Amuria       130      110       120.0   5.2  70.8  322.4   
          5      greater\_Amuria       145       97       121.0   4.2  58.3   51.0   
          6     Zavkhan\_Baidrag       125      118       121.5   4.3  82.0  172.3   
          7      greater\_Amuria       133      125       129.0   7.4  86.8   61.8   
          8      greater\_Amuria       161      145       153.0   3.1  58.9  327.3   
          9      greater\_Amuria       176      145       160.5   4.2  59.6  279.0   
          10     greater\_Amuria       245      208       226.5  16.8  32.0   32.7   
          11  southern\_terranes       260      228       244.0   8.0  50.0  201.0   
          12    Zavkhan\_Baidrag       260      240       250.0  11.0  55.0  131.3   
          13     greater\_Amuria       271      260       265.5  14.4  63.1  151.0   
          14     greater\_Amuria       290      256       273.0  11.6  44.8  335.1   
          15  southern\_terranes       310      245       277.5   8.0  46.0  273.0   
          16  southern\_terranes       300      280       290.0   7.8  71.0  188.0   
          17    Zavkhan\_Baidrag       323      290       306.5  10.4  37.5  320.1   
          18  southern\_terranes       363      323       343.0  13.0  -1.0  354.1   
          19  southern\_terranes       391      363       377.0   3.4  39.9  244.3   
          20  southern\_terranes       391      363       377.0   4.6  51.7  282.7   
          21  southern\_terranes       391      363       377.0   3.5  38.0  244.0   
          22  southern\_terranes       391      363       377.0   5.1  52.0  280.0   
          23  southern\_terranes       363      245       304.0  11.9  50.0  354.0   
          24  southern\_terranes       340      299       319.5  13.0   5.0  341.0   
          25    Zavkhan\_Baidrag       440      200       320.0   5.6  40.8  269.4   
          26  southern\_terranes       360      320       340.0   4.9  10.0  330.0   
          27          Lake\_Zone       423      397       410.0   5.8 -13.3   63.7   
          28          Lake\_Zone       428      397       412.5   6.1  26.3  144.0   
          29          Lake\_Zone       428      416       422.0   3.6 -17.5  100.1   
          30    Zavkhan\_Baidrag       450      410       430.0  12.3   7.0  106.7   
          31    Zavkhan\_Baidrag       449      443       446.0   5.2  36.5  196.0   
          32    Zavkhan\_Baidrag       542      360       451.0   4.6   3.5  114.9   
          33    Zavkhan\_Baidrag       545      518       531.5  13.5  21.4  347.1   
          34    Zavkhan\_Baidrag       545      518       531.5  10.1  14.7  228.6   
          35    Zavkhan\_Baidrag       545      518       531.5   4.4  24.1  283.3   
          36    Zavkhan\_Baidrag       650      518       584.0   4.7  17.6  309.7   
          37    Zavkhan\_Baidrag       650      518       584.0   5.4  22.6  285.6   
          
                                           Reference  Paleolat  
          0             van Hinsbergen et al. (2008)    49.393  
          1                  GPDB2443, Pruner (1992)    44.225  
          2             van Hinsbergen et al. (2008)    50.579  
          3             van Hinsbergen et al. (2008)    57.658  
          4                      Cogne et al. (2005)    32.628  
          5                      Halim et al. (1998)    61.511  
          6             van Hinsbergen et al. (2008)    48.319  
          7                      Cogne et al. (2005)    49.735  
          8                      Cogne et al. (2005)    24.229  
          9                Kravchinsky et al. (2002)    16.741  
          10                 GPDB2443, Pruner (1992)    40.779  
          11                      Edel et al. (2014)    26.317  
          12                        Kovalenko (2010)    66.385  
          13               Kravchinsky et al. (2002)    55.811  
          14                 GPDB2443, Pruner (1992)    15.820  
          15                      Edel et al. (2014)     3.123  
          16                        Kovalenko (2010)    43.039  
          17                 GPDB2443, Pruner (1992)     3.567  
          18  GPDB3045, Pechersky and Didenko (1995)    -8.390  
          19  GPDB3045, Pechersky and Didenko (1995)     1.297  
          20  GPDB3045, Pechersky and Didenko (1995)     8.999  
          21         GPDB2594, Grishin et al. (1991)    -0.399  
          22         GPDB2594, Grishin et al. (1991)     9.179  
          23         GPDB2594, Grishin et al. (1991)    28.348  
          24                      Edel et al. (2014)   -12.479  
          25                              This study    -1.939  
          26                        Kovalenko (2010)   -15.126  
          27                 Bachtadse et al. (2000)    23.280  
          28                 Bachtadse et al. (2000)    46.717  
          29                 Bachtadse et al. (2000)    25.260  
          30                      Kravchinsky (2010)    48.745  
          31                              This study    19.566  
          32                     Evans et al. (1996)    43.245  
          33                      Kravchinsky (2001)     3.927  
          34  GPDB3045, Pechersky and Didenko (1995)   -15.368  
          35  GPDB3045, Pechersky and Didenko (1995)   -18.442  
          36  GPDB3045, Pechersky and Didenko (1995)   -18.324  
          37  GPDB3045, Pechersky and Didenko (1995)   -19.692  
\end{Verbatim}
        

    Calculate paleolatitudes for the Mongolia poles, considering that many
may have experienced horizonatal-axis rotations during the formation of
the COAB and possibly earlier.



    \begin{center}
    \adjustimage{max size={0.9\linewidth}{0.9\paperheight}}{Teel_Formation_pmag_files/Teel_Formation_pmag_234_0.png}
    \end{center}
    { \hspace*{\fill} \\}
    
    \subsection{Paleolatitude diagram}\label{paleolatitude-diagram}

    We plot the data plot all of the data from Siberia, North China, and
Monogolia (from Zavkhan, Baidrag, Lake Zone, and other southern
terranes) on a paleolatitude v. time plot. The paleolatitudes given for
each block are for specific reference points on each terrane. For
Mongolia, the site of the Teel Formation is used (95.38 ºN, 47.1 ºE).
For Siberia, coordinates at the southern tip of the craton (51.7 ºN,
103.5 ºE) are given seeing as this would be the proposed conjugate
margin for Mongolia. For north China, a reference point on the northern
margin (42 ºN, 109 ºE) is given to represent the an alternative
conjugate margin that would have shared very similar paleolatitudes with
Mongolia if they were attached.

    There are a handful of Mongolian poles that we exclude because of wide
age uncertainties or lack of statistical robustness. The Bachtadse et
al. (2000) component B pole is dismissed because of the small number or
samples used to calculate the mean direction (25 samples; unblocking
temperatures of 270--420 ºC); it is very similar to two Levashova (2010)
directions which may be overprints (see below). The Evans et al. (1996)
Bayan-Gol pole was superceded by results from Kravchinsky (2001) and may
likely be a pre-folding overprint, given the increase in precision after
tilt correction. The Kravchinsky et al. (2010) pole, that they call a
remagnetization, has a very uncertain age and is not tilt-corrected,
therefore we see it as unreliable for a paleolatitude estimate. The
Kovalenko (2010) pole from the granite at Hanbogd is excluded because it
is only from one site and is in the Trans-Altai zone, which is severely
affected by early Triassic deformation along the Gobi-Tienshan fault
(Lehmann et al., 2010). The ``Mongolian sediments and volcanics,
Gurvan-Sayhan Range, post-folding'' pole from Grishin et al. (1991)
(also discussed in Pechersky and Didenko (1995)) is dismissed because it
is likely underaveraged (only from one site) and because of its large
age uncertainty; it is also a post-folding remanence.


    
    \begin{verbatim}
<matplotlib.figure.Figure at 0x116988d90>
    \end{verbatim}

    
    \begin{center}
    \adjustimage{max size={0.9\linewidth}{0.9\paperheight}}{Teel_Formation_pmag_files/Teel_Formation_pmag_238_1.png}
    \end{center}
    { \hspace*{\fill} \\}
    
    \subsection{Regional overprints in Precambrian
rocks}\label{regional-overprints-in-precambrian-rocks}

    In order to understand the regional paleomagnetic directions,
specifically possible overprints, we compare results from the Zavkhan
block to see if there are dominant overprints that affected all rocks.
These results are from Levashova et al. (2010), Kravchinsky et al.
(2001), Evans et al. (1996), this study, and preliminary data from the
Zavkhan volcanics.


            \begin{Verbatim}[commandchars=\\\{\}]
{\color{outcolor}Out[{\color{outcolor}135}]:}               ID    N  k\_geo  Dec\_geo  Inc\_geo  a95\_geo   k\_tc  Dec\_tc  \textbackslash{}
          0      Lev10-INT   18   19.0    272.8    -66.2      8.2    3.0   158.7   
          1       Lev10-HT   11    3.0    311.7    -11.0     30.1   19.0   321.9   
          2      Lev11-INT   27   14.0    207.9    -30.6      7.8    6.0   210.4   
          3       Lev11-HT   18   10.0    194.2     29.2     11.5   41.0   179.6   
          4     Krav01-LOW   10   59.7      4.1     70.9      6.3   96.8   181.2   
          5     Krav01-INT    9   31.3    209.2    -66.0      9.3  117.6   284.3   
          6      Krav01-HT    6   14.9    118.5      5.3     17.9   13.3   118.3   
          7       Teel\_mag   23   29.1    186.6    -64.9      5.7   38.5   236.6   
          8       Teel\_hem   18   14.9    184.6      3.8      9.3   14.2   182.9   
          9       Evans\_HT  193    NaN      NaN      NaN      NaN    5.8   331.9   
          10   Z09\_cgl\_INT   20   32.3    200.7    -62.2      5.8   32.3   212.3   
          11  Z104\_cgl\_INT   31  165.4    174.8    -61.7      2.0  165.2    61.9   
          
              Inc\_tc  a95\_tc                    comments  
          0    -42.8    24.7                         NaN  
          1    -65.0    10.7                         NaN  
          2     -4.0    12.4                         NaN  
          3     53.7     5.4                         NaN  
          4     85.6     4.9  data from both B-G and T-O  
          5    -79.7     4.8  data from both B-G and T-O  
          6     -6.3    19.0               data from B-G  
          7    -35.0     4.9                         NaN  
          8      6.1     9.5                         NaN  
          9    -62.6     4.6                         NaN  
          10    -5.8     5.8                         NaN  
          11   -71.4     2.0                         NaN  
\end{Verbatim}
        
    All directions are first plotted in geographic coordinates.


    \begin{center}
    \adjustimage{max size={0.9\linewidth}{0.9\paperheight}}{Teel_Formation_pmag_files/Teel_Formation_pmag_243_0.png}
    \end{center}
    { \hspace*{\fill} \\}
    
    Then the overprint data are plotted in tilt-corrected coordinates.


    \begin{center}
    \adjustimage{max size={0.9\linewidth}{0.9\paperheight}}{Teel_Formation_pmag_files/Teel_Formation_pmag_245_0.png}
    \end{center}
    { \hspace*{\fill} \\}
    
    Given geological and statistical paleomagnetic context (improvement
during tilt correction) we creat a plot of what we believe to be the
actual overprint direction if they were acquired either before or after
folding.


    \begin{center}
    \adjustimage{max size={0.9\linewidth}{0.9\paperheight}}{Teel_Formation_pmag_files/Teel_Formation_pmag_247_0.png}
    \end{center}
    { \hspace*{\fill} \\}
    
    As stated in the main text, the Kravchinsky et al. (2001) intermediate
component in geographic coordinates is similar to the Teel magnetite
component in geographic coordinates. This similarity may be meaningless
because the intermediate component of Kravshinsky et al. (2001) improves
in precision after tilt-correction. This argues for the primary nature
of the tilt-corrected Teel magnetite component, or rather that this
componenet was acquired before folding/tilting of bedding.

    \subsection{Additional References}\label{additional-references}

    Bachtadse, V., Pavlov, V. E., Kazansky, A. Y., and Tait, J. A., 2000,
Siluro-Devonian paleomagnetic results from the Tuva Terrane (southern
Siberia, Russia): implications for the paleogeography of Siberia:
Journal of Geophysical Research: Solid Earth, vol.~105,
pp.~13,509-13,518, doi:10.1029/1999JB900429.

Cogne, J.-P., Kravchinsky, V. A., Halim, N., and Hankard, F., 2005, Late
Jurassic-Early Cretaceous closure of the Mongol-Okhotsk Ocean
demonstrated by new Mesozoic palaeomagnetic results from the
Trans-Baikal area (SE Siberia): Geophysical Journal International,
vol.~163, pp.~813-832, doi:10.1111/j.1365-246X.2005.02782.x.

Edel, J. B., Schulmann, K., Hanzl, P., and Lexa, O., 2014,
Palaeomagnetic and structural constraints on 90º anticlockwise rotation
in SW Mongolia during the Permo-Triassic: Implications for Altaid
oroclinal bending. Preliminary palaeomagnetic results: Journal of Asian
Earth Sciences, vol.~94, pp.~157-171, doi:10.1016/j.jseaes.2014.07.039.

Evans, D. A., Zhuravlev, A. Y., Budney, C. J., and Kirschvink, J. L.,
1996, Palaeomagnetism of the Bayan Gol Formation, western Mongolia:
Geological Magazine, vol.~133, pp.~487-496.

Grishin, D., Didenko, A., Pechersky, D., and T.L., T., 1991,
{[}Paleomagnetic study and petromagnetic study of structure and
evolution of paleooceanic lithosphere (Phanerozoic ophiolites of
Asia){]} (in Russian), VNIGRI, Leningrad, Russia, pp.~135-149.

Halim, N., Kravchinsky, V., Gilder, S., Cogne, J.-P., Alexyutin, M.,
Sorokin, A., Courtillot, V., and Chen, Y., 1998, A palaeomagnetic study
from the Mongol-Okhotsk region: rotated Early Cretaceous volcanics and
remagnetized Mesozoic sediments: Earth and Planetary Science Letters,
vol.~159, pp.~133-145, doi:10.1016/S0012-821X(98)00072-7.

Kovalenko, D., 2010, Paleomagnetism of Late Paleozoic, Mesozoic, and
Cenozoic rocks in Mongolia: Russian Geology and Geophysics, vol.~51,
pp.~387-403, doi:10.1016/j.rgg.2010.03.006.

Kravchinsky, V., Konstantinov, K., and Conge, J., 2001, Palaeomagnetic
study of Vendian and Early Cambrian rocks of South Siberia and Central
Mongolia: Was the Siberian platform assembled at this time?: Precambrian
Research, vol.~110, pp.~61-92.

Kravchinsky, V. A., Cogne, J.-P., Harbert, W. P., and Kuzmin, M. I.,
2002, Evolution of the Mongol-Okhotsk Ocean as constrained by new
palaeomagnetic data from the Mongol-Okhotsk suture zone, Siberia:
Geophysical Journal International, vol.~148, pp.~34-57.

Kravchinsky, V. A., Sklyarov, E. V., Gladkochub, D. P., and Harbert, W.
P., 2010, Paleomagnetism of the Precambrian Eastern Sayan rocks:
Implications for the Ediacaran-Early Cambrian paleogeography of the
Tuva-Mongolian composite terrane: Tectonophysics, vol.~486, pp.~65-80,
doi:10.1016/j.tecto.2010.02.010.

Pechersky, D. and Didenko, A., 1995, Paleo-Asian Ocean (in Russian):
United Institute of Earth's Physics Publ., Moscow, 298 pp.

Pruner, P., 1992, Palaeomagnetism and palaeogeography of Mongolia from
the Carboniferous to the Cretaceous - final report: Physics of the Earth
and Planetary Interiors, vol.~70, pp.~169-177,
doi:10.1016/0031-9201(92)90179-Y.

Van Hinsbergen, D. J. J., Straathof, G. B., Kuiper, K. F., Cunningham,
W. D., and Wijbrans, J., 2008, No vertical axis rotations during Neogene
transpressional orogeny in the NE Gobi Altai: coinciding Mongolian and
Eurasian early Cretaceous apparent polar wander paths: Geophysical
Journal International, vol.~173, pp.~105-126,
doi:10.1111/j.1365-246X.2007.03712.x.




    % Add a bibliography block to the postdoc
    
    
    
    \end{document}
