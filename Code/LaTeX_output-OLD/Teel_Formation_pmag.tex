
% Default to the notebook output style

    

% Default to the notebook output style


% Inherit from the specified cell style.




    
\documentclass[11pt]{article}

    \parskip = 0.1in
	\parindent = 0.0in
    
    \usepackage{rotating}
    \usepackage{graphicx} % Used to insert images
    \usepackage{adjustbox} % Used to constrain images to a maximum size 
    \usepackage{color} % Allow colors to be defined
    \usepackage{enumerate} % Needed for markdown enumerations to work
    \usepackage{geometry} % Used to adjust the document margins
    \usepackage{amsmath} % Equations
    \usepackage{amssymb} % Equations
    \usepackage{eurosym} % defines \euro
    \usepackage[mathletters]{ucs} % Extended unicode (utf-8) support
    \usepackage[utf8x]{inputenc} % Allow utf-8 characters in the tex document
    \usepackage{fancyvrb} % verbatim replacement that allows latex
    \usepackage{grffile} % extends the file name processing of package graphics 
                         % to support a larger range 
    % The hyperref package gives us a pdf with properly built
    % internal navigation ('pdf bookmarks' for the table of contents,
    % internal cross-reference links, web links for URLs, etc.)
    \usepackage{hyperref}
    \usepackage{longtable} % longtable support required by pandoc >1.10
    \usepackage{booktabs}  % table support for pandoc > 1.12.2
    \usepackage{ulem} % ulem is needed to support strikethroughs (\sout)
    \usepackage{url}

    
    
    \definecolor{orange}{cmyk}{0,0.4,0.8,0.2}
    \definecolor{darkorange}{rgb}{.71,0.21,0.01}
    \definecolor{darkgreen}{rgb}{.12,.54,.11}
    \definecolor{myteal}{rgb}{.26, .44, .56}
    \definecolor{gray}{gray}{0.45}
    \definecolor{lightgray}{gray}{.95}
    \definecolor{mediumgray}{gray}{.8}
    \definecolor{inputbackground}{rgb}{.95, .95, .85}
    \definecolor{outputbackground}{rgb}{.95, .95, .95}
    \definecolor{traceback}{rgb}{1, .95, .95}
    % ansi colors
    \definecolor{red}{rgb}{.6,0,0}
    \definecolor{green}{rgb}{0,.65,0}
    \definecolor{brown}{rgb}{0.6,0.6,0}
    \definecolor{blue}{rgb}{0,.145,.698}
    \definecolor{purple}{rgb}{.698,.145,.698}
    \definecolor{cyan}{rgb}{0,.698,.698}
    \definecolor{lightgray}{gray}{0.5}
    
    % bright ansi colors
    \definecolor{darkgray}{gray}{0.25}
    \definecolor{lightred}{rgb}{1.0,0.39,0.28}
    \definecolor{lightgreen}{rgb}{0.48,0.99,0.0}
    \definecolor{lightblue}{rgb}{0.53,0.81,0.92}
    \definecolor{lightpurple}{rgb}{0.87,0.63,0.87}
    \definecolor{lightcyan}{rgb}{0.5,1.0,0.83}
    
    % commands and environments needed by pandoc snippets
    % extracted from the output of `pandoc -s`
    \providecommand{\tightlist}{%
      \setlength{\itemsep}{0pt}\setlength{\parskip}{0pt}}
    \DefineVerbatimEnvironment{Highlighting}{Verbatim}{commandchars=\\\{\}}
    % Add ',fontsize=\small' for more characters per line
    \newenvironment{Shaded}{}{}
    \newcommand{\KeywordTok}[1]{\textcolor[rgb]{0.00,0.44,0.13}{\textbf{{#1}}}}
    \newcommand{\DataTypeTok}[1]{\textcolor[rgb]{0.56,0.13,0.00}{{#1}}}
    \newcommand{\DecValTok}[1]{\textcolor[rgb]{0.25,0.63,0.44}{{#1}}}
    \newcommand{\BaseNTok}[1]{\textcolor[rgb]{0.25,0.63,0.44}{{#1}}}
    \newcommand{\FloatTok}[1]{\textcolor[rgb]{0.25,0.63,0.44}{{#1}}}
    \newcommand{\CharTok}[1]{\textcolor[rgb]{0.25,0.44,0.63}{{#1}}}
    \newcommand{\StringTok}[1]{\textcolor[rgb]{0.25,0.44,0.63}{{#1}}}
    \newcommand{\CommentTok}[1]{\textcolor[rgb]{0.38,0.63,0.69}{\textit{{#1}}}}
    \newcommand{\OtherTok}[1]{\textcolor[rgb]{0.00,0.44,0.13}{{#1}}}
    \newcommand{\AlertTok}[1]{\textcolor[rgb]{1.00,0.00,0.00}{\textbf{{#1}}}}
    \newcommand{\FunctionTok}[1]{\textcolor[rgb]{0.02,0.16,0.49}{{#1}}}
    \newcommand{\RegionMarkerTok}[1]{{#1}}
    \newcommand{\ErrorTok}[1]{\textcolor[rgb]{1.00,0.00,0.00}{\textbf{{#1}}}}
    \newcommand{\NormalTok}[1]{{#1}}
    
    % Additional commands for more recent versions of Pandoc
    \newcommand{\ConstantTok}[1]{\textcolor[rgb]{0.53,0.00,0.00}{{#1}}}
    \newcommand{\SpecialCharTok}[1]{\textcolor[rgb]{0.25,0.44,0.63}{{#1}}}
    \newcommand{\VerbatimStringTok}[1]{\textcolor[rgb]{0.25,0.44,0.63}{{#1}}}
    \newcommand{\SpecialStringTok}[1]{\textcolor[rgb]{0.73,0.40,0.53}{{#1}}}
    \newcommand{\ImportTok}[1]{{#1}}
    \newcommand{\DocumentationTok}[1]{\textcolor[rgb]{0.73,0.13,0.13}{\textit{{#1}}}}
    \newcommand{\AnnotationTok}[1]{\textcolor[rgb]{0.38,0.63,0.69}{\textbf{\textit{{#1}}}}}
    \newcommand{\CommentVarTok}[1]{\textcolor[rgb]{0.38,0.63,0.69}{\textbf{\textit{{#1}}}}}
    \newcommand{\VariableTok}[1]{\textcolor[rgb]{0.10,0.09,0.49}{{#1}}}
    \newcommand{\ControlFlowTok}[1]{\textcolor[rgb]{0.00,0.44,0.13}{\textbf{{#1}}}}
    \newcommand{\OperatorTok}[1]{\textcolor[rgb]{0.40,0.40,0.40}{{#1}}}
    \newcommand{\BuiltInTok}[1]{{#1}}
    \newcommand{\ExtensionTok}[1]{{#1}}
    \newcommand{\PreprocessorTok}[1]{\textcolor[rgb]{0.74,0.48,0.00}{{#1}}}
    \newcommand{\AttributeTok}[1]{\textcolor[rgb]{0.49,0.56,0.16}{{#1}}}
    \newcommand{\InformationTok}[1]{\textcolor[rgb]{0.38,0.63,0.69}{\textbf{\textit{{#1}}}}}
    \newcommand{\WarningTok}[1]{\textcolor[rgb]{0.38,0.63,0.69}{\textbf{\textit{{#1}}}}}
    
    
    % Define a nice break command that doesn't care if a line doesn't already
    % exist.
    \def\br{\hspace*{\fill} \\* }
    % Math Jax compatability definitions
    \def\gt{>}
    \def\lt{<}
    % Document parameters
    \title{Teel\_Formation\_pmag}
    
    
    

    % Pygments definitions
    
\makeatletter
\def\PY@reset{\let\PY@it=\relax \let\PY@bf=\relax%
    \let\PY@ul=\relax \let\PY@tc=\relax%
    \let\PY@bc=\relax \let\PY@ff=\relax}
\def\PY@tok#1{\csname PY@tok@#1\endcsname}
\def\PY@toks#1+{\ifx\relax#1\empty\else%
    \PY@tok{#1}\expandafter\PY@toks\fi}
\def\PY@do#1{\PY@bc{\PY@tc{\PY@ul{%
    \PY@it{\PY@bf{\PY@ff{#1}}}}}}}
\def\PY#1#2{\PY@reset\PY@toks#1+\relax+\PY@do{#2}}

\expandafter\def\csname PY@tok@nv\endcsname{\def\PY@tc##1{\textcolor[rgb]{0.10,0.09,0.49}{##1}}}
\expandafter\def\csname PY@tok@nt\endcsname{\let\PY@bf=\textbf\def\PY@tc##1{\textcolor[rgb]{0.00,0.50,0.00}{##1}}}
\expandafter\def\csname PY@tok@mf\endcsname{\def\PY@tc##1{\textcolor[rgb]{0.40,0.40,0.40}{##1}}}
\expandafter\def\csname PY@tok@gp\endcsname{\let\PY@bf=\textbf\def\PY@tc##1{\textcolor[rgb]{0.00,0.00,0.50}{##1}}}
\expandafter\def\csname PY@tok@nf\endcsname{\def\PY@tc##1{\textcolor[rgb]{0.00,0.00,1.00}{##1}}}
\expandafter\def\csname PY@tok@gu\endcsname{\let\PY@bf=\textbf\def\PY@tc##1{\textcolor[rgb]{0.50,0.00,0.50}{##1}}}
\expandafter\def\csname PY@tok@vc\endcsname{\def\PY@tc##1{\textcolor[rgb]{0.10,0.09,0.49}{##1}}}
\expandafter\def\csname PY@tok@m\endcsname{\def\PY@tc##1{\textcolor[rgb]{0.40,0.40,0.40}{##1}}}
\expandafter\def\csname PY@tok@sh\endcsname{\def\PY@tc##1{\textcolor[rgb]{0.73,0.13,0.13}{##1}}}
\expandafter\def\csname PY@tok@cp\endcsname{\def\PY@tc##1{\textcolor[rgb]{0.74,0.48,0.00}{##1}}}
\expandafter\def\csname PY@tok@s1\endcsname{\def\PY@tc##1{\textcolor[rgb]{0.73,0.13,0.13}{##1}}}
\expandafter\def\csname PY@tok@w\endcsname{\def\PY@tc##1{\textcolor[rgb]{0.73,0.73,0.73}{##1}}}
\expandafter\def\csname PY@tok@mi\endcsname{\def\PY@tc##1{\textcolor[rgb]{0.40,0.40,0.40}{##1}}}
\expandafter\def\csname PY@tok@il\endcsname{\def\PY@tc##1{\textcolor[rgb]{0.40,0.40,0.40}{##1}}}
\expandafter\def\csname PY@tok@sx\endcsname{\def\PY@tc##1{\textcolor[rgb]{0.00,0.50,0.00}{##1}}}
\expandafter\def\csname PY@tok@sd\endcsname{\let\PY@it=\textit\def\PY@tc##1{\textcolor[rgb]{0.73,0.13,0.13}{##1}}}
\expandafter\def\csname PY@tok@sc\endcsname{\def\PY@tc##1{\textcolor[rgb]{0.73,0.13,0.13}{##1}}}
\expandafter\def\csname PY@tok@ow\endcsname{\let\PY@bf=\textbf\def\PY@tc##1{\textcolor[rgb]{0.67,0.13,1.00}{##1}}}
\expandafter\def\csname PY@tok@no\endcsname{\def\PY@tc##1{\textcolor[rgb]{0.53,0.00,0.00}{##1}}}
\expandafter\def\csname PY@tok@vg\endcsname{\def\PY@tc##1{\textcolor[rgb]{0.10,0.09,0.49}{##1}}}
\expandafter\def\csname PY@tok@si\endcsname{\let\PY@bf=\textbf\def\PY@tc##1{\textcolor[rgb]{0.73,0.40,0.53}{##1}}}
\expandafter\def\csname PY@tok@vi\endcsname{\def\PY@tc##1{\textcolor[rgb]{0.10,0.09,0.49}{##1}}}
\expandafter\def\csname PY@tok@c\endcsname{\let\PY@it=\textit\def\PY@tc##1{\textcolor[rgb]{0.25,0.50,0.50}{##1}}}
\expandafter\def\csname PY@tok@mo\endcsname{\def\PY@tc##1{\textcolor[rgb]{0.40,0.40,0.40}{##1}}}
\expandafter\def\csname PY@tok@cm\endcsname{\let\PY@it=\textit\def\PY@tc##1{\textcolor[rgb]{0.25,0.50,0.50}{##1}}}
\expandafter\def\csname PY@tok@sb\endcsname{\def\PY@tc##1{\textcolor[rgb]{0.73,0.13,0.13}{##1}}}
\expandafter\def\csname PY@tok@s2\endcsname{\def\PY@tc##1{\textcolor[rgb]{0.73,0.13,0.13}{##1}}}
\expandafter\def\csname PY@tok@kp\endcsname{\def\PY@tc##1{\textcolor[rgb]{0.00,0.50,0.00}{##1}}}
\expandafter\def\csname PY@tok@gh\endcsname{\let\PY@bf=\textbf\def\PY@tc##1{\textcolor[rgb]{0.00,0.00,0.50}{##1}}}
\expandafter\def\csname PY@tok@gi\endcsname{\def\PY@tc##1{\textcolor[rgb]{0.00,0.63,0.00}{##1}}}
\expandafter\def\csname PY@tok@nd\endcsname{\def\PY@tc##1{\textcolor[rgb]{0.67,0.13,1.00}{##1}}}
\expandafter\def\csname PY@tok@k\endcsname{\let\PY@bf=\textbf\def\PY@tc##1{\textcolor[rgb]{0.00,0.50,0.00}{##1}}}
\expandafter\def\csname PY@tok@ss\endcsname{\def\PY@tc##1{\textcolor[rgb]{0.10,0.09,0.49}{##1}}}
\expandafter\def\csname PY@tok@c1\endcsname{\let\PY@it=\textit\def\PY@tc##1{\textcolor[rgb]{0.25,0.50,0.50}{##1}}}
\expandafter\def\csname PY@tok@ne\endcsname{\let\PY@bf=\textbf\def\PY@tc##1{\textcolor[rgb]{0.82,0.25,0.23}{##1}}}
\expandafter\def\csname PY@tok@err\endcsname{\def\PY@bc##1{\setlength{\fboxsep}{0pt}\fcolorbox[rgb]{1.00,0.00,0.00}{1,1,1}{\strut ##1}}}
\expandafter\def\csname PY@tok@mh\endcsname{\def\PY@tc##1{\textcolor[rgb]{0.40,0.40,0.40}{##1}}}
\expandafter\def\csname PY@tok@na\endcsname{\def\PY@tc##1{\textcolor[rgb]{0.49,0.56,0.16}{##1}}}
\expandafter\def\csname PY@tok@gr\endcsname{\def\PY@tc##1{\textcolor[rgb]{1.00,0.00,0.00}{##1}}}
\expandafter\def\csname PY@tok@kn\endcsname{\let\PY@bf=\textbf\def\PY@tc##1{\textcolor[rgb]{0.00,0.50,0.00}{##1}}}
\expandafter\def\csname PY@tok@cs\endcsname{\let\PY@it=\textit\def\PY@tc##1{\textcolor[rgb]{0.25,0.50,0.50}{##1}}}
\expandafter\def\csname PY@tok@nc\endcsname{\let\PY@bf=\textbf\def\PY@tc##1{\textcolor[rgb]{0.00,0.00,1.00}{##1}}}
\expandafter\def\csname PY@tok@nb\endcsname{\def\PY@tc##1{\textcolor[rgb]{0.00,0.50,0.00}{##1}}}
\expandafter\def\csname PY@tok@ni\endcsname{\let\PY@bf=\textbf\def\PY@tc##1{\textcolor[rgb]{0.60,0.60,0.60}{##1}}}
\expandafter\def\csname PY@tok@go\endcsname{\def\PY@tc##1{\textcolor[rgb]{0.53,0.53,0.53}{##1}}}
\expandafter\def\csname PY@tok@kt\endcsname{\def\PY@tc##1{\textcolor[rgb]{0.69,0.00,0.25}{##1}}}
\expandafter\def\csname PY@tok@ch\endcsname{\let\PY@it=\textit\def\PY@tc##1{\textcolor[rgb]{0.25,0.50,0.50}{##1}}}
\expandafter\def\csname PY@tok@gs\endcsname{\let\PY@bf=\textbf}
\expandafter\def\csname PY@tok@sr\endcsname{\def\PY@tc##1{\textcolor[rgb]{0.73,0.40,0.53}{##1}}}
\expandafter\def\csname PY@tok@kd\endcsname{\let\PY@bf=\textbf\def\PY@tc##1{\textcolor[rgb]{0.00,0.50,0.00}{##1}}}
\expandafter\def\csname PY@tok@nn\endcsname{\let\PY@bf=\textbf\def\PY@tc##1{\textcolor[rgb]{0.00,0.00,1.00}{##1}}}
\expandafter\def\csname PY@tok@o\endcsname{\def\PY@tc##1{\textcolor[rgb]{0.40,0.40,0.40}{##1}}}
\expandafter\def\csname PY@tok@gd\endcsname{\def\PY@tc##1{\textcolor[rgb]{0.63,0.00,0.00}{##1}}}
\expandafter\def\csname PY@tok@gt\endcsname{\def\PY@tc##1{\textcolor[rgb]{0.00,0.27,0.87}{##1}}}
\expandafter\def\csname PY@tok@ge\endcsname{\let\PY@it=\textit}
\expandafter\def\csname PY@tok@s\endcsname{\def\PY@tc##1{\textcolor[rgb]{0.73,0.13,0.13}{##1}}}
\expandafter\def\csname PY@tok@mb\endcsname{\def\PY@tc##1{\textcolor[rgb]{0.40,0.40,0.40}{##1}}}
\expandafter\def\csname PY@tok@kr\endcsname{\let\PY@bf=\textbf\def\PY@tc##1{\textcolor[rgb]{0.00,0.50,0.00}{##1}}}
\expandafter\def\csname PY@tok@kc\endcsname{\let\PY@bf=\textbf\def\PY@tc##1{\textcolor[rgb]{0.00,0.50,0.00}{##1}}}
\expandafter\def\csname PY@tok@bp\endcsname{\def\PY@tc##1{\textcolor[rgb]{0.00,0.50,0.00}{##1}}}
\expandafter\def\csname PY@tok@cpf\endcsname{\let\PY@it=\textit\def\PY@tc##1{\textcolor[rgb]{0.25,0.50,0.50}{##1}}}
\expandafter\def\csname PY@tok@se\endcsname{\let\PY@bf=\textbf\def\PY@tc##1{\textcolor[rgb]{0.73,0.40,0.13}{##1}}}
\expandafter\def\csname PY@tok@nl\endcsname{\def\PY@tc##1{\textcolor[rgb]{0.63,0.63,0.00}{##1}}}

\def\PYZbs{\char`\\}
\def\PYZus{\char`\_}
\def\PYZob{\char`\{}
\def\PYZcb{\char`\}}
\def\PYZca{\char`\^}
\def\PYZam{\char`\&}
\def\PYZlt{\char`\<}
\def\PYZgt{\char`\>}
\def\PYZsh{\char`\#}
\def\PYZpc{\char`\%}
\def\PYZdl{\char`\$}
\def\PYZhy{\char`\-}
\def\PYZsq{\char`\'}
\def\PYZdq{\char`\"}
\def\PYZti{\char`\~}
% for compatibility with earlier versions
\def\PYZat{@}
\def\PYZlb{[}
\def\PYZrb{]}
\makeatother


    % Exact colors from NB
    \definecolor{incolor}{rgb}{0.0, 0.0, 0.5}
    \definecolor{outcolor}{rgb}{0.545, 0.0, 0.0}



    
    % Prevent overflowing lines due to hard-to-break entities
    \sloppy 
    % Setup hyperref package
    \hypersetup{
      breaklinks=true,  % so long urls are correctly broken across lines
      colorlinks=true,
      urlcolor=blue,
      linkcolor=darkorange,
      citecolor=darkgreen,
      }
    % Slightly bigger margins than the latex defaults
    
    \geometry{verbose,tmargin=1in,bmargin=1in,lmargin=1in,rmargin=1in}
    
    

    \begin{document}

    
{\Large Date repository materials for Paleomagnetism of the Late Ordovician Teel basalts of the
Zavkhan Terrane -- Paleozoic paleogeography of Mongolia}
    
\tableofcontents

    \section{Teel Volcanics Paleomagnetic Data
Analysis}\label{teel-volcanics-data-analysis}

    This notebook is the data repository for a manuscript entitled:

\textbf{Paleomagnetism of the Late Ordovician Teel basalts of the
Zavkhan Terrane -- Paleozoic paleogeography of Mongolia}

T. M. Kilian, N. L. Swanson-Hysell, U. Bold, J. Crowley, and F. A.
Macdonald

This manuscript is in review at the GSA journal Lithosphere.

This notebook contains data analysis related to paleomagnetic data
generated from basalt flows of the Late Ordovician Teel Formation from
the Zavkhan Terrane. The underlying data are availible both within the
MagIC Database and in the GitHub repository association with this work: \url{https://github.com/Swanson-Hysell-Group/2016\_Teel\_Basalts}

    \subsection{Import Modules}\label{import-modules}



    Write template file (no\_code.tpl) so that when the notebook is
converted to a latex (then pdf) it excludes the large code blocks. This
requires an additional arguement when using nbconveter and also requires
that adding tables of content term after the document begins; can also
add author. Can't include examples here because they affect the file
when in latex.


    \begin{Verbatim}[commandchars=\\\{\}]
Overwriting no\_code\_latex.tplx
    \end{Verbatim}

    \subsection{Sampling localities}\label{sampling-localities}

    Table of site locality coordinates given in WGS84 as well as a very
simple map. Exact stratigraphic positions are shown in main text; order
of stratigraphic position is given in ``stat\_pos'' column of table. The
paleomagnetic data from these sites may need to be tilt-corrected
(depending on the the age of magnetization) according to nearby
measurements of bedding. The bedding tilts used for sites Z30 to Z55
come from interflow sedimentary rocks. The tilt-corrections for Z56 to
Z58 are more uncertain and are based on interpretted flow banding.

\texttt{\color{outcolor}Out[{\color{outcolor}5}]:}
    
{\tiny\begin{tabular}{llllrrrrr}
\toprule
{} & er\_citation\_names & er\_location\_name & er\_site\_name &  site\_lat &  site\_lon &  strat\_pos &  sample\_bed\_dip\_direction &  sample\_bed\_dip \\
\midrule
0  &        This study &     Khukh\_Davaa  &          Z30 &  47.10038 &  95.37550 &        4.0 &                        88 &              58 \\
1  &        This study &     Khukh\_Davaa  &          Z31 &  47.10049 &  95.37604 &        5.0 &                        88 &              58 \\
2  &        This study &     Khukh\_Davaa  &          Z32 &  47.10094 &  95.37684 &        6.0 &                        84 &              55 \\
3  &        This study &     Khukh\_Davaa  &          Z33 &  47.10107 &  95.37705 &        7.0 &                        84 &              55 \\
4  &        This study &     Khukh\_Davaa  &          Z34 &  47.10111 &  95.37712 &        8.0 &                        84 &              55 \\
5  &        This study &     Khukh\_Davaa  &          Z35 &  47.10069 &  95.37747 &        9.0 &                        84 &              55 \\
6  &        This study &     Khukh\_Davaa  &          Z36 &  47.10221 &  95.37959 &       11.0 &                        89 &              47 \\
7  &        This study &     Khukh\_Davaa  &          Z37 &  47.10211 &  95.37971 &       12.0 &                        89 &              47 \\
8  &        This study &     Khukh\_Davaa  &          Z38 &  47.09855 &  95.38445 &       13.0 &                        87 &              46 \\
9  &        This study &     Khukh\_Davaa  &          Z39 &  47.09860 &  95.38467 &       14.0 &                        87 &              46 \\
10 &        This study &     Khukh\_Davaa  &          Z40 &  47.09859 &  95.38474 &       15.0 &                        87 &              46 \\
11 &        This study &     Khukh\_Davaa  &          Z41 &  47.10109 &  95.37744 &       10.0 &                        84 &              55 \\
12 &        This study &     Khukh\_Davaa  &          Z42 &  47.09577 &  95.38577 &       16.0 &                        87 &              37 \\
13 &        This study &     Khukh\_Davaa  &          Z43 &  47.09570 &  95.38638 &       17.0 &                        87 &              37 \\
14 &        This study &     Khukh\_Davaa  &          Z44 &  47.09571 &  95.38651 &       18.0 &                        87 &              37 \\
15 &        This study &     Khukh\_Davaa  &          Z45 &  47.09562 &  95.38676 &       19.0 &                        87 &              37 \\
16 &        This study &     Khukh\_Davaa  &          Z46 &  47.09563 &  95.38692 &       20.0 &                        87 &              37 \\
17 &        This study &     Khukh\_Davaa  &          Z47 &  47.09568 &  95.38727 &       21.0 &                        87 &              37 \\
18 &        This study &     Khukh\_Davaa  &          Z48 &  47.09570 &  95.38744 &       22.0 &                        91 &              28 \\
19 &        This study &     Khukh\_Davaa  &          Z49 &  47.09581 &  95.38747 &       23.0 &                        91 &              28 \\
20 &        This study &     Khukh\_Davaa  &          Z50 &  47.09575 &  95.38781 &       24.0 &                        91 &              28 \\
21 &        This study &     Khukh\_Davaa  &          Z51 &  47.09584 &  95.38802 &       25.0 &                        91 &              28 \\
22 &        This study &     Khukh\_Davaa  &          Z52 &  47.09583 &  95.38815 &       26.0 &                        91 &              28 \\
23 &        This study &     Khukh\_Davaa  &          Z53 &  47.09442 &  95.37205 &        1.0 &                        88 &              58 \\
24 &        This study &     Khukh\_Davaa  &          Z54 &  47.09502 &  95.37299 &        2.0 &                        88 &              58 \\
25 &        This study &     Khukh\_Davaa  &          Z55 &  47.09525 &  95.37351 &        3.0 &                        88 &              58 \\
26 &        This study &     Khukh\_Davaa  &          Z56 &  47.06403 &  95.42075 &        NaN &                       165 &              24 \\
27 &        This study &     Khukh\_Davaa  &          Z57 &  47.06277 &  95.42039 &        NaN &                       165 &              24 \\
28 &        This study &     Khukh\_Davaa  &          Z58 &  47.06277 &  95.42045 &        NaN &                       165 &              24 \\
\bottomrule
\end{tabular}}

    


    \begin{center}
    \adjustimage{max size={0.9\linewidth}{0.9\paperheight}}{Teel_Formation_pmag_files/Teel_Formation_pmag_11_0.pdf}
    \end{center}
    { \hspace*{\fill} \\}
    
    \subsection{Site level data}\label{site-level-data}

    Below we import paleomagnetic results that were analyzed using
demag\_gui.py from the PmagPy python package. These are the vector
component fits to all Teel sample data. Components have been classified
according to their relative temperature ranges. `LOW' components are
typically below 200ºC, `MAG' refers to a temperature range within the
unblocking range of magnetite (up to 580ºC), `HEM' refers to vector
components fit to data points in the unblocking range of hematite (up to
680ºC), and `MID' refers to components with temperature ranges between
`LOW' and `MAG'.


    \paragraph{Z30}\label{z30}

    Site Z30 is a rhyolite. The site has `LOW', `MAG' and poorly resolved
`HEM' components.



    \begin{center}
    \adjustimage{max size={0.9\linewidth}{0.9\paperheight}}{Teel_Formation_pmag_files/Teel_Formation_pmag_18_0.pdf}
    \end{center}
    { \hspace*{\fill} \\}
    
    \newpage
    
    \paragraph{Z31}\label{z31}

    The site has `LOW' (less than 200ºC) and `MAG' components.



    \begin{center}
    \adjustimage{max size={0.9\linewidth}{0.9\paperheight}}{Teel_Formation_pmag_files/Teel_Formation_pmag_22_0.pdf}
    \end{center}
    { \hspace*{\fill} \\}
    
    \newpage
    
    \paragraph{Z32}\label{z32}

    The site has `LOW' (less than 200ºC) and `MAG' components.



    \begin{center}
    \adjustimage{max size={0.9\linewidth}{0.9\paperheight}}{Teel_Formation_pmag_files/Teel_Formation_pmag_26_0.pdf}
    \end{center}
    { \hspace*{\fill} \\}
    
    \newpage
    
    \paragraph{Z33}\label{z33}

    The site has `LOW' (less than 200ºC) and `MAG' components.



    \begin{center}
    \adjustimage{max size={0.9\linewidth}{0.9\paperheight}}{Teel_Formation_pmag_files/Teel_Formation_pmag_30_0.pdf}
    \end{center}
    { \hspace*{\fill} \\}
    
    \newpage
    
    \paragraph{Z34}\label{z34}

    The site has `LOW' (less than 200ºC) and `MAG' components.



    \begin{center}
    \adjustimage{max size={0.9\linewidth}{0.9\paperheight}}{Teel_Formation_pmag_files/Teel_Formation_pmag_34_0.pdf}
    \end{center}
    { \hspace*{\fill} \\}
    
    \newpage
    
    \paragraph{Z35}\label{z35}

    The site has `LOW', `MAG' and `HEM' components.



    \begin{center}
    \adjustimage{max size={0.9\linewidth}{0.9\paperheight}}{Teel_Formation_pmag_files/Teel_Formation_pmag_38_0.pdf}
    \end{center}
    { \hspace*{\fill} \\}
    
    Data points shown with triangles were vectors from sample Z35.3 that
were dropped from the mean calculation.

\newpage

    \paragraph{Z36}\label{z36}

    The site has `LOW' and `MAG' components.



    \begin{center}
    \adjustimage{max size={0.9\linewidth}{0.9\paperheight}}{Teel_Formation_pmag_files/Teel_Formation_pmag_43_0.pdf}
    \end{center}
    { \hspace*{\fill} \\}
    
    \newpage
    
    \paragraph{Z37}\label{z37}

The site has `LOW' and `MAG' components.



    \begin{center}
    \adjustimage{max size={0.9\linewidth}{0.9\paperheight}}{Teel_Formation_pmag_files/Teel_Formation_pmag_46_0.pdf}
    \end{center}
    { \hspace*{\fill} \\}
    
    \newpage
    
    \paragraph{Z38}\label{z38}

    The site has `LOW', `MAG' and `HEM' components.



    \begin{center}
    \adjustimage{max size={0.9\linewidth}{0.9\paperheight}}{Teel_Formation_pmag_files/Teel_Formation_pmag_50_0.pdf}
    \end{center}
    { \hspace*{\fill} \\}
    
    Only two samples yielded hematite components, therefore no mean was
calculated for the hematite component.

\newpage

    \paragraph{Z39}\label{z39}

    The site has `LOW', `MAG' and `HEM' components.



    \begin{center}
    \adjustimage{max size={0.9\linewidth}{0.9\paperheight}}{Teel_Formation_pmag_files/Teel_Formation_pmag_55_0.pdf}
    \end{center}
    { \hspace*{\fill} \\}
    
    \newpage
    
    \paragraph{Z40}\label{z40}

    The site has `LOW', `MAG' and `HEM' components.



    \begin{center}
    \adjustimage{max size={0.9\linewidth}{0.9\paperheight}}{Teel_Formation_pmag_files/Teel_Formation_pmag_59_0.pdf}
    \end{center}
    { \hspace*{\fill} \\}
    
    \newpage
    
    \paragraph{Z41}\label{z41}

The site has `LOW', `MAG' and `HEM' components.



    \begin{center}
    \adjustimage{max size={0.9\linewidth}{0.9\paperheight}}{Teel_Formation_pmag_files/Teel_Formation_pmag_62_0.pdf}
    \end{center}
    { \hspace*{\fill} \\}
    
    \newpage
    
    \paragraph{Z42}\label{z42}

    The site has `LOW', `MAG' and `HEM' components. A middle temperature
component, MID, was also fit to demagnetization steps between `LOW' and
`MAG'.



    \begin{center}
    \adjustimage{max size={0.9\linewidth}{0.9\paperheight}}{Teel_Formation_pmag_files/Teel_Formation_pmag_66_0.pdf}
    \end{center}
    { \hspace*{\fill} \\}
    
    \newpage
    
    \paragraph{Z43}\label{z43}

    The site has `LOW', `MAG' and `HEM' components. A middle temperature
component, MID, was also fit to demagnetization steps between `LOW' and
`MAG'.



    \begin{center}
    \adjustimage{max size={0.9\linewidth}{0.9\paperheight}}{Teel_Formation_pmag_files/Teel_Formation_pmag_70_0.pdf}
    \end{center}
    { \hspace*{\fill} \\}
    
    \newpage
    
    \paragraph{Z44}\label{z44}

    The site has `LOW', `MAG' and `HEM' components.



    \begin{center}
    \adjustimage{max size={0.9\linewidth}{0.9\paperheight}}{Teel_Formation_pmag_files/Teel_Formation_pmag_74_0.pdf}
    \end{center}
    { \hspace*{\fill} \\}
    
    The remanence of Z44 specimens is dominated by hematite such that the
magnetite component is poorly resolved.

\newpage

    \paragraph{Z45}\label{z45}

    Hematite, magnetite, and low temperature, LOW (less than 200ºC),
components were calculated for Z45.



    \begin{center}
    \adjustimage{max size={0.9\linewidth}{0.9\paperheight}}{Teel_Formation_pmag_files/Teel_Formation_pmag_79_0.pdf}
    \end{center}
    { \hspace*{\fill} \\}
    
    The remanence of Z45 specimens is dominated by hematite such that the
magnetite component is poorly resolved.

\newpage

    \paragraph{Z46}\label{z46}

    The site has `LOW', `MAG' and `HEM' components.



    \begin{center}
    \adjustimage{max size={0.9\linewidth}{0.9\paperheight}}{Teel_Formation_pmag_files/Teel_Formation_pmag_84_0.pdf}
    \end{center}
    { \hspace*{\fill} \\}
    
    \newpage
    
    \paragraph{Z47}\label{z47}

    The site has `LOW', `MAG' and `HEM' components.



    \begin{center}
    \adjustimage{max size={0.9\linewidth}{0.9\paperheight}}{Teel_Formation_pmag_files/Teel_Formation_pmag_88_0.pdf}
    \end{center}
    { \hspace*{\fill} \\}
    
    \newpage
    
    \paragraph{Z48}\label{z48}

    The site has `LOW', `MAG' and `HEM' components.



    \begin{center}
    \adjustimage{max size={0.9\linewidth}{0.9\paperheight}}{Teel_Formation_pmag_files/Teel_Formation_pmag_92_0.pdf}
    \end{center}
    { \hspace*{\fill} \\}
    
    \newpage
    
    \paragraph{Z49}\label{z49}

    The site has `LOW', `MAG' and `HEM' components.



    \begin{center}
    \adjustimage{max size={0.9\linewidth}{0.9\paperheight}}{Teel_Formation_pmag_files/Teel_Formation_pmag_96_0.pdf}
    \end{center}
    { \hspace*{\fill} \\}
    
    \newpage
    
    \paragraph{Z50}\label{z50}

    The site has `MAG' and `HEM' components.



    \begin{center}
    \adjustimage{max size={0.9\linewidth}{0.9\paperheight}}{Teel_Formation_pmag_files/Teel_Formation_pmag_100_0.pdf}
    \end{center}
    { \hspace*{\fill} \\}
    
    \newpage
    
    \paragraph{Z51}\label{z51}

    The site has `LOW', `MAG' and `HEM' components.



    \begin{center}
    \adjustimage{max size={0.9\linewidth}{0.9\paperheight}}{Teel_Formation_pmag_files/Teel_Formation_pmag_104_0.pdf}
    \end{center}
    { \hspace*{\fill} \\}
    
    Magnetite components from two samples (Z51.1 and Z51.2) were excluded
because of their similarity to hematite components (the hematite
remanence mixed with that of magnetite) and different demagnetization
behavior compared to the other magnetite components.

\newpage

    \paragraph{Z52}\label{z52}

    The site has `LOW', `MAG' and `HEM' components.



    \begin{center}
    \adjustimage{max size={0.9\linewidth}{0.9\paperheight}}{Teel_Formation_pmag_files/Teel_Formation_pmag_109_0.pdf}
    \end{center}
    { \hspace*{\fill} \\}
    
    \newpage
    
    \paragraph{Z53}\label{z53}

    The site has `LOW', `MAG' and `HEM' components.



    \begin{center}
    \adjustimage{max size={0.9\linewidth}{0.9\paperheight}}{Teel_Formation_pmag_files/Teel_Formation_pmag_113_0.pdf}
    \end{center}
    { \hspace*{\fill} \\}
    
    \newpage
    
    \paragraph{Z54}\label{z54}

    The site has `LOW', `MAG' and `HEM' components.



    \begin{center}
    \adjustimage{max size={0.9\linewidth}{0.9\paperheight}}{Teel_Formation_pmag_files/Teel_Formation_pmag_117_0.pdf}
    \end{center}
    { \hspace*{\fill} \\}
    
    \newpage
    
    \paragraph{Z55}\label{z55}

    The site has `LOW' and `MAG'.



    \begin{center}
    \adjustimage{max size={0.9\linewidth}{0.9\paperheight}}{Teel_Formation_pmag_files/Teel_Formation_pmag_121_0.pdf}
    \end{center}
    { \hspace*{\fill} \\}
    
    \newpage
    
    \paragraph{Z56}\label{z56}

    The site has `LOW', `MAG' and `HEM' components.



    \begin{center}
    \adjustimage{max size={0.9\linewidth}{0.9\paperheight}}{Teel_Formation_pmag_files/Teel_Formation_pmag_125_0.pdf}
    \end{center}
    { \hspace*{\fill} \\}
    
    This site is in a different dip panel along with Z57 and Z58. A
magnetite component distinct from the hematite component was not
resolved. Z57 has similar behavior.

\newpage

    \paragraph{Z57}\label{z57}

    The site has `LOW', `MAG' and `HEM' components.



    \begin{center}
    \adjustimage{max size={0.9\linewidth}{0.9\paperheight}}{Teel_Formation_pmag_files/Teel_Formation_pmag_130_0.pdf}
    \end{center}
    { \hspace*{\fill} \\}
    
    \newpage
    
    \paragraph{Z58}\label{z58}

    Magnetite, mid-, and low- temperature, LOW (less than 200ºC), components
were calculated for Z58. The middle temperature component derives from
demagnetization steps between LOW and magnetite.



    \begin{center}
    \adjustimage{max size={0.9\linewidth}{0.9\paperheight}}{Teel_Formation_pmag_files/Teel_Formation_pmag_134_0.pdf}
    \end{center}
    { \hspace*{\fill} \\}
    
    Results from flow Z58 are very different from all of the other sites.

    \subsection{Paleomagnetic site data
summary}\label{paleomagnetic-site-data-summary}

    Create tables, distinguished by component type, of mean directions for
all Teel flows.

    \paragraph{Magnetite directions}\label{magnetite-directions}

    \subparagraph{Geographic coordinates -
magnetite}\label{geographic-coordinates---magnetite}

\texttt{\color{outcolor}Out[{\color{outcolor}66}]:}
    
    \begin{sidewaystable}
    {\tiny\begin{tabular}{lrrrrrrrrrrrrrrr}
\toprule
{} &  strat\_pos &  site\_lat &  site\_lon &     dec\_geo &    inc\_geo &    alpha95 &   n &           k &         r &        csd &  paleolatitude &    vgp\_lat &     vgp\_lon &  vgp\_lat\_rev &  vgp\_lon\_rev \\
\midrule
Z30\_mag\_geo &        4.0 &  47.10038 &  95.37550 &   61.199035 & -73.861110 &  11.712582 &   7 &   27.513985 &  6.781929 &  15.442168 &     -59.940264 & -28.018244 &  245.559596 &    28.018244 &    65.559596 \\
Z31\_mag\_geo &        5.0 &  47.10049 &  95.37604 &  168.103623 & -72.234668 &   3.753459 &   8 &  218.756351 &  7.968001 &   5.476520 &     -57.348450 & -77.463763 &  244.552326 &    77.463763 &    64.552326 \\
Z32\_mag\_geo &        6.0 &  47.10094 &  95.37684 &  170.358827 & -65.228366 &   7.987595 &   9 &   42.504477 &  8.811785 &  12.424178 &     -47.295204 & -83.450449 &  190.612182 &    83.450449 &    10.612182 \\
Z33\_mag\_geo &        7.0 &  47.10107 &  95.37705 &  184.611118 & -78.580341 &   4.579774 &   8 &  147.251468 &  7.952462 &   6.675060 &     -68.002034 & -68.966892 &  280.189909 &    68.966892 &   100.189909 \\
Z34\_mag\_geo &        8.0 &  47.10111 &  95.37712 &  165.710954 & -76.241270 &   2.846954 &  10 &  288.897760 &  9.968847 &   4.765549 &     -63.908090 & -71.445651 &  255.430852 &    71.445651 &    75.430852 \\
Z35\_mag\_geo &        9.0 &  47.10069 &  95.37747 &  180.035625 & -56.117492 &   6.392221 &   6 &  110.823428 &  5.954883 &   7.694302 &     -36.670242 & -79.569519 &   95.219636 &    79.569519 &   275.219636 \\
Z36\_mag\_geo &       11.0 &  47.10221 &  95.37959 &  184.602274 & -73.483650 &   4.160116 &   8 &  178.256399 &  7.960731 &   6.066839 &     -59.330031 & -77.472930 &  286.256471 &    77.472930 &   106.256471 \\
Z37\_mag\_geo &       12.0 &  47.10211 &  95.37971 &  172.808147 & -74.694707 &   7.476051 &   9 &   48.384817 &  8.834659 &  11.644758 &     -61.306496 & -75.207168 &  261.763513 &    75.207168 &    81.763513 \\
Z38\_mag\_geo &       13.0 &  47.09855 &  95.38445 &  170.143849 & -68.504682 &   8.040139 &  10 &   37.061501 &  9.757160 &  13.305265 &     -51.774894 & -82.077055 &  225.175390 &    82.077055 &    45.175390 \\
Z39\_mag\_geo &       14.0 &  47.09860 &  95.38467 &  196.103583 & -56.022026 &   6.360400 &   6 &  111.925588 &  5.955327 &   7.656324 &     -36.571555 & -74.094885 &   41.007252 &    74.094885 &   221.007252 \\
Z40\_mag\_geo &       15.0 &  47.09859 &  95.38474 &  171.882175 & -71.414505 &   4.837145 &   9 &  114.250096 &  8.929978 &   7.578037 &     -56.078745 & -79.716310 &  249.190583 &    79.716310 &    69.190583 \\
Z41\_mag\_geo &       10.0 &  47.10109 &  95.37744 &  175.528998 & -59.942821 &   4.580179 &   8 &  147.225628 &  7.952454 &   6.675646 &     -40.828095 & -82.952676 &  124.113660 &    82.952676 &   304.113660 \\
Z42\_mag\_geo &       16.0 &  47.09577 &  95.38577 &  196.391432 & -44.308602 &  11.150650 &   8 &   25.632057 &  7.726904 &  15.999017 &     -26.015899 & -65.259739 &   58.086778 &    65.259739 &   238.086778 \\
Z43\_mag\_geo &       17.0 &  47.09570 &  95.38638 &  188.500123 & -36.936066 &  10.749181 &   8 &   27.509935 &  7.745546 &  15.443305 &     -20.601392 & -62.620370 &   77.877325 &    62.620370 &   257.877325 \\
Z44\_mag\_geo &       18.0 &  47.09571 &  95.38651 &  177.320106 &  25.952966 &  12.333051 &   5 &   39.443485 &  4.898589 &  12.897258 &      13.677537 & -29.179269 &   98.369133 &    29.179269 &   278.369133 \\
Z45\_mag\_geo &       19.0 &  47.09562 &  95.38676 &  185.281245 &  22.577189 &   8.142853 &   8 &   47.231012 &  7.851792 &  11.786134 &      11.744294 & -30.970825 &   89.353703 &    30.970825 &   269.353703 \\
Z46\_mag\_geo &       20.0 &  47.09563 &  95.38692 &  209.744992 & -58.036511 &   6.728463 &   8 &   68.732148 &  7.898155 &   9.770236 &     -38.705303 & -66.823872 &   15.721243 &    66.823872 &   195.721243 \\
Z47\_mag\_geo &       21.0 &  47.09568 &  95.38727 &  208.133979 & -59.587799 &   5.606451 &   8 &   98.575940 &  7.928989 &   8.158298 &     -40.424813 & -68.748397 &   13.361225 &    68.748397 &   193.361225 \\
Z48\_mag\_geo &       22.0 &  47.09570 &  95.38744 &  207.921657 & -49.345442 &   6.760918 &   8 &   68.082983 &  7.897184 &   9.816705 &     -30.209549 & -62.672462 &   33.565359 &    62.672462 &   213.565359 \\
Z49\_mag\_geo &       23.0 &  47.09581 &  95.38747 &  213.177701 & -57.526629 &   9.488165 &   7 &   41.429419 &  6.855175 &  12.584345 &     -38.154890 & -64.234334 &   13.523120 &    64.234334 &   193.523120 \\
Z50\_mag\_geo &       24.0 &  47.09575 &  95.38781 &  183.717358 &  16.532802 &   3.910452 &   8 &  201.619149 &  7.965281 &   5.704520 &       8.442107 & -34.363744 &   90.931921 &    34.363744 &   270.931921 \\
Z51\_mag\_geo &       25.0 &  47.09584 &  95.38802 &  206.717692 & -54.030732 &  10.393892 &   5 &   55.146896 &  4.927466 &  10.907481 &     -34.565590 & -66.394972 &   27.782142 &    66.394972 &   207.782142 \\
Z52\_mag\_geo &       26.0 &  47.09583 &  95.38815 &  182.835182 & -55.213178 &  12.719642 &   6 &   28.697188 &  5.825767 &  15.120472 &     -35.744789 & -78.453731 &   83.817965 &    78.453731 &   263.817965 \\
Z53\_mag\_geo &        1.0 &  47.09442 &  95.37205 &  172.278467 & -71.983776 &   3.594132 &   8 &  238.494787 &  7.970649 &   5.245001 &     -56.957385 & -79.068440 &  252.646197 &    79.068440 &    72.646197 \\
Z54\_mag\_geo &        2.0 &  47.09502 &  95.37299 &  170.318109 & -69.453189 &   8.193477 &   8 &   46.660914 &  7.849982 &  11.857917 &     -53.143586 & -81.346675 &  233.270089 &    81.346675 &    53.270089 \\
Z55\_mag\_geo &        3.0 &  47.09525 &  95.37351 &  166.386553 & -64.854630 &   5.625088 &   8 &   97.930102 &  7.928520 &   8.185155 &     -46.807981 & -80.714545 &  188.596346 &    80.714545 &     8.596346 \\
Z56\_mag\_geo &        NaN &  47.06403 &  95.42075 &  179.680351 &  -4.756236 &   5.057226 &   6 &  176.489683 &  5.971670 &   6.097129 &      -2.382222 & -45.317329 &   95.874936 &    45.317329 &   275.874936 \\
Z57\_mag\_geo &        NaN &  47.06277 &  95.42039 &  178.786826 &  -8.456342 &  13.052562 &   6 &   27.299686 &  5.816848 &  15.502659 &      -4.251323 & -47.175716 &   97.200362 &    47.175716 &   277.200362 \\
Z58\_mag\_geo &        NaN &  47.06277 &  95.42045 &  128.167765 & -19.352349 &  17.533117 &   6 &   15.552391 &  5.678506 &  20.539338 &      -9.960183 & -32.768009 &  162.478280 &    32.768009 &   342.478280 \\
\bottomrule
\end{tabular}}
\end{sidewaystable}
    


    \begin{center}
    \adjustimage{max size={0.9\linewidth}{0.9\paperheight}}{Teel_Formation_pmag_files/Teel_Formation_pmag_141_0.pdf}
    \end{center}
    { \hspace*{\fill} \\}
    

    \begin{center}
    \adjustimage{max size={0.9\linewidth}{0.9\paperheight}}{Teel_Formation_pmag_files/Teel_Formation_pmag_142_0.pdf}
    \end{center}
    { \hspace*{\fill} \\}
    
    \subparagraph{Tilt-corrected coordinates -
magnetite}\label{tilt-corrected-coordinates---magnetite}

\texttt{\color{outcolor}Out[{\color{outcolor}69}]:}
    
\begin{sidewaystable}
    {\tiny\begin{tabular}{lrrrrrrrrrrrrrrr}
\toprule
{} &  strat\_pos &  site\_lat &  site\_lon &      dec\_tc &     inc\_tc &    alpha95 &   n &           k &         r &        csd &  paleolatitude &    vgp\_lat &     vgp\_lon &  vgp\_lat\_rev &  vgp\_lon\_rev \\
\midrule
Z30\_mag &        4.0 &  47.10038 &  95.37550 &  278.379367 & -46.010280 &  11.715664 &   7 &   27.500012 &  6.781818 &  15.446091 &     -27.381947 & -14.408220 &  340.470733 &    14.408220 &   160.470733 \\
Z31\_mag &        5.0 &  47.10049 &  95.37604 &  246.928001 & -33.319165 &   3.753646 &   8 &  218.734693 &  7.967998 &   5.476791 &     -18.194560 & -28.826591 &    9.314505 &    28.826591 &   189.314505 \\
Z32\_mag &        6.0 &  47.10094 &  95.37684 &  234.132395 & -32.854953 &   7.997970 &   9 &   42.396758 &  8.811306 &  12.439952 &     -17.895640 & -37.203135 &   19.865666 &    37.203135 &   199.865666 \\
Z33\_mag &        7.0 &  47.10107 &  95.37705 &  250.685156 & -32.176114 &   4.581718 &   8 &  147.127336 &  7.952422 &   6.677876 &     -17.462312 & -25.759713 &    7.086542 &    25.759713 &   187.086542 \\
Z34\_mag &        8.0 &  47.10111 &  95.37712 &  247.131951 & -35.840117 &   2.844796 &  10 &  289.334789 &  9.968894 &   4.761948 &     -19.856873 & -29.843343 &    7.791244 &    29.843343 &   187.791244 \\
Z35\_mag &        9.0 &  47.10069 &  95.37747 &  226.181747 & -25.332749 &   6.399180 &   6 &  110.584559 &  5.954786 &   7.702608 &     -13.316749 & -38.856026 &   31.001433 &    38.856026 &   211.001433 \\
Z36\_mag &       11.0 &  47.10221 &  95.37959 &  247.531592 & -39.317142 &   4.170654 &   8 &  177.361534 &  7.960533 &   6.082124 &     -22.268860 & -31.221631 &    5.455763 &    31.221631 &   185.455763 \\
Z37\_mag &       12.0 &  47.10211 &  95.37971 &  248.052322 & -42.739898 &   7.478423 &   9 &   48.354733 &  8.834556 &  11.648380 &     -24.798438 & -32.562210 &    2.867308 &    32.562210 &   182.867308 \\
Z38\_mag &       13.0 &  47.09855 &  95.38445 &  237.351592 & -42.681665 &   8.027422 &  10 &   37.175977 &  9.757908 &  13.284763 &     -24.753981 & -39.808772 &   10.899760 &    39.808772 &   190.899760 \\
Z39\_mag &       14.0 &  47.09860 &  95.38467 &  230.880551 & -26.400672 &   6.363220 &   6 &  111.827254 &  5.955288 &   7.659690 &     -13.939647 & -36.393048 &   26.087975 &    36.393048 &   206.087975 \\
Z40\_mag &       15.0 &  47.09859 &  95.38474 &  241.380972 & -42.753133 &   4.822532 &   9 &  114.937717 &  8.930397 &   7.555335 &     -24.808549 & -37.108819 &    7.746124 &    37.108819 &   187.746124 \\
Z41\_mag &       10.0 &  47.10109 &  95.37744 &  229.065998 & -29.074319 &   4.590751 &   8 &  146.552710 &  7.952236 &   6.690955 &     -15.536089 & -38.748850 &   26.425574 &    38.748850 &   206.425574 \\
Z42\_mag &       16.0 &  47.09577 &  95.38577 &  219.102263 & -24.487183 &  11.152830 &   8 &   25.622407 &  7.726802 &  16.002029 &     -12.829071 & -42.668310 &   38.628730 &    42.668310 &   218.628730 \\
Z43\_mag &       17.0 &  47.09570 &  95.38638 &  208.946090 & -22.598975 &  10.757156 &   8 &   27.470571 &  7.745182 &  15.454365 &     -11.756542 & -47.094786 &   51.278792 &    47.094786 &   231.278792 \\
Z44\_mag &       18.0 &  47.09571 &  95.38651 &  160.933464 &  20.616106 &  12.332689 &   5 &   39.445745 &  4.898595 &  12.896888 &      10.652755 & -29.797321 &  117.098874 &    29.797321 &   297.098874 \\
Z45\_mag &       19.0 &  47.09562 &  95.38676 &  169.181253 &  22.723096 &   8.151115 &   8 &   47.137245 &  7.851497 &  11.797851 &      11.826377 & -30.288956 &  107.671016 &    30.288956 &   287.671016 \\
Z46\_mag &       20.0 &  47.09563 &  95.38692 &  235.938073 & -30.336640 &   6.723991 &   8 &   68.822329 &  7.898289 &   9.763833 &     -16.309766 & -34.865831 &   19.680166 &    34.865831 &   199.680166 \\
Z47\_mag &       21.0 &  47.09568 &  95.38727 &  236.225390 & -32.077321 &   5.605537 &   8 &   98.607774 &  7.929012 &   8.156981 &     -17.399618 & -35.463534 &   18.516120 &    35.463534 &   198.516120 \\
Z48\_mag &       22.0 &  47.09570 &  95.38744 &  227.700040 & -32.087392 &   6.771057 &   8 &   67.882078 &  7.896880 &   9.831221 &     -17.406005 & -41.018935 &   26.092301 &    41.018935 &   206.092301 \\
Z49\_mag &       23.0 &  47.09581 &  95.38747 &  235.973353 & -37.612989 &   9.479951 &   7 &   41.499592 &  6.855420 &  12.573700 &     -21.068304 & -38.228444 &   15.487683 &    38.228444 &   195.487683 \\
Z50\_mag &       24.0 &  47.09575 &  95.38781 &  175.405740 &  15.829522 &   3.922362 &   8 &  200.402350 &  7.965070 &   5.721812 &       8.068715 & -34.684493 &  100.922274 &    34.684493 &   280.922274 \\
Z51\_mag &       25.0 &  47.09584 &  95.38802 &  229.796663 & -36.494284 &  10.382678 &   5 &   55.264038 &  4.927620 &  10.895915 &     -20.299603 & -41.779935 &   21.526652 &    41.779935 &   201.526652 \\
Z52\_mag &       26.0 &  47.09583 &  95.38815 &  216.149983 & -45.789543 &  12.698064 &   6 &   28.791577 &  5.826338 &  15.095667 &     -27.201966 & -55.462438 &   27.657758 &    55.462438 &   207.657758 \\
Z53\_mag &        1.0 &  47.09442 &  95.37205 &  246.701511 & -31.999187 &   3.599051 &   8 &  237.845848 &  7.970569 &   5.252151 &     -17.350100 & -28.388510 &   10.165158 &    28.388510 &   190.165158 \\
Z54\_mag &        2.0 &  47.09502 &  95.37299 &  243.678159 & -32.414475 &   8.199873 &   8 &   46.589636 &  7.849752 &  11.866984 &     -17.613960 & -30.621692 &   12.284724 &    30.621692 &   192.284724 \\
Z55\_mag &        3.0 &  47.09525 &  95.37351 &  238.040663 & -33.518307 &   5.611932 &   8 &   98.385329 &  7.928851 &   8.166197 &     -18.323455 & -34.914855 &   16.200615 &    34.914855 &   196.200615 \\
Z56\_mag &        NaN &  47.06403 &  95.42075 &  181.601802 & -27.881082 &   5.069905 &   6 &  175.612751 &  5.971528 &   6.112333 &     -14.816796 & -57.725144 &   92.519901 &    57.725144 &   272.519901 \\
Z57\_mag &        NaN &  47.06277 &  95.42039 &  181.062329 & -31.634060 &  13.082813 &   6 &   27.177964 &  5.816027 &  15.537336 &     -17.119456 & -60.043844 &   93.386864 &    60.043844 &   273.386864 \\
Z58\_mag &        NaN &  47.06277 &  95.42045 &  119.444412 & -37.602921 &  17.552705 &   6 &   15.519815 &  5.677831 &  20.560882 &     -21.061319 & -35.140582 &  179.028128 &    35.140582 &   359.028128 \\
\bottomrule
\end{tabular}}
\end{sidewaystable}
    


    \begin{center}
    \adjustimage{max size={0.9\linewidth}{0.9\paperheight}}{Teel_Formation_pmag_files/Teel_Formation_pmag_145_0.pdf}
    \end{center}
    { \hspace*{\fill} \\}
    

    \begin{center}
    \adjustimage{max size={0.9\linewidth}{0.9\paperheight}}{Teel_Formation_pmag_files/Teel_Formation_pmag_146_0.pdf}
    \end{center}
    { \hspace*{\fill} \\}
    
    \paragraph{Hematite directions}\label{hematite-directions}

    \subparagraph{Geographic coordinates -
hematite}\label{geographic-coordinates---hematite}

\texttt{\color{outcolor}Out[{\color{outcolor}72}]:}
\begin{sidewaystable}    
    {\tiny\begin{tabular}{lrrrrrrrrrrrrrrr}
\toprule
{} &  strat\_pos &  site\_lat &  site\_lon &     dec\_geo &    inc\_geo &    alpha95 &  n &           k &         r &        csd &  paleolatitude &    vgp\_lat &     vgp\_lon &  vgp\_lat\_rev &  vgp\_lon\_rev \\
\midrule
Z35\_hem\_geo &        4.0 &  47.10038 &  95.37550 &  185.352162 &  -5.560896 &   5.747798 &  6 &  136.842402 &  5.963462 &   6.924281 &      -2.787011 & -45.444034 &   87.744611 &    45.444034 &   267.744611 \\
Z30\_hem\_geo &        9.0 &  47.10069 &  95.37747 &  148.799913 & -68.209834 &  49.957167 &  7 &    2.413452 &  4.513935 &  52.139357 &     -51.356099 & -69.351223 &  208.833918 &    69.351223 &    28.833918 \\
Z38\_hem\_geo &       13.0 &  47.09855 &  95.38445 &  195.565071 & -26.364592 &   8.833229 &  2 &  801.477440 &  1.998752 &   2.861142 &     -13.918471 & -54.362780 &   68.832083 &    54.362780 &   248.832083 \\
Z39\_hem\_geo &       14.0 &  47.09860 &  95.38467 &  196.313095 & -15.623428 &   8.663295 &  5 &   78.962990 &  4.949343 &   9.115347 &      -7.959659 & -48.458523 &   70.583219 &    48.458523 &   250.583219 \\
Z40\_hem\_geo &       15.0 &  47.09859 &  95.38474 &  189.173033 & -23.507115 &   7.794855 &  7 &   60.926778 &  6.901521 &  10.377217 &     -12.269627 & -54.326542 &   79.891757 &    54.326542 &   259.891757 \\
Z41\_hem\_geo &       10.0 &  47.10109 &  95.37744 &  178.834997 & -26.225350 &  16.086290 &  5 &   23.576274 &  4.830338 &  16.681974 &     -13.836835 & -56.721476 &   97.439300 &    56.721476 &   277.439300 \\
Z42\_hem\_geo &       16.0 &  47.09577 &  95.38577 &  165.448451 &  27.278167 &  16.388007 &  6 &   17.664613 &  5.716948 &  19.272274 &      14.457588 & -27.077402 &  111.243174 &    27.077402 &   291.243174 \\
Z43\_hem\_geo &       17.0 &  47.09570 &  95.38638 &  182.753379 &  12.734651 &   6.220142 &  8 &   80.262873 &  7.912787 &   9.041233 &       6.446942 & -36.401745 &   91.986451 &    36.401745 &   271.986451 \\
Z44\_hem\_geo &       18.0 &  47.09571 &  95.38651 &  182.138846 &  26.690156 &   2.023836 &  7 &  890.663468 &  6.993263 &   2.714115 &      14.109891 & -28.764330 &   93.020083 &    28.764330 &   273.020083 \\
Z45\_hem\_geo &       19.0 &  47.09562 &  95.38676 &  179.431606 &  23.111384 &   4.657126 &  8 &  142.431974 &  7.950854 &   6.787053 &      12.045434 & -30.856760 &   96.034300 &    30.856760 &   276.034300 \\
Z46\_hem\_geo &       20.0 &  47.09563 &  95.38692 &  199.630710 &  23.039855 &  43.076043 &  2 &   35.743651 &  1.972023 &  13.548324 &      12.005012 & -28.348190 &   73.462329 &    28.348190 &   253.462329 \\
Z47\_hem\_geo &       21.0 &  47.09568 &  95.38727 &  186.530284 &  -3.818259 &  14.407211 &  7 &   18.506577 &  6.675791 &  18.828771 &      -1.911251 & -44.460110 &   86.223719 &    44.460110 &   266.223719 \\
Z48\_hem\_geo &       22.0 &  47.09570 &  95.38744 &  186.680865 &  20.745840 &   5.248580 &  8 &  112.342494 &  7.937691 &   7.642105 &      10.724316 & -31.873027 &   87.651814 &    31.873027 &   267.651814 \\
Z49\_hem\_geo &       23.0 &  47.09581 &  95.38747 &  201.506939 &   6.066986 &  11.535574 &  6 &   34.686217 &  5.855851 &  13.753288 &       3.042020 & -36.413576 &   68.327763 &    36.413576 &   248.327763 \\
Z50\_hem\_geo &       24.0 &  47.09575 &  95.38781 &  185.576900 &  22.848691 &   3.695319 &  8 &  225.663876 &  7.968980 &   5.392051 &      11.897134 & -30.796567 &   89.031882 &    30.796567 &   269.031882 \\
Z51\_hem\_geo &       25.0 &  47.09584 &  95.38802 &  184.630448 &  22.563173 &  10.926624 &  5 &   49.990867 &  4.919985 &  11.456176 &      11.736415 & -31.022182 &   90.095926 &    31.022182 &   270.095926 \\
Z52\_hem\_geo &       26.0 &  47.09583 &  95.38815 &  182.777548 &  21.100104 &  13.120981 &  6 &   27.025581 &  5.814990 &  15.581078 &      10.920183 & -31.930956 &   92.174224 &    31.930956 &   272.174224 \\
Z53\_hem\_geo &        1.0 &  47.09442 &  95.37205 &  176.883446 & -21.454440 &   5.750351 &  7 &  111.157505 &  6.946023 &   7.682731 &     -11.116766 & -53.926099 &  100.570110 &    53.926099 &   280.570110 \\
Z54\_hem\_geo &        2.0 &  47.09502 &  95.37299 &  166.032855 & -40.835976 &  52.801938 &  8 &    2.056913 &  4.596843 &  56.477701 &     -23.370939 & -63.768703 &  125.457583 &    63.768703 &   305.457583 \\
Z56\_hem\_geo &        NaN &  47.06403 &  95.42075 &  179.194265 &   1.511589 &   8.593573 &  6 &   61.741195 &  5.919017 &  10.308548 &       0.755926 & -42.174837 &   96.507901 &    42.174837 &   276.507901 \\
Z57\_hem\_geo &        NaN &  47.06277 &  95.42039 &  181.947932 &   4.332196 &   7.473594 &  4 &  152.109621 &  3.980277 &   6.567600 &       2.169198 & -40.738280 &   92.851012 &    40.738280 &   272.851012 \\
\bottomrule
\end{tabular}}
\end{sidewaystable}
    


    \begin{center}
    \adjustimage{max size={0.9\linewidth}{0.9\paperheight}}{Teel_Formation_pmag_files/Teel_Formation_pmag_150_0.pdf}
    \end{center}
    { \hspace*{\fill} \\}
    

    \begin{center}
    \adjustimage{max size={0.9\linewidth}{0.9\paperheight}}{Teel_Formation_pmag_files/Teel_Formation_pmag_151_0.pdf}
    \end{center}
    { \hspace*{\fill} \\}
    
    \subparagraph{Tilt-corrected coordinates -
hematite}\label{tilt-corrected-coordinates---hematite}

\texttt{\color{outcolor}Out[{\color{outcolor}75}]:}
    \begin{sidewaystable}
    {\tiny\begin{tabular}{lrrrrrrrrrrrrrrr}
\toprule
{} &  strat\_pos &  site\_lat &  site\_lon &      dec\_tc &     inc\_tc &    alpha95 &  n &           k &         r &        csd &  paleolatitude &    vgp\_lat &     vgp\_lon &  vgp\_lat\_rev &  vgp\_lon\_rev \\
\midrule
Z35\_hem &        4.0 &  47.10038 &  95.37550 &  185.114311 &   6.025809 &   5.727598 &  6 &  137.802632 &  5.963716 &   6.900114 &       3.021259 & -39.676603 &   88.733786 &    39.676603 &   268.733786 \\
Z30\_hem &        9.0 &  47.10069 &  95.37747 &  242.928054 & -40.237690 &  49.931876 &  7 &    2.414928 &  4.515454 &  52.123423 &     -22.933069 & -34.802858 &    8.295178 &    34.802858 &   188.295178 \\
Z38\_hem &       13.0 &  47.09855 &  95.38445 &  208.352571 &  -5.900502 &   8.908547 &  2 &  788.017587 &  1.998731 &   2.885473 &      -2.958094 & -39.500385 &   57.459211 &    39.500385 &   237.459211 \\
Z39\_hem &       14.0 &  47.09860 &  95.38467 &  201.539571 &   2.413034 &   8.658960 &  5 &   79.041121 &  4.949393 &   9.110841 &       1.207052 & -38.143076 &   67.563030 &    38.143076 &   247.563030 \\
Z40\_hem &       15.0 &  47.09859 &  95.38474 &  202.157748 &  -7.928196 &   7.788149 &  7 &   61.030111 &  6.901688 &  10.368429 &      -3.983165 & -42.830083 &   64.518459 &    42.830083 &   244.518459 \\
Z41\_hem &       10.0 &  47.10109 &  95.37744 &  198.411151 & -11.058439 &  16.113432 &  5 &   23.500114 &  4.829788 &  16.708984 &      -5.581195 & -45.565273 &   68.698517 &    45.565273 &   248.698517 \\
Z42\_hem &       16.0 &  47.09577 &  95.38577 &  151.386122 &  14.995751 &  16.374487 &  6 &   17.692229 &  5.717390 &  19.257227 &       7.628503 & -29.676516 &  128.501026 &    29.676516 &   308.501026 \\
Z43\_hem &       17.0 &  47.09570 &  95.38638 &  173.787706 &  13.587096 &   6.227989 &  8 &   80.063147 &  7.912569 &   9.052503 &       6.890414 & -35.733259 &  102.991555 &    35.733259 &   282.991555 \\
Z44\_hem &       18.0 &  47.09571 &  95.38651 &  163.946455 &  23.981346 &   2.033549 &  7 &  882.184819 &  6.993199 &   2.727126 &      12.539568 & -28.658505 &  113.302682 &    28.658505 &   293.302682 \\
Z45\_hem &       19.0 &  47.09562 &  95.38676 &  164.416633 &  19.689336 &   4.665334 &  8 &  141.934618 &  7.950682 &   6.798934 &      10.144077 & -31.097230 &  113.374942 &    31.097230 &   293.374942 \\
Z46\_hem &       20.0 &  47.09563 &  95.38692 &  180.134056 &  31.708243 &  43.051197 &  2 &   35.782409 &  1.972053 &  13.540984 &      17.166216 & -25.738041 &   95.244729 &    25.738041 &   275.244729 \\
Z47\_hem &       21.0 &  47.09568 &  95.38727 &  186.913270 &   2.679432 &  14.418197 &  7 &   18.479833 &  6.675322 &  18.842390 &       1.340449 & -41.186055 &   86.186406 &    41.186055 &   266.186406 \\
Z48\_hem &       22.0 &  47.09570 &  95.38744 &  175.819816 &  20.880528 &   5.256939 &  8 &  111.988554 &  7.937494 &   7.654172 &      10.798704 & -31.985346 &  100.230031 &    31.985346 &   280.230031 \\
Z49\_hem &       23.0 &  47.09581 &  95.38747 &  196.482372 &  14.863765 &  11.529913 &  6 &   34.719353 &  5.855988 &  13.746724 &       7.559053 & -33.419750 &   75.695091 &    33.419750 &   255.695091 \\
Z50\_hem &       24.0 &  47.09575 &  95.38781 &  173.713227 &  22.176830 &   3.699165 &  8 &  225.196841 &  7.968916 &   5.397639 &      11.519700 & -31.115698 &  102.587676 &    31.115698 &   282.587676 \\
Z51\_hem &       25.0 &  47.09584 &  95.38802 &  173.080330 &  21.478208 &  10.923883 &  5 &   50.015473 &  4.920025 &  11.453358 &      11.129977 & -31.446845 &  103.352735 &    31.446845 &   283.352735 \\
Z52\_hem &       26.0 &  47.09583 &  95.38815 &  172.236460 &  19.385124 &  13.100041 &  6 &   27.109018 &  5.815560 &  15.557081 &       9.978044 & -32.507612 &  104.465028 &    32.507612 &   284.465028 \\
Z53\_hem &        1.0 &  47.09442 &  95.37205 &  195.874742 & -12.092261 &   5.754882 &  7 &  110.984038 &  6.945938 &   7.688733 &      -6.114213 & -46.812655 &   71.955927 &    46.812655 &   251.955927 \\
Z54\_hem &        2.0 &  47.09502 &  95.37299 &  210.517882 & -28.621488 &  52.793211 &  8 &    2.057276 &  4.597443 &  56.472722 &     -15.261779 & -49.341300 &   46.618190 &    49.341300 &   226.618190 \\
Z56\_hem &        NaN &  47.06403 &  95.42075 &  180.301491 & -21.711722 &   8.592210 &  6 &   61.760484 &  5.919042 &  10.306938 &     -11.259937 & -54.195001 &   94.915322 &    54.195001 &   274.915322 \\
Z57\_hem &        NaN &  47.06277 &  95.42039 &  182.841002 & -18.592084 &   7.462069 &  4 &  152.576865 &  3.980338 &   6.557536 &      -9.547317 & -52.406934 &   90.824846 &    52.406934 &   270.824846 \\
\bottomrule
\end{tabular}}
\end{sidewaystable}
    


    \begin{center}
    \adjustimage{max size={0.9\linewidth}{0.9\paperheight}}{Teel_Formation_pmag_files/Teel_Formation_pmag_154_0.pdf}
    \end{center}
    { \hspace*{\fill} \\}
    

    \begin{center}
    \adjustimage{max size={0.9\linewidth}{0.9\paperheight}}{Teel_Formation_pmag_files/Teel_Formation_pmag_155_0.pdf}
    \end{center}
    { \hspace*{\fill} \\}
    
    \paragraph{Mid-temperature directions}\label{mid-temperature-directions}

\texttt{\color{outcolor}Out[{\color{outcolor}78}]:}
    \begin{sidewaystable}
    {\tiny\begin{tabular}{lrrrrrrrrrrrrrrr}
\toprule
{} &  strat\_pos &  site\_lat &  site\_lon &     dec\_geo &    inc\_geo &    alpha95 &  n &           k &         r &        csd &  paleolatitude &    vgp\_lat &     vgp\_lon &  vgp\_lat\_rev &  vgp\_lon\_rev \\
\midrule
Z42\_mid &       16.0 &  47.09577 &  95.38577 &  332.936127 & -68.973663 &  22.704625 &  4 &   17.342624 &  3.827016 &  19.450359 &     -52.447527 & -12.194273 &  291.867415 &    12.194273 &   111.867415 \\
Z43\_mid &       17.0 &  47.09570 &  95.38638 &  214.749151 & -62.316153 &  27.256978 &  4 &   12.329873 &  3.756688 &  23.067776 &     -43.621783 & -65.542782 &    0.679322 &    65.542782 &   180.679322 \\
Z58\_mid &        NaN &  47.06277 &  95.42045 &  169.686782 & -61.093602 &   4.004771 &  6 &  280.878503 &  5.982199 &   4.833100 &     -42.161138 & -81.183719 &  155.401297 &    81.183719 &   335.401297 \\
\bottomrule
\end{tabular}}
\end{sidewaystable}
    


    \begin{center}
    \adjustimage{max size={0.9\linewidth}{0.9\paperheight}}{Teel_Formation_pmag_files/Teel_Formation_pmag_158_0.pdf}
    \end{center}
    { \hspace*{\fill} \\}
    
    Flow Z58 yielded a completely different mid-temperature result compared
to all other sites. The magnetite direction is completely different than
all other results. The mean direction is very imprecise (SE and
moderately-shallow down) but is closest in orientation to the Middle to
Late Carboniferous `A' component of Edel et al. (2014).

    \subsection{Paleomagnetic Poles for the Teel
Formation}\label{paleomagnetic-poles-for-the-teel-formation}

    We interpret the primary paleomagnetic direction for the Teel basalts to
be held by (titano)magnetite with a secondary remanence held by hematite
holding a distinct direction. However, demagnetization data from some
sites within the most oxidized flows show similarities betwee the
remanence directions of magnetite and hematite that correspond to the
hematite direction seen in other flows that have a distinct magnetite
direction. We suspect that these flows were overprinted in the
remagnetization event that led to the hematite overprint. As a result we
exclude this flows where magnetite corresponds to the distinct hematite
direction from the calculation of the mean magnetite pole.

    \paragraph{Primary magnetite pole - including fold
test}\label{primary-magnetite-pole---including-fold-test}




            \begin{Verbatim}[commandchars=\\\{\}]
{\color{outcolor}Out[{\color{outcolor}82}]:} \{'alpha95': 4.9431252055096131,
          'csd': 13.058928417331206,
          'dec': 236.6128693429381,
          'inc': -34.995940978983391,
          'k': 38.472902790170508,
          'n': 23,
          'r': 22.428168960372265\}
\end{Verbatim}
        

            \begin{Verbatim}[commandchars=\\\{\}]
{\color{outcolor}Out[{\color{outcolor}83}]:} \{'alpha95': 5.7116821039230583,
          'csd': 15.025346173655382,
          'dec': 186.62753154703333,
          'inc': -64.852074989296611,
          'k': 29.061703291478157,
          'n': 23,
          'r': 22.24299000029874\}
\end{Verbatim}
        

    \begin{center}
    \adjustimage{max size={0.9\linewidth}{0.9\paperheight}}{Teel_Formation_pmag_files/Teel_Formation_pmag_167_0.pdf}
    \end{center}
    { \hspace*{\fill} \\}
    
    Bootstrap fold test (Tauxe and Watson, 1994)


    \begin{Verbatim}[commandchars=\\\{\}]
doing  1000  iterations{\ldots}please be patient{\ldots}
    \end{Verbatim}

    \begin{center}
    \adjustimage{max size={0.9\linewidth}{0.9\paperheight}}{Teel_Formation_pmag_files/Teel_Formation_pmag_169_1.pdf}
    \end{center}
    { \hspace*{\fill} \\}
    
    \begin{center}
    \adjustimage{max size={0.9\linewidth}{0.9\paperheight}}{Teel_Formation_pmag_files/Teel_Formation_pmag_169_2.pdf}
    \end{center}
    { \hspace*{\fill} \\}
    
    \begin{Verbatim}[commandchars=\\\{\}]
tightest grouping of vectors obtained at (95\% confidence bounds):
52 - 97 percent unfolding
range of all bootstrap samples: 
-10  -  106 percent unfolding
    \end{Verbatim}

    \begin{center}
    \adjustimage{max size={0.9\linewidth}{0.9\paperheight}}{Teel_Formation_pmag_files/Teel_Formation_pmag_169_4.pdf}
    \end{center}
    { \hspace*{\fill} \\}
    
    Below the tilt-corrected magnetite VGPs are plotted on the globe and
shaded according to their relative stratigraphic positions.


    \begin{center}
    \adjustimage{max size={0.9\linewidth}{0.9\paperheight}}{Teel_Formation_pmag_files/Teel_Formation_pmag_171_0.pdf}
    \end{center}
    { \hspace*{\fill} \\}
    
    \paragraph{Secondary hematite pole - including fold
test}\label{secondary-hematite-pole---including-fold-test}




            \begin{Verbatim}[commandchars=\\\{\}]
{\color{outcolor}Out[{\color{outcolor}89}]:} \{'alpha95': 9.2608689622177227,
          'csd': 20.97901071644646,
          'dec': 184.58560616364625,
          'inc': 3.8459585385145614,
          'k': 14.907335584818336,
          'n': 18,
          'r': 16.85962183494998\}
\end{Verbatim}
        

            \begin{Verbatim}[commandchars=\\\{\}]
{\color{outcolor}Out[{\color{outcolor}90}]:} \{'alpha95': 9.4955663885263011,
          'csd': 21.474669531178204,
          'dec': 182.89089648457883,
          'inc': 6.1094491428847837,
          'k': 14.227122114351561,
          'n': 18,
          'r': 16.805099171613119\}
\end{Verbatim}
        

    \begin{center}
    \adjustimage{max size={0.9\linewidth}{0.9\paperheight}}{Teel_Formation_pmag_files/Teel_Formation_pmag_177_0.pdf}
    \end{center}
    { \hspace*{\fill} \\}
    
    Bootstrap fold test (Tauxe and Watson, 1994)


    \begin{Verbatim}[commandchars=\\\{\}]
doing  1000  iterations{\ldots}please be patient{\ldots}
    \end{Verbatim}

    \begin{center}
    \adjustimage{max size={0.9\linewidth}{0.9\paperheight}}{Teel_Formation_pmag_files/Teel_Formation_pmag_179_1.pdf}
    \end{center}
    { \hspace*{\fill} \\}
    
    \begin{center}
    \adjustimage{max size={0.9\linewidth}{0.9\paperheight}}{Teel_Formation_pmag_files/Teel_Formation_pmag_179_2.pdf}
    \end{center}
    { \hspace*{\fill} \\}
    
    \begin{Verbatim}[commandchars=\\\{\}]
tightest grouping of vectors obtained at (95\% confidence bounds):
-20 - 119 percent unfolding
range of all bootstrap samples: 
-20  -  119 percent unfolding
    \end{Verbatim}

    \begin{center}
    \adjustimage{max size={0.9\linewidth}{0.9\paperheight}}{Teel_Formation_pmag_files/Teel_Formation_pmag_179_4.pdf}
    \end{center}
    { \hspace*{\fill} \\}
    
    Below the geographic and tilt-corrected hematite VGPs are plotted on the
globe and shaded according to their relative stratigraphic positions.
Note the similar positions between the two coordinate system means.


    \begin{center}
    \adjustimage{max size={0.9\linewidth}{0.9\paperheight}}{Teel_Formation_pmag_files/Teel_Formation_pmag_181_0.pdf}
    \end{center}
    { \hspace*{\fill} \\}
    

    \begin{center}
    \adjustimage{max size={0.9\linewidth}{0.9\paperheight}}{Teel_Formation_pmag_files/Teel_Formation_pmag_182_0.pdf}
    \end{center}
    { \hspace*{\fill} \\}
    
    \paragraph{Present local field overprint - negative fold
test}\label{present-local-field-overprint---negative-fold-test}

\texttt{\color{outcolor}Out[{\color{outcolor}95}]:}
\begin{sidewaystable}    
    {\tiny\begin{tabular}{lrrrrrrrrrrrrrrr}
\toprule
{} &  strat\_pos &  site\_lat &  site\_lon &     dec\_geo &    inc\_geo &    alpha95 &   n &           k &         r &        csd &  paleolatitude &    vgp\_lat &     vgp\_lon &  vgp\_lat\_rev &  vgp\_lon\_rev \\
\midrule
Z30\_low\_geo &        4.0 &  47.10038 &  95.37550 &  346.473053 &  68.261419 &  11.458361 &   7 &   28.705783 &  6.790983 &  15.118208 &      51.429142 &  80.188638 &   36.526238 &   -80.188638 &   216.526238 \\
Z31\_low\_geo &        5.0 &  47.10049 &  95.37604 &    0.054372 &  64.923839 &   3.618073 &   8 &  235.361535 &  7.970259 &   5.279798 &      46.897847 &  89.793993 &  264.986185 &   -89.793993 &    84.986185 \\
Z32\_low\_geo &        6.0 &  47.10094 &  95.37684 &   25.452043 &  61.325382 &   8.812910 &   8 &   40.461392 &  7.826996 &  12.733993 &      42.434489 &  71.428192 &  190.579622 &   -71.428192 &    10.579622 \\
Z33\_low\_geo &        7.0 &  47.10107 &  95.37705 &    0.930272 &  64.380739 &   6.003504 &   8 &   86.089978 &  7.918690 &   8.729889 &      46.196993 &  88.893147 &  239.802233 &   -88.893147 &    59.802233 \\
Z34\_low\_geo &        8.0 &  47.10111 &  95.37712 &    0.648074 &  62.405014 &   3.865736 &  10 &  157.128808 &  9.942722 &   6.461854 &      43.729780 &  86.598151 &  267.460060 &   -86.598151 &    87.460060 \\
\bottomrule
\end{tabular}}
\end{sidewaystable}
    

\texttt{\color{outcolor}Out[{\color{outcolor}96}]:}
\begin{sidewaystable}    
    {\tiny\begin{tabular}{lrrrrrrrrrrrrrrr}
\toprule
{} &  strat\_pos &  site\_lat &  site\_lon &     dec\_tc &     inc\_tc &    alpha95 &   n &           k &         r &        csd &  paleolatitude &    vgp\_lat &     vgp\_lon &  vgp\_lat\_rev &  vgp\_lon\_rev \\
\midrule
Z30\_low &        4.0 &  47.10038 &  95.37550 &  62.140855 &  33.713657 &  11.463681 &   7 &   28.680029 &  6.790795 &  15.124995 &      18.450285 &  32.248029 &  192.800152 &   -32.248029 &    12.800152 \\
Z31\_low &        5.0 &  47.10049 &  95.37604 &  59.403835 &  27.876060 &   3.615270 &   8 &  235.725142 &  7.970304 &   5.275724 &      14.813793 &  31.483537 &  198.002746 &   -31.483537 &    18.002746 \\
Z32\_low &        6.0 &  47.10094 &  95.37684 &  25.452043 &  61.325382 &   8.812910 &   8 &   40.461392 &  7.826996 &  12.733993 &      42.434489 &  71.428192 &  190.579622 &   -71.428192 &    10.579622 \\
Z33\_low &        7.0 &  47.10107 &  95.37705 &  54.800714 &  28.344304 &   6.009620 &   8 &   85.916769 &  7.918526 &   8.738684 &      15.094685 &  34.722681 &  201.658195 &   -34.722681 &    21.658195 \\
Z34\_low &        8.0 &  47.10111 &  95.37712 &  52.688703 &  27.653908 &   3.866823 &  10 &  157.041051 &  9.942690 &   6.463659 &      14.681124 &  35.788473 &  203.849471 &   -35.788473 &    23.849471 \\
\bottomrule
\end{tabular}}
\end{sidewaystable}
    

    A number of poles are excluded because of inconsistencies between
samples within site which resulted in large a95 values for these sites:
Z45, Z51, and Z52.



    \begin{center}
    \adjustimage{max size={0.9\linewidth}{0.9\paperheight}}{Teel_Formation_pmag_files/Teel_Formation_pmag_188_0.pdf}
    \end{center}
    { \hspace*{\fill} \\}
    
    \subparagraph{Bootstrap fold test (Tauxe and Watson,
1994)}\label{bootstrap-fold-test-tauxe-and-watson-1994}


    \begin{Verbatim}[commandchars=\\\{\}]
doing  1000  iterations{\ldots}please be patient{\ldots}
    \end{Verbatim}

    \begin{center}
    \adjustimage{max size={0.9\linewidth}{0.9\paperheight}}{Teel_Formation_pmag_files/Teel_Formation_pmag_190_1.pdf}
    \end{center}
    { \hspace*{\fill} \\}
    
    \begin{center}
    \adjustimage{max size={0.9\linewidth}{0.9\paperheight}}{Teel_Formation_pmag_files/Teel_Formation_pmag_190_2.pdf}
    \end{center}
    { \hspace*{\fill} \\}
    
    \begin{Verbatim}[commandchars=\\\{\}]
tightest grouping of vectors obtained at (95\% confidence bounds):
-19 - 9 percent unfolding
range of all bootstrap samples: 
-20  -  18 percent unfolding
    \end{Verbatim}

    \begin{center}
    \adjustimage{max size={0.9\linewidth}{0.9\paperheight}}{Teel_Formation_pmag_files/Teel_Formation_pmag_190_4.pdf}
    \end{center}
    { \hspace*{\fill} \\}
    
    \paragraph{Teel poles summary}\label{teel-poles-summary}

\texttt{\color{outcolor}Out[{\color{outcolor}100}]:}
    
    {\tiny\begin{tabular}{lrrrrrrrr}
\toprule
{} &   Pole\_Lat &  Pole\_Long &      A\_95 &          K &        CSD &   N &          r &   Paleolat \\
\midrule
Teel\_magnetite\_tc & -36.495314 &  16.038788 &  5.236274 &  34.392364 &  13.811918 &  23 &  22.360323 & -19.292649 \\
Teel\_hematite\_tc  & -39.717588 &  91.918678 &  7.536314 &  22.013320 &  17.264033 &  18 &  17.227740 &   3.063432 \\
Teel\_hematite\_geo & -40.795648 &  89.426839 &  5.608638 &  38.960325 &  12.976983 &  18 &  17.563659 &   1.925148 \\
\bottomrule
\end{tabular}}

    

    \section{Pole compilation for Siberia, North China, and Mongolian
terranes}\label{pole-compilation-for-siberia-north-china-and-mongolian-terranes}

    \subsection{Import existing paleomagnetic
data}\label{import-existing-paleomagnetic-data}

    \subsubsection{Siberia}\label{siberia}

    Note that a rotation needs to be applied for relative rotation between
Aldan and Anabar blocks before the Devonian Period due to Devonian
rifting in the Viljuy Basin near the centre of the craton. Here are two
rotations used in the literature: Euler Pole (Lat, Long, rotation)

\begin{itemize}
\tightlist
\item
  (60ºN, 120ºE, 13º) from Smethurst et al. (1998)
\item
  (60ºN, 115ºE, 25º) everything pre-Devonian (Evans, 2009)
\item
  (60ºN, 120ºE, 16º) Cambrian to Early Silurian correction (Cocks and
  Torsvik, 2007)
\item
  (62ºN, 117ºE, 20º) pre-Devonian (Pavlov et al. (2008) also used in
  Powerman et al. (2013))
\end{itemize}

Most Siberia poles are imported from Cocks and Torsvik (2007) in which
the data are rotated from the ``southern'' Siberia (Aldan) into the
northern Siberia (Anabar) reference frame according to Smethurst et al.
(1998), which the authors argue brings N and S pre-Devonian poles into
the best agreement.

    Torsvik et al. (2012) updated their Siberia apparent polar wander path
by adding data from Shatsillo et al. (2007) that superceded results from
the coeval Lena River sediments (Rodianov et al., 1982; Torsvik et al.,
1995). However, there are more results from Siberia that must have been
discarded by Cocks and Torsvik (2007) and subsequently by other authors.
We discuss these poles later on.



    Below we plot the paleomagnetic data compiled by Cocks and Torsvik
(2007) for Siberia, shaded according to age.


    \begin{center}
    \adjustimage{max size={0.9\linewidth}{0.9\paperheight}}{Teel_Formation_pmag_files/Teel_Formation_pmag_201_0.pdf}
    \end{center}
    { \hspace*{\fill} \\}
    
    In the following analyses, we update this pole list to include
additional poles from the area in order to construct a paleolatitdue
plot of Siberia through the Phanerozoic Eon.

    We use the Haversine formula to calculate the distance between the VGPs
and a reference point on a given plate. This is then used to calculate
the paleolatite of the reference point.


    We first load poles for stable Europe from 250 Ma to the present day
from Torsvik et al. (2012).

\texttt{\color{outcolor}Out[{\color{outcolor}105}]:}
    
    {\tiny\begin{tabular}{lrrrrrrrrr}
\toprule
{} &  high\_age &  low\_age &  median\_age &   A95 &  PLat &   PLon &  Paleolat &  PLat\_N &  PLon\_N \\
\midrule
0 &       1.0 &      0.0 &         0.5 &   3.6 & -80.6 &  267.5 &    60.638 &    80.6 &    87.5 \\
1 &       1.0 &      0.0 &         0.5 &   4.4 & -86.4 &  296.1 &    55.206 &    86.4 &   116.1 \\
2 &      10.0 &      6.0 &         8.0 &  12.9 & -84.3 &  357.7 &    52.907 &    84.3 &   177.7 \\
3 &      11.0 &      8.0 &         9.5 &   1.8 & -78.9 &  328.3 &    58.733 &    78.9 &   148.3 \\
4 &      11.0 &      9.0 &        10.0 &   3.5 & -77.4 &  314.2 &    61.900 &    77.4 &   134.2 \\
\bottomrule
\end{tabular}}

    

    Poles from Siberia are then loaded (545 to 250 Ma). Most of the poles
were taken from Cocks and Torsvik (2007) and Torsvik et al. (2012), but
we also added additional data gathered from the Global Paleomagnetic
Database.

\texttt{\color{outcolor}Out[{\color{outcolor}106}]:}
    \begin{sidewaystable}
    {\tiny\begin{tabular}{lrrrrrrrlrrr}
\toprule
{} &  plate\_ID &  high\_age &  low\_age &  median\_age &   A95 &  PLat &   PLon &                                          Reference &  Paleolat &  PLat\_N &  PLon\_N \\
\midrule
0  &       401 &       245 &      243 &       244.0 &  10.0 & -59.0 &  330.0 &  GPDB2832, Gurevitch et al. (1995) from Cocks a... &    63.178 &    59.0 &   150.0 \\
1  &       401 &       258 &      238 &       248.0 &   7.8 & -59.3 &  325.8 &  Walderhaug et al. (2005) from Cocks and Torsvi... &    65.344 &    59.3 &   145.8 \\
2  &       401 &       253 &      248 &       251.0 &   3.3 & -56.2 &  326.0 &  Gurevitch et al. (2004) from Cocks and Torsvik... &    65.004 &    56.2 &   146.0 \\
3  &       401 &       253 &      248 &       251.0 &   9.7 & -52.8 &  334.4 &  GPDB3486, Kravchinsky et al. (2002) from Cocks... &    59.477 &    52.8 &   154.4 \\
4  &       401 &       253 &      248 &       251.0 &   2.2 & -56.6 &  307.9 &  Pavlov and Gallet (1996) from Cocks and Torsvi... &    74.988 &    56.6 &   127.9 \\
5  &       401 &       285 &      265 &       275.0 &   8.6 & -50.5 &  301.4 &  Pisarevsky et al. (2006) from Cocks and Torsvi... &    78.724 &    50.5 &   121.4 \\
6  &       401 &       363 &      290 &       326.5 &   1.3 & -21.0 &  350.0 &           GPDB1991, Davydov and Kravchinsky (1973) &    30.794 &    21.0 &   170.0 \\
7  &       401 &       352 &      332 &       342.0 &  17.0 & -16.0 &  295.0 &                         GPDB1986, Kamysheva (1971) &    53.142 &    16.0 &   115.0 \\
8  &       401 &       348 &      340 &       344.0 &   5.8 & -25.2 &  320.0 &                    GPDB3041, Zhitkov et al. (1994) &    51.715 &    25.2 &   140.0 \\
9  &       401 &       377 &      350 &       360.0 &   8.9 & -11.1 &  329.7 &  GPDB3486, Kravchinsky et al. (2002) from Cocks... &    34.892 &    11.1 &   149.7 \\
10 &       401 &       377 &      350 &       363.5 &  10.1 & -27.8 &  339.9 &                GPDB3486, Kravchinsky et al. (2002) &    42.021 &    27.8 &   159.9 \\
11 &       401 &       377 &      350 &       363.5 &  11.9 & -22.8 &  339.4 &                GPDB3486, Kravchinsky et al. (2002) &    38.641 &    22.8 &   159.4 \\
12 &       401 &       391 &      363 &       377.0 &   5.0 & -13.0 &  302.0 &                         GPDB1997, Kamysheva (1975) &    48.523 &    13.0 &   122.0 \\
13 &       410 &       430 &      397 &       413.5 &   3.2 &   8.2 &  292.0 &                             Powerman et al. (2013) &    29.655 &    -8.2 &   112.0 \\
14 &       401 &       443 &      423 &       433.0 &   4.6 &  19.0 &  308.0 &  Shatsillo et al. (2007) from Cocks and Torsvik... &    16.126 &   -19.0 &   128.0 \\
15 &       410 &       444 &      423 &       433.5 &   4.4 &  18.4 &  302.7 &                             Powerman et al. (2013) &    17.919 &   -18.4 &   122.7 \\
16 &       401 &       454 &      424 &       439.0 &   8.0 &  14.0 &  304.0 &  Smethurst et al. (1998) from Cocks and Torsvik... &    21.927 &   -14.0 &   124.0 \\
17 &       401 &       460 &      440 &       450.0 &  17.3 &  19.4 &  315.3 &  Smethurst et al. (1998) from Cocks and Torsvik... &    13.660 &   -19.4 &   135.3 \\
18 &       410 &       461 &      443 &       452.0 &   5.1 &  27.5 &  332.0 &                             Powerman et al. (2013) &     0.109 &   -27.5 &   152.0 \\
19 &       401 &       464 &      458 &       461.0 &   2.5 &  22.8 &  334.2 &  GPDB3473, Iosifidi et al. (1999) from Cocks an... &     3.312 &   -22.8 &   154.2 \\
20 &       401 &       464 &      458 &       461.0 &   5.1 &  22.1 &  324.9 &  GPDB3473, Iosifidi et al. (1999) from Cocks an... &     7.787 &   -22.1 &   144.9 \\
21 &       401 &       473 &      453 &       463.0 &   4.0 &  23.0 &  338.0 &  Smethurst et al. (1998) from Cocks and Torsvik... &     1.413 &   -23.0 &   158.0 \\
22 &       401 &       470 &      464 &       467.0 &   3.2 &  30.9 &  332.7 &  GPDB3448 Gallet and Pavlov (1998) from Cocks a... &    -3.183 &   -30.9 &   152.7 \\
23 &       401 &       478 &      458 &       468.0 &   3.1 &  24.4 &  346.0 &  Smethurst et al. (1998) from Cocks and Torsvik... &    -3.645 &   -24.4 &   166.0 \\
24 &       401 &       479 &      459 &       469.0 &   4.0 &  30.0 &  337.0 &  Smethurst et al. (1998) from Cocks and Torsvik... &    -4.193 &   -30.0 &   157.0 \\
25 &       401 &       480 &      460 &       470.0 &   9.0 &  17.9 &  342.8 &  Smethurst et al. (1998) from Cocks and Torsvik... &     3.434 &   -17.9 &   162.8 \\
26 &       401 &       488 &      468 &       478.0 &   2.2 &  33.9 &  331.7 &  Smethurst et al. (1998) from Cocks and Torsvik... &    -5.441 &   -33.9 &   151.7 \\
27 &       401 &       495 &      470 &       482.5 &   5.8 &  36.2 &  338.8 &  GPDB3474, Surkis et al. (1999) from Cocks and ... &   -10.298 &   -36.2 &   158.8 \\
28 &       401 &       493 &      473 &       483.0 &   9.0 &  40.0 &  318.0 &  Smethurst et al. (1998) from Cocks and Torsvik... &    -6.498 &   -40.0 &   138.0 \\
29 &       401 &       495 &      485 &       490.0 &   4.9 &  35.2 &  307.2 &  GPDB3448, Gallet and Pavlov (1998) from Cocks ... &     0.651 &   -35.2 &   127.2 \\
30 &       401 &       495 &      485 &       490.0 &   2.3 &  41.9 &  315.8 &  GPDB3192, Pavlov and Gallet (1998) from Cocks ... &    -7.711 &   -41.9 &   135.8 \\
31 &       401 &       510 &      490 &       500.0 &   6.0 &  37.0 &  318.0 &  Smethurst et al. (1998) from Cocks and Torsvik... &    -3.690 &   -37.0 &   138.0 \\
32 &       401 &       505 &      495 &       500.0 &   3.0 &  36.1 &  310.7 &  GPDB3192, Pavlov and Gallet (1998) from Cocks ... &    -0.974 &   -36.1 &   130.7 \\
33 &       401 &       518 &      495 &       506.5 &   4.5 &  32.6 &  333.8 &  GPDB3472, Rodionov et al. (1998) from Cocks an... &    -5.123 &   -32.6 &   153.8 \\
34 &       401 &       514 &      500 &       507.0 &   2.6 &  43.7 &  320.5 &  GPDB3537, Gallet et al. (2003) from Cocks and ... &   -10.622 &   -43.7 &   140.5 \\
35 &       401 &       518 &      505 &       511.5 &   4.6 &  36.4 &  319.6 &  GPDB3164, Pisarevsky et al. (1997) from Cocks ... &    -3.591 &   -36.4 &   139.6 \\
36 &       401 &       520 &      510 &       515.0 &   5.1 &  53.3 &  315.0 &  GPDB3537, Gallet et al. (2003) from Cocks and ... &   -18.264 &   -53.3 &   135.0 \\
37 &       401 &       535 &      518 &       526.5 &   6.8 &  44.8 &  338.7 &  GPDB3164, Pisarevsky et al. (1997) from Cocks ... &   -17.577 &   -44.8 &   158.7 \\
38 &       401 &       538 &      518 &       528.0 &   7.0 &  32.0 &  317.0 &  Smethurst et al. (1998) from Cocks and Torsvik... &     1.285 &   -32.0 &   137.0 \\
39 &       401 &       545 &      525 &       535.0 &   6.2 &  16.6 &  244.5 &  GPDB1627, Kirschvink and Rozanov (1984) from C... &    13.732 &   -16.6 &    64.5 \\
40 &       401 &       545 &      535 &       540.0 &  12.8 &  37.6 &  345.0 &  GPDB3164, Pisarevsky et al. (1997) from Cocks ... &   -14.154 &   -37.6 &   165.0 \\
\bottomrule
\end{tabular}}
\end{sidewaystable}
    


    \begin{center}
    \adjustimage{max size={0.9\linewidth}{0.9\paperheight}}{Teel_Formation_pmag_files/Teel_Formation_pmag_209_0.pdf}
    \end{center}
    { \hspace*{\fill} \\}
    
    \subsubsection{North China}\label{north-china}

    We first load the paleomagnetic data for North China compiled in Cocks
and Torsvik (2013).


    We also add some additional poles from North China that were not
included in Cocks and Torsvik (2013) including a compilation of data
from Huang et al. (1999) and additional poles from Embleton et al.
(1996), Huang et al. (2001), and Doh and Piper (1994).

\texttt{\color{outcolor}Out[{\color{outcolor}109}]:}
    
    {\tiny\begin{tabular}{lrrrrrrlr}
\toprule
{} &  high\_age &  low\_age &  median\_age &   A95 &  PLat &   PLon &                        References &  Paleolat \\
\midrule
0  &        88 &       68 &        78.0 &   5.8 &  79.7 &  170.8 &     see Van der Voo et al. (2015) &    46.149 \\
1  &        98 &       78 &        88.0 &   5.3 &  81.1 &  294.5 &     see Van der Voo et al. (2015) &    33.136 \\
2  &       110 &       90 &       100.0 &   4.7 &  70.6 &  156.7 &     see Van der Voo et al. (2015) &    52.870 \\
3  &       123 &      103 &       113.0 &   5.2 &  76.8 &  192.1 &     see Van der Voo et al. (2015) &    42.209 \\
4  &       126 &      106 &       116.0 &   4.6 &  83.6 &  172.3 &     see Van der Voo et al. (2015) &    44.602 \\
5  &       128 &      108 &       118.0 &   5.9 &  75.2 &  147.7 &     see Van der Voo et al. (2015) &    52.663 \\
6  &       128 &      108 &       118.0 &  12.2 &  69.0 &  200.7 &     see Van der Voo et al. (2015) &    38.082 \\
7  &       131 &      111 &       121.0 &   3.0 &  84.8 &   30.2 &     see Van der Voo et al. (2015) &    42.801 \\
8  &       131 &      111 &       121.0 &   6.5 &  82.7 &  159.6 &     see Van der Voo et al. (2015) &    46.356 \\
9  &       138 &      118 &       128.0 &   6.8 &  73.9 &  153.3 &     see Van der Voo et al. (2015) &    52.221 \\
10 &       165 &      145 &       155.0 &   6.8 &  59.9 &  240.3 &     see Van der Voo et al. (2015) &    19.446 \\
11 &       165 &      145 &       155.0 &   7.9 &  72.8 &  180.3 &     see Van der Voo et al. (2015) &    45.207 \\
12 &       165 &      145 &       155.0 &   5.9 &  74.9 &  175.4 &     see Van der Voo et al. (2015) &    46.347 \\
13 &       165 &      145 &       155.0 &   6.9 &  78.0 &  166.8 &     see Van der Voo et al. (2015) &    47.463 \\
14 &       165 &      145 &       155.0 &   5.7 &  70.7 &  162.2 &     see Van der Voo et al. (2015) &    51.138 \\
15 &       170 &      150 &       160.0 &   6.6 &  68.7 &  159.1 &     see Van der Voo et al. (2015) &    52.805 \\
16 &       179 &      159 &       169.0 &   8.3 &  71.8 &  148.2 &     see Van der Voo et al. (2015) &    54.640 \\
17 &       179 &      159 &       169.0 &   6.4 &  78.9 &  218.6 &     see Van der Voo et al. (2015) &    37.490 \\
18 &       179 &      159 &       169.0 &   4.5 &  78.1 &  175.2 &     see Van der Voo et al. (2015) &    45.774 \\
19 &       198 &      178 &       188.0 &   6.8 &  84.0 &  112.9 &     see Van der Voo et al. (2015) &    47.985 \\
20 &       198 &      178 &       188.0 &   5.5 &  84.4 &   98.6 &     see Van der Voo et al. (2015) &    47.499 \\
21 &       235 &      208 &       221.5 &   5.8 &  70.9 &   49.8 &     see Van der Voo et al. (2015) &    49.184 \\
22 &       235 &      208 &       221.5 &   3.8 &  65.6 &   36.1 &     see Van der Voo et al. (2015) &    44.398 \\
23 &       241 &      235 &       238.0 &   4.0 &  64.7 &   22.1 &     see Van der Voo et al. (2015) &    38.471 \\
24 &       241 &      235 &       238.0 &   3.1 &  52.8 &   36.6 &     see Van der Voo et al. (2015) &    41.977 \\
25 &       241 &      235 &       238.0 &   3.9 &  52.2 &   46.9 &     see Van der Voo et al. (2015) &    47.889 \\
26 &       245 &      241 &       243.0 &   6.6 &  55.1 &  359.9 &     see Van der Voo et al. (2015) &    24.183 \\
27 &       251 &      245 &       248.0 &   3.8 &  67.7 &    7.5 &     see Van der Voo et al. (2015) &    34.254 \\
28 &       251 &      245 &       248.0 &   4.8 &  55.7 &   19.8 &     see Van der Voo et al. (2015) &    33.960 \\
29 &       261 &      251 &       256.0 &  18.9 &  64.7 &   27.4 &     see Van der Voo et al. (2015) &    40.643 \\
30 &       261 &      251 &       256.0 &   6.9 &  48.8 &    5.6 &     see Van der Voo et al. (2015) &    22.956 \\
31 &       261 &      251 &       256.0 &   4.0 &  52.9 &    7.4 &     see Van der Voo et al. (2015) &    26.331 \\
32 &       261 &      251 &       256.0 &   6.9 &  42.0 &    9.5 &     see Van der Voo et al. (2015) &    20.891 \\
33 &       270 &      256 &       263.0 &  12.2 &  20.9 &    1.4 &  GPDB3086, Embleton et al. (1996) &     1.649 \\
34 &       303 &      295 &       299.0 &  16.7 &  33.3 &   10.2 &     GPDB3468, Huang et al. (2001) &    15.804 \\
35 &       320 &      307 &       313.5 &   7.2 &  44.6 &  335.7 &    GPDB2734, Doh and Piper (1994) &     6.139 \\
36 &       350 &      321 &       335.5 &   6.2 &  10.5 &   14.0 &     GPDB3468, Huang et al. (2001) &     3.340 \\
37 &       416 &      360 &       388.0 &   8.8 &  34.2 &  228.7 &           see Huang et al. (1999) &     4.105 \\
38 &       444 &      416 &       430.0 &   8.2 &  26.2 &  228.4 &           see Huang et al. (1999) &    -1.828 \\
39 &       488 &      444 &       466.0 &  12.3 &  28.8 &  310.9 &           see Huang et al. (1999) &   -16.372 \\
40 &       472 &      461 &       466.5 &   7.0 &  31.5 &  327.7 &               Huang et al. (1999) &    -8.331 \\
41 &       472 &      461 &       466.5 &  10.6 &  43.2 &  332.5 &           see Huang et al. (1999) &     3.732 \\
42 &       472 &      461 &       466.5 &   9.2 &  27.9 &  310.4 &           see Huang et al. (1999) &   -17.360 \\
43 &       488 &      472 &       480.0 &   8.5 &  37.4 &  324.3 &               Huang et al. (1999) &    -4.325 \\
44 &       501 &      488 &       494.5 &   5.4 &  31.7 &  329.6 &               Huang et al. (1999) &    -7.381 \\
45 &       513 &      501 &       507.0 &   5.5 &  37.0 &  326.7 &               Huang et al. (1999) &    -3.836 \\
46 &       542 &      513 &       527.5 &   6.5 &  18.5 &  341.9 &               Huang et al. (1999) &   -12.286 \\
47 &       542 &      513 &       527.5 &  12.4 &  21.2 &  335.2 &           see Huang et al. (1999) &   -13.744 \\
48 &       542 &      513 &       527.5 &   9.9 &  15.0 &  298.6 &           see Huang et al. (1999) &   -32.316 \\
49 &       542 &      513 &       527.5 &   8.9 &  26.8 &  334.5 &           see Huang et al. (1999) &    -9.395 \\
\bottomrule
\end{tabular}}

    \begin{center}
    \adjustimage{max size={0.9\linewidth}{0.9\paperheight}}{Teel_Formation_pmag_files/Teel_Formation_pmag_215_1.pdf}
    \end{center}
    { \hspace*{\fill} \\}
    
    \subsubsection{Mongolia pole
compilation}\label{mongolia-pole-compilation}

    Edel et al. (2014) published paleomagnetic data from 12 sites in the
Trans-Altai and South Gobi zones. This work identified magnetic
overprint directions for which a variety of arguments are made as to
their temporal relationship. The progression of directions as
interpretted by the authors leads to an appreciable change in magnetic
declination from overprints intrepretted to be Middle--Late
Carboniferous in age to magnetizations that are interpreted to be
Permian in age. The authors propose that this declination change is the
result of vertical axis rotation associated with oroclinal bending of a
Mongolian ribbon continent (an illustraion of this model shown below
form that paper).


    \begin{center}
    \adjustimage{max size={0.9\linewidth}{0.9\paperheight}}{Teel_Formation_pmag_files/Teel_Formation_pmag_218_0.png}
    \end{center}
    { \hspace*{\fill} \\}
    
    We import paleomagnetic data that have been compiled for Mongolia and
some of the surrounding terranes between Siberia and North China and
assigned them to established terranes pertinent to this study.

\texttt{\color{outcolor}Out[{\color{outcolor}112}]:}
    
    {\tiny\begin{tabular}{llrrrrrrlr}
\toprule
{} &            terrane &  high\_age &  low\_age &  median\_age &   A95 &  PLat &   PLon &                               Reference &  Paleolat \\
\midrule
0  &    Zavkhan\_Baidrag &       105 &       92 &        98.5 &   3.9 &  81.1 &  165.7 &            van Hinsbergen et al. (2008) &    49.393 \\
1  &    Zavkhan\_Baidrag &       146 &       65 &       105.5 &  21.4 &  86.9 &  252.8 &                 GPDB2443, Pruner (1992) &    44.225 \\
2  &    Zavkhan\_Baidrag &       124 &       92 &       108.0 &   2.5 &  80.8 &  158.4 &            van Hinsbergen et al. (2008) &    50.579 \\
3  &    Zavkhan\_Baidrag &       119 &      115 &       117.0 &   4.9 &  75.6 &  132.3 &            van Hinsbergen et al. (2008) &    57.658 \\
4  &     greater\_Amuria &       130 &      110 &       120.0 &   5.2 &  70.8 &  322.4 &                     Cogne et al. (2005) &    32.628 \\
5  &     greater\_Amuria &       145 &       97 &       121.0 &   4.2 &  58.3 &   51.0 &                     Halim et al. (1998) &    61.511 \\
6  &    Zavkhan\_Baidrag &       125 &      118 &       121.5 &   4.3 &  82.0 &  172.3 &            van Hinsbergen et al. (2008) &    48.319 \\
7  &     greater\_Amuria &       133 &      125 &       129.0 &   7.4 &  86.8 &   61.8 &                     Cogne et al. (2005) &    49.735 \\
8  &     greater\_Amuria &       161 &      145 &       153.0 &   3.1 &  58.9 &  327.3 &                     Cogne et al. (2005) &    24.229 \\
9  &     greater\_Amuria &       176 &      145 &       160.5 &   4.2 &  59.6 &  279.0 &               Kravchinsky et al. (2002) &    16.741 \\
10 &     greater\_Amuria &       245 &      208 &       226.5 &  16.8 &  32.0 &   32.7 &                 GPDB2443, Pruner (1992) &    40.779 \\
11 &  southern\_terranes &       260 &      228 &       244.0 &   8.0 &  50.0 &  201.0 &                      Edel et al. (2014) &    26.317 \\
12 &    Zavkhan\_Baidrag &       260 &      240 &       250.0 &  11.0 &  55.0 &  131.3 &                        Kovalenko (2010) &    66.385 \\
13 &     greater\_Amuria &       271 &      260 &       265.5 &  14.4 &  63.1 &  151.0 &               Kravchinsky et al. (2002) &    55.811 \\
14 &     greater\_Amuria &       290 &      256 &       273.0 &  11.6 &  44.8 &  335.1 &                 GPDB2443, Pruner (1992) &    15.820 \\
15 &  southern\_terranes &       310 &      245 &       277.5 &   8.0 &  46.0 &  273.0 &                      Edel et al. (2014) &     3.123 \\
16 &  southern\_terranes &       300 &      280 &       290.0 &   7.8 &  71.0 &  188.0 &                        Kovalenko (2010) &    43.039 \\
17 &    Zavkhan\_Baidrag &       323 &      290 &       306.5 &  10.4 &  37.5 &  320.1 &                 GPDB2443, Pruner (1992) &     3.567 \\
18 &  southern\_terranes &       363 &      323 &       343.0 &  13.0 &  -1.0 &  354.1 &  GPDB3045, Pechersky and Didenko (1995) &    -8.390 \\
19 &  southern\_terranes &       391 &      363 &       377.0 &   3.4 &  39.9 &  244.3 &  GPDB3045, Pechersky and Didenko (1995) &     1.297 \\
20 &  southern\_terranes &       391 &      363 &       377.0 &   4.6 &  51.7 &  282.7 &  GPDB3045, Pechersky and Didenko (1995) &     8.999 \\
21 &  southern\_terranes &       391 &      363 &       377.0 &   3.5 &  38.0 &  244.0 &         GPDB2594, Grishin et al. (1991) &    -0.399 \\
22 &  southern\_terranes &       391 &      363 &       377.0 &   5.1 &  52.0 &  280.0 &         GPDB2594, Grishin et al. (1991) &     9.179 \\
23 &  southern\_terranes &       363 &      245 &       304.0 &  11.9 &  50.0 &  354.0 &         GPDB2594, Grishin et al. (1991) &    28.348 \\
24 &  southern\_terranes &       340 &      299 &       319.5 &  13.0 &   5.0 &  341.0 &                      Edel et al. (2014) &   -12.479 \\
25 &    Zavkhan\_Baidrag &       440 &      200 &       320.0 &   5.6 &  40.8 &  269.4 &                              This study &    -1.939 \\
26 &  southern\_terranes &       360 &      320 &       340.0 &   4.9 &  10.0 &  330.0 &                        Kovalenko (2010) &   -15.126 \\
27 &          Lake\_Zone &       423 &      397 &       410.0 &   5.8 & -13.3 &   63.7 &                 Bachtadse et al. (2000) &    23.280 \\
28 &          Lake\_Zone &       428 &      397 &       412.5 &   6.1 &  26.3 &  144.0 &                 Bachtadse et al. (2000) &    46.717 \\
29 &          Lake\_Zone &       428 &      416 &       422.0 &   3.6 & -17.5 &  100.1 &                 Bachtadse et al. (2000) &    25.260 \\
30 &    Zavkhan\_Baidrag &       450 &      410 &       430.0 &  12.3 &   7.0 &  106.7 &                      Kravchinsky (2010) &    48.745 \\
31 &    Zavkhan\_Baidrag &       449 &      443 &       446.0 &   5.2 &  36.5 &  196.0 &                              This study &    19.566 \\
32 &    Zavkhan\_Baidrag &       542 &      360 &       451.0 &   4.6 &   3.5 &  114.9 &                     Evans et al. (1996) &    43.245 \\
33 &    Zavkhan\_Baidrag &       545 &      518 &       531.5 &  13.5 &  21.4 &  347.1 &                      Kravchinsky (2001) &     3.927 \\
34 &    Zavkhan\_Baidrag &       545 &      518 &       531.5 &  10.1 &  14.7 &  228.6 &  GPDB3045, Pechersky and Didenko (1995) &   -15.368 \\
35 &    Zavkhan\_Baidrag &       545 &      518 &       531.5 &   4.4 &  24.1 &  283.3 &  GPDB3045, Pechersky and Didenko (1995) &   -18.442 \\
36 &    Zavkhan\_Baidrag &       650 &      518 &       584.0 &   4.7 &  17.6 &  309.7 &  GPDB3045, Pechersky and Didenko (1995) &   -18.324 \\
37 &    Zavkhan\_Baidrag &       650 &      518 &       584.0 &   5.4 &  22.6 &  285.6 &  GPDB3045, Pechersky and Didenko (1995) &   -19.692 \\
\bottomrule
\end{tabular}}

    


    Calculate paleolatitudes for the Mongolia poles, considering that many
may have experienced horizonatal-axis rotations during the formation of
the COAB and possibly earlier.



    \begin{center}
    \adjustimage{max size={0.9\linewidth}{0.9\paperheight}}{Teel_Formation_pmag_files/Teel_Formation_pmag_224_0.pdf}
    \end{center}
    { \hspace*{\fill} \\}
    
    \subsection{Paleolatitude diagram}\label{paleolatitude-diagram}

    We plot the data from Siberia, North China, and Mongolia (from Zavkhan,
Baidrag, Lake Zone, and other southern terranes) on a paleolatitude
versus time plot. The paleolatitudes given for each block are for
specific reference points on each terrane. For Mongolia, the site of the
Teel Formation is used (95.38 ºN, 47.1 ºE). For Siberia, coordinates at
the southern tip of the craton (51.7 ºN, 103.5 ºE) are given seeing as
this would be the proposed conjugate margin for Mongolia. For North
China, a reference point on the northern margin (42 ºN, 109 ºE) is given
to represent the alternative conjugate margin that would have shared
similar paleolatitudes with Mongolia if they were attached.

    There are a handful of Mongolian poles that we exclude because of wide
age uncertainties or lack of statistical robustness. The Bachtadse et
al. (2000) component B pole is excluded because of the small number or
samples used to calculate the mean direction (25 samples; unblocking
temperatures of 270--420 ºC); it is very similar to two Levashova (2010)
directions which may be overprints (see below). The Evans et al. (1996)
Bayan-Gol pole was superseded by results from Kravchinsky (2001) and may
be a pre-folding overprint, given the increase in precision after tilt
correction. The Kravchinsky et al. (2010) pole, that they call a
remagnetization, has a very uncertain age and is not tilt-corrected,
therefore we see it as unreliable for a paleolatitude estimate. The
Kovalenko (2010) pole from the granite at Hanbogd is excluded because it
is only from one site and is in the Trans-Altai zone, which is severely
affected by early Triassic deformation along the Gobi-Tienshan fault
(Lehmann et al., 2010). The ``Mongolian sediments and volcanics,
Gurvan-Sayhan Range, post-folding'' pole from Grishin et al. (1991)
(also discussed in Pechersky and Didenko (1995)) is not included because
it is likely underaveraged (only from one site) and because of its large
age uncertainty; it is also a post-folding remanence.


    
    \begin{verbatim}
<matplotlib.figure.Figure at 0x11bb2bcd0>
    \end{verbatim}

    
    \begin{center}
    \adjustimage{max size={0.9\linewidth}{0.9\paperheight}}{Teel_Formation_pmag_files/Teel_Formation_pmag_228_1.pdf}
    \end{center}
    { \hspace*{\fill} \\}
    
    \section{Regional overprints in Precambrian
rocks}\label{regional-overprints-in-precambrian-rocks}

    In order to understand the regional paleomagnetic directions,
specifically possible overprints, we compare results from the Zavkhan
block to see if there are dominant overprints that affected all rocks.
These results are from Levashova et al. (2010), Kravchinsky et al.
(2001), Evans et al. (1996), this study, and preliminary data from the
Zavkhan volcanics.

\texttt{\color{outcolor}Out[{\color{outcolor}117}]:}
    
    {\tiny\begin{tabular}{llrrrrrrrrrl}
\toprule
{} &            ID &    N &  k\_geo &  Dec\_geo &  Inc\_geo &  a95\_geo &   k\_tc &  Dec\_tc &  Inc\_tc &  a95\_tc &                    comments \\
\midrule
0  &     Lev10-INT &   18 &   19.0 &    272.8 &    -66.2 &      8.2 &    3.0 &   158.7 &   -42.8 &    24.7 &                         NaN \\
1  &      Lev10-HT &   11 &    3.0 &    311.7 &    -11.0 &     30.1 &   19.0 &   321.9 &   -65.0 &    10.7 &                         NaN \\
2  &     Lev11-INT &   27 &   14.0 &    207.9 &    -30.6 &      7.8 &    6.0 &   210.4 &    -4.0 &    12.4 &                         NaN \\
3  &      Lev11-HT &   18 &   10.0 &    194.2 &     29.2 &     11.5 &   41.0 &   179.6 &    53.7 &     5.4 &                         NaN \\
4  &    Krav01-LOW &   10 &   59.7 &      4.1 &     70.9 &      6.3 &   96.8 &   181.2 &    85.6 &     4.9 &  data from both B-G and T-O \\
5  &    Krav01-INT &    9 &   31.3 &    209.2 &    -66.0 &      9.3 &  117.6 &   284.3 &   -79.7 &     4.8 &  data from both B-G and T-O \\
6  &     Krav01-HT &    6 &   14.9 &    118.5 &      5.3 &     17.9 &   13.3 &   118.3 &    -6.3 &    19.0 &               data from B-G \\
7  &      Teel\_mag &   23 &   29.1 &    186.6 &    -64.9 &      5.7 &   38.5 &   236.6 &   -35.0 &     4.9 &                         NaN \\
8  &      Teel\_hem &   18 &   14.9 &    184.6 &      3.8 &      9.3 &   14.2 &   182.9 &     6.1 &     9.5 &                         NaN \\
9  &      Evans\_HT &  193 &    NaN &      NaN &      NaN &      NaN &    5.8 &   331.9 &   -62.6 &     4.6 &                         NaN \\
10 &   Z09\_cgl\_INT &   20 &   32.3 &    200.7 &    -62.2 &      5.8 &   32.3 &   212.3 &    -5.8 &     5.8 &                         NaN \\
11 &  Z104\_cgl\_INT &   31 &  165.4 &    174.8 &    -61.7 &      2.0 &  165.2 &    61.9 &   -71.4 &     2.0 &                         NaN \\
\bottomrule
\end{tabular}}

    

    All directions are first plotted in geographic coordinates.


    \begin{center}
    \adjustimage{max size={0.9\linewidth}{0.9\paperheight}}{Teel_Formation_pmag_files/Teel_Formation_pmag_233_0.pdf}
    \end{center}
    { \hspace*{\fill} \\}
    
    Then the overprint data are plotted in tilt-corrected coordinates.


    \begin{center}
    \adjustimage{max size={0.9\linewidth}{0.9\paperheight}}{Teel_Formation_pmag_files/Teel_Formation_pmag_235_0.pdf}
    \end{center}
    { \hspace*{\fill} \\}
    
    Given geological and statistical paleomagnetic context (improvement
during tilt correction) we create a plot of what we consider to be the
overprint directions if they were acquired either before or after
folding.


    \begin{center}
    \adjustimage{max size={0.9\linewidth}{0.9\paperheight}}{Teel_Formation_pmag_files/Teel_Formation_pmag_237_0.pdf}
    \end{center}
    { \hspace*{\fill} \\}
    
    As stated in the main text, the Kravchinsky et al. (2001) intermediate
component from late Neoproterozoic to early Cambrian in geographic
coordinates is similar to the Teel magnetite component in geographic
coordinates. It is possible that the Ordovician magmatism recorded in
the Teel Formation basalts caused this overprint in older lithologies.

    \section{Data Repository References}\label{data-repository-references}

    Bachtadse, V., Pavlov, V. E., Kazansky, A. Y., and Tait, J. A., 2000,
Siluro-Devonian paleomagnetic results from the Tuva Terrane (southern
Siberia, Russia): implications for the paleogeography of Siberia:
Journal of Geophysical Research: Solid Earth, vol.~105,
pp.~13,509-13,518, doi:10.1029/1999JB900429.

Cogne, J.-P., Kravchinsky, V. A., Halim, N., and Hankard, F., 2005, Late
Jurassic-Early Cretaceous closure of the Mongol-Okhotsk Ocean
demonstrated by new Mesozoic palaeomagnetic results from the
Trans-Baikal area (SE Siberia): Geophysical Journal International,
vol.~163, pp.~813-832, doi:10.1111/j.1365-246X.2005.02782.x.

Edel, J. B., Schulmann, K., Hanzl, P., and Lexa, O., 2014,
Palaeomagnetic and structural constraints on 90º anticlockwise rotation
in SW Mongolia during the Permo-Triassic: Implications for Altaid
oroclinal bending. Preliminary palaeomagnetic results: Journal of Asian
Earth Sciences, vol.~94, pp.~157-171, doi:10.1016/j.jseaes.2014.07.039.

Evans, D. A., Zhuravlev, A. Y., Budney, C. J., and Kirschvink, J. L.,
1996, Palaeomagnetism of the Bayan Gol Formation, western Mongolia:
Geological Magazine, vol.~133, pp.~487-496.

Grishin, D., Didenko, A., Pechersky, D., and T.L., T., 1991,
{[}Paleomagnetic study and petromagnetic study of structure and
evolution of paleooceanic lithosphere (Phanerozoic ophiolites of
Asia){]} (in Russian), VNIGRI, Leningrad, Russia, pp.~135-149.

Halim, N., Kravchinsky, V., Gilder, S., Cogne, J.-P., Alexyutin, M.,
Sorokin, A., Courtillot, V., and Chen, Y., 1998, A palaeomagnetic study
from the Mongol-Okhotsk region: rotated Early Cretaceous volcanics and
remagnetized Mesozoic sediments: Earth and Planetary Science Letters,
vol.~159, pp.~133-145, doi:10.1016/S0012-821X(98)00072-7.

Kovalenko, D., 2010, Paleomagnetism of Late Paleozoic, Mesozoic, and
Cenozoic rocks in Mongolia: Russian Geology and Geophysics, vol.~51,
pp.~387-403, doi:10.1016/j.rgg.2010.03.006.

Kravchinsky, V., Konstantinov, K., and Conge, J., 2001, Palaeomagnetic
study of Vendian and Early Cambrian rocks of South Siberia and Central
Mongolia: Was the Siberian platform assembled at this time?: Precambrian
Research, vol.~110, pp.~61-92.

Kravchinsky, V. A., Cogne, J.-P., Harbert, W. P., and Kuzmin, M. I.,
2002, Evolution of the Mongol-Okhotsk Ocean as constrained by new
palaeomagnetic data from the Mongol-Okhotsk suture zone, Siberia:
Geophysical Journal International, vol.~148, pp.~34-57.

Kravchinsky, V. A., Sklyarov, E. V., Gladkochub, D. P., and Harbert, W.
P., 2010, Paleomagnetism of the Precambrian Eastern Sayan rocks:
Implications for the Ediacaran-Early Cambrian paleogeography of the
Tuva-Mongolian composite terrane: Tectonophysics, vol.~486, pp.~65-80,
doi:10.1016/j.tecto.2010.02.010.

Pechersky, D. and Didenko, A., 1995, Paleo-Asian Ocean (in Russian):
United Institute of Earth's Physics Publ., Moscow, 298 pp.

Pruner, P., 1992, Palaeomagnetism and palaeogeography of Mongolia from
the Carboniferous to the Cretaceous - final report: Physics of the Earth
and Planetary Interiors, vol.~70, pp.~169-177,
doi:10.1016/0031-9201(92)90179-Y.

Van Hinsbergen, D. J. J., Straathof, G. B., Kuiper, K. F., Cunningham,
W. D., and Wijbrans, J., 2008, No vertical axis rotations during Neogene
transpressional orogeny in the NE Gobi Altai: coinciding Mongolian and
Eurasian early Cretaceous apparent polar wander paths: Geophysical
Journal International, vol.~173, pp.~105-126,
doi:10.1111/j.1365-246X.2007.03712.x.




    % Add a bibliography block to the postdoc
    
    
    
    \end{document}
